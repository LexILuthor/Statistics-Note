\chapter{Statistics}
\vspace{15pt}



The notation of statistic was introduced by fisher (1920).\\
The importance of sufficiency is that ot can be found in any statistical decision (point estimation, testing, confidential bound)
\begin{defi}
	Let $X=(X_1... X_n)$ be a random sample from a parametric model $X\sim f_X(x,\theta)$ for some $\theta \in \Theta$ unknown.\\
	We say that $T_n=T(X)$ \textbf{sufficient for the parameter $\theta$} if the conditional distribution of $X$ given $T_n$ does not depend of $\theta$ i.e.
	\begin{itemize}
		\item $f_{\uX|T_n=t}(X;t,\theta)$ conditional distribution of $\uX$ given $T_n$
		\item $h_{\uX,T_n}(z,t,\theta)$ joint distribution of $\uX$ and $T_n$
		\item $g_{T_n}(t,\theta)$ marginal distribution of $T_n$
	\end{itemize}
then $T_n$ is \textbf{sufficient for the parameter $\theta$} only if $f_{\uX|T_n=t}(X;t,\theta)$ does not depend of $\theta$.\\
Note that
\[
f_{\uX|T_n=t}(X;t,\theta)=\frac{h_{\uX,T_n}(z,t,\theta)}{g_{T_n}(t,\theta)}
\]
\end{defi}
\begin{eg} \label{eg:ber}
	$(X_1... X_n)\in \{0,1\}^n$ from a $Ber(\theta)$, $\theta \in (0,1)$.\\
	Define $T_n=\sum_{i=1}^{n}X_i$ then we want to verify if $T_n$ is sufficient.\\
	\begin{itemize}
		\item $f_{\uX}(\ux;\theta) = \prod_{i=1}^{n} f_{\underline{X_i}}(\underline{x_i};\theta)=\theta^{\sum_{i=1}^{n}x_i}(1-\theta)^{n-\sum_{i=1}^{n}x_i}$
		\item$g_{T_n}(t,\theta)={{n}\choose{t}}\sigma^t(1-\sigma)^{n-t}\mathbbm{1}_{0,1...n}(t)$
		\item$h_{\uX,T_n(z,t,\theta)}\p(\uX=\ux,T_n=t)=\sigma^t(1-\sigma)^{n-t}$
	\end{itemize}
	so 
	$$f_{\uX|T_n=t}=\frac{\sigma^t(1-\sigma)^{n-t}}{{{n}\choose{t}}\sigma^t(1-\sigma)^{n-t}}=\frac{1}{{{n}\choose{t}}}$$
	So $T_n$ is a sufficient statistic for $\theta$.\\
	This is a really special case because all the $X_i$ are already in function of $T_n$.
\end{eg}

\begin{oss}
	If $T_n$ is sufficient for $\theta$ then all the statistical information of $\theta$ is contained in the random sample relocated in $T_n$. In the example above we just need $\sum_{i=1}^{n}X_i$.\\
\end{oss}
\begin{oss}
	The notation of sufficiency derive from the probability structure of the parametric family $X\sim f_X(x;\theta)$ We can talk about sufficiency for a parameter $\theta$ only after we have specified $X\sim f_X(x;\theta)$
\end{oss}
The definition of sufficiency based on conditional probability is not of practical use because we need this two distributions $
\begin{cases}
g_{T_n}(\cdot)\\
h_{\uX,T_n}(\cdot,\cdot)
\end{cases}
$ that can be difficult to find.
To avoid that there is a corollary of the \textit{Fisher Factorization Theorem}:
\begin{corol}\label{corol:Savage}
	Let $\uX=(X_1... X_n)$ from $X\sim f_x(x,\theta)$. Then a statistic $T_n$ is sufficient for $\theta$ if and only if there exist two non negative functions $g(\cdot), h(\cdot)$ such that $\lf=g(T(\uX);\theta)h(\uX)$
\end{corol}
\begin{oss}
	\begin{itemize}
		\item $g$ is a function of the observed sample via $T_n$
		\item $h$ is a function of the observed sample and does not depend on $\theta$
 	\end{itemize}
\end{oss}
\begin{eg}
	Recall The example \ref{eg:ber} 
		$(X_1... X_n)\in \{0,1\}^n$ from a $Ber(\theta)$, $\theta \in (0,1)$.\\
	Define $T_n=\sum_{i=1}^{n}X_i$ then we want to verify if $T_n$ is sufficient.
	\begin{itemize}
	 \item $f_{\uX}(\ux;\theta) = \prod_{i=1}^{n} f_{\underline{X_i}}(\underline{x_i};\theta)=\theta^{\sum_{i=1}^{n}x_i}(1-\theta)^{n-\sum_{i=1}^{n}x_i}$\\
	\item $h(\uX)=1$
	\item$g(T_n(\uX),\theta)=\theta^{\sum_{i=1}^{n}x_i}(1-\theta)^{n-\sum_{i=1}^{n}x_i}$
	\end{itemize}
\end{eg}
\begin{eg}\label{eg:gauss}
		$(X_1... X_n)\in \{0,1\}^n$ from a $N(\theta,1)$. We want to verify that $T_n=\sum_{i=1}^{n} X_i$ is a sufficient statistic
		\[
		\begin{split}
			\lf
			&=\prod_{i=1}^{n}\frac{1}{\sqrt{2 \pi}} \exp \bigg\{ -\frac{1}{2} (x_i-\theta)^2 \bigg\}\\
			&=(2\pi)^{-n/2}\exp \bigg\{ -\frac{1}{2} \sum_{i=1}^{n}(x_i-\theta)^2 \bigg\}\\
			&=(2\pi)^{-n/2}\exp \bigg\{ -\frac{1}{2} \sum_{i=1}^{n}x_i^2- \frac{n\theta^2}{2}+\theta\sum_{i=1}^{n} x_i \bigg\}\\
			&=(2\pi)^{-n/2}\exp \bigg\{ -\frac{1}{2} \sum_{i=1}^{n}x_i^2 \bigg\} \exp \bigg\{- \frac{n\theta^2}{2}+\theta\sum_{i=1}^{n} x_i \bigg\}
		\end{split}
		\]
		so
		\begin{itemize}
			\item $h(x)=\exp \bigg\{ -\frac{1}{2} \sum_{i=1}^{n}x_i^2 \bigg\}$
			\item $g(\sum_{i=1}^n,\theta) = \exp \bigg\{- \frac{n\theta^2}{2}+\theta\sum_{i=1}^{n} x_i \bigg\}$
		\end{itemize}
\end{eg}

\begin{teo}\textbf{Fisher Theorem}\\
	If $\lfd$ is the joint density function or the joint probability mass function of $\uX$ and $q(t;\theta)$ is the density function or the probability mass function of $T_n(\uX)$, then $T_n(\uX)$ is sufficient for $\theta$ if for eery point in the sample space, the ratio
	$$\frac{\lfd}{q(t;\theta)}$$ 
	is a constant function of $\theta$
\end{teo}
\begin{proof}
	MISSING
\end{proof}
We can see now the prof of the corollary \ref{corol:Savage}
\begin{corol}\textbf{Savage}
	Let $\lfd$ be the jonit PDF or PMFof a random sample $\uX=(X_1... X_n)$. A statistic $T_n$ is sufficient for $\theta$ if and only if there exist two non negative functions $g(t,\theta), h(\ux)$ such that for all $\ux$ in the sample space and for all $\theta\in \Theta$ 
	$$\lfd=g(T(\ux);\theta)h(\ux)$$
\end{corol}
\begin{proof}
	We re going to prove that in the discrete settings.\\
	\begin{itemize}
		\item[$\Rightarrow$] Suppose that $T(\uX)$ is sufficient for $\theta$.\\
		\begin{itemize}
			\item $g(t,\theta)=\p(T(\uX)=t)$
			\item $h(\ux)=\p(\uX=\ux|T(\uX)=T(\ux)$
		\end{itemize}
	Because $T(\uX)$ is sufficient for $\theta$ the conditional probability defining $h(\ux)$ does not depend on $\theta$. Hence the choice of $g(t,\theta)$ and $h(\ux)$ is legitimate and for this choice we have
	\[
	\begin{split}
	\p(\uX=\ux)
	&=\p(\uX=\ux \wedge T(\uX)=T(\ux))\\
	&=\p(T(\uX)=T(\ux))\p(\uX=\ux|T(\uX)=T(\ux)\\
	&=g(t,\theta)h(\ux)
	\end{split}
	\]
	So we have the factorization and in particular we can see that
	$$\p(T(\uX)=T(\ux))=g(t,\theta)$$
	$\implies g(T(\ux),\theta)$ is the PMF of $T(s)$
	\item[$\Leftarrow$] We assume that the factorization holds.\\
	Let $q(t,\theta)$ be the PMF of $T(\uX)$ we study the ratio
	$$\frac{\lfd}{q(T(\ux);\theta)}$$
	in particular define
	$$A_{T(\ux)}=\{\underline y | T(\underline y)= T(\ux) \}$$
	Then 
	\[
	\begin{split}
	\frac{\lfd}{q(T(\ux);\theta)}
	&=\frac{g(T(\ux);\theta)h(\ux)}{q(T(\ux);\theta)}\\
	&=\frac{g(T(\ux);\theta)h(\ux)}{\sum_{\underline y \in A_{T(\ux)}}{g(T(\ux);\theta)h(\underline y)}}\\
	&=\frac{g(T(\ux);\theta)h(\ux)}{g(T(\ux);\theta) {\sum_{\underline y \in A_{T(\ux)}}h(\underline y)}}\\
	&=\frac{h(\ux)}{\sum_{\underline y \in A_{T(\ux)}}h(\underline y)}
	\end{split}
	\]
	This is constant \wrt $\theta$.\\
	Then by the Fisher Theorem $T(\uX)$ is sufficient for $\theta$.
	\end{itemize}

\end{proof}

\begin{eg}In with similar condition of the example \ref{eg:gauss} we want to find if $T_n=\frac{1}{n}\sum_{i=1}^{n} X_i$ is a sufficient statistic for $\mu$.
	$(X_1... X_n)\in \{0,1\}^n$ from a $N(\mu,\sigma^2)$, $\sigma^2$ known.
	\[
	\begin{split}
	f_{\uX}(\ux;\mu\sigma^2)
	&=(2 \pi \sigma^2)^{-n/2} \exp \bigg\{ -\frac{1}{2\sigma^2}\sum_{i=1}^{n} (x_i-\mu)^2 \bigg\}\\
	&=(2 \pi \sigma^2)^{-n/2} \exp \bigg\{ -\frac{1}{2\sigma^2} \sum_{i=1}^{n}(x_i-\bar x_n -\bar x_n -\mu)^2 \bigg\}\\
	&=(2 \pi \sigma^2)^{-n/2} \exp \bigg\{ -\frac{1}{2\sigma^2}\bigg( \sum_{i=1}^{n} (x_i-\bar x_n)^2 + n(\bar x_n-\mu)^2  \bigg) \bigg\}\\
	\end{split}
	\]
	we already have the distribution of $T(\ux)=\bar x_n=\frac{1}{n}\sum_{i=1}^{n}x_i$ which is $\bar X_n\sim N(\mu,\sigma^2/n)$.
	so we try o apply Fisher theorem to the ratio:
	\[
	\frac{(2 \pi \sigma^2)^{-n/2} \exp \bigg\{ -\frac{1}{2\sigma^2}\bigg( \sum_{i=1}^{n} (x_i-\bar x_n)^2 + n(\bar x_n-\mu)^2  \bigg) \bigg\}}{(2 \pi \sigma^2/n)^{-1/2}\exp \bigg\{- \frac{n}{2\sigma^2} (\bar x_n -\mu)^2 \bigg\}}
	\]
	So by Fisher Theorem $T_n=\frac{1}{n}\sum_{i=1}^{n} X_i$ is sufficient for $\mu$
\end{eg}
