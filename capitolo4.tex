\chapter{Statistics}
\vspace{15pt}



The notation of statistic was introduced by Fisher (1920).\\
The importance of sufficiency is that ot can be found in any statistical decision (point estimation, testing, confidential bound)
\begin{defi}
	Let $X=(X_1... X_n)$ be a random sample from a parametric model $X\sim f_X(x,\theta)$ for some $\theta \in \Theta$ unknown.\\
	We say that $T_n=T(X)$ is \textbf{sufficient for the parameter $\theta$} if the conditional distribution of $X$ given $T_n$ does not depend of $\theta$ i.e. defined:
	\begin{itemize}
		\item $f_{\uX|T_n=t}(\ux;t,\theta)$ the conditional distribution of $\uX$ given $T_n$
		\item $h_{\uX,T_n}(z,t,\theta)$ the joint distribution of $\uX$ and $T_n$
		\item $g_{T_n}(t,\theta)$ the marginal distribution of $T_n$
	\end{itemize}
then $T_n$ is \textbf{sufficient for the parameter $\theta$} only if $f_{\uX|T_n=t}(\ux;t,\theta)$ does not depend of $\theta$.\\
Note that
\[
f_{\uX|T_n=t}(\ux;t,\theta)=\frac{h_{\uX,T_n}(\ux,t,\theta)}{g_{T_n}(t,\theta)}
\]
\end{defi}
\begin{eg} \label{eg:ber}
	$(X_1... X_n)\in \{0,1\}^n$ from a $Ber(\theta)$, $\theta \in (0,1)$.\\
	Define $T_n=\sum_{i=1}^{n}X_i$, then we want to verify if $T_n$ is sufficient.\\
	\begin{itemize}
		\item $f_{\uX}(\ux;\theta) = \prod_{i=1}^{n} f_{{X_i}}({x_i};\theta)=\theta^{\sum_{i=1}^{n}x_i}(1-\theta)^{n-\sum_{i=1}^{n}x_i}$
		\item$g_{T_n}(t,\theta)={{n}\choose{t}}\sigma^t(1-\sigma)^{n-t}\mathbbm{1}_{0,1...n}(t)$
		\item$h_{\uX,T_n(z,t,\theta)}=\p(\uX=\ux,T_n=t)=\sigma^t(1-\sigma)^{n-t}$
	\end{itemize}
	so 
	$$f_{\uX|T_n=t}=\frac{\sigma^t(1-\sigma)^{n-t}}{{{n}\choose{t}}\sigma^t(1-\sigma)^{n-t}}=\frac{1}{{{n}\choose{t}}}$$
	So $T_n$ is a sufficient statistic for $\theta$.\\
	This is a really special case because all the $X_i$ are already in function of $T_n$.
\end{eg}

\begin{oss}
	If $T_n$ is sufficient for $\theta$ then all the statistical information of $\theta$ contained in the random sample is relocated in $T_n$. In the example above to infer about $\theta$ we just need $\sum_{i=1}^{n}X_i$.\\
\end{oss}
\begin{oss}
	The notation of sufficiency derive from the probability structure of the parametric family $X\sim f_X(x;\theta)$. We can talk about sufficiency for a parameter $\theta$ only after we have specified $X\sim f_X(x;\theta)$
\end{oss}
The definition of sufficiency based on conditional probability is not of practical use because we need this two distributions $
\begin{cases}
g_{T_n}(\cdot)\\
h_{\uX,T_n}(\cdot,\cdot)
\end{cases}
$ that can be difficult to find.
To avoid that we could use a corollary of the \textit{Fisher Factorization Theorem}:
\begin{corol}\label{corol:Savage}
	Let $\uX=(X_1... X_n)$ from $X\sim f_x(x,\theta)$. Then a statistic $T_n$ is sufficient for $\theta$ if and only if there exist two non negative functions $g(\cdot), h(\cdot)$ such that $\lf=g(T(\ux);\theta)h(\ux)$
\end{corol}
\begin{oss}
	\begin{itemize}
		\item $g$ is a function of the observed sample via $T_n$
		\item $h$ is a function of the observed sample and does not depend on $\theta$
 	\end{itemize}
\end{oss}
\begin{eg}
	Recall the example \ref{eg:ber} 
		$(X_1... X_n)\in \{0,1\}^n$ from a $Ber(\theta)$, $\theta \in (0,1)$.\\
	Define $T_n=\sum_{i=1}^{n}X_i$, then we want to verify if $T_n$ is sufficient. \\
	We have
	 $$f_{\uX}(\ux;\theta) = \prod_{i=1}^{n} f_{{X_i}}(x_i;\theta)=\theta^{\sum_{i=1}^{n}x_i}(1-\theta)^{n-\sum_{i=1}^{n}x_i}$$
	hence we can apply the previous theorem \ref{corol:Savage} using 
	\begin{enumerate}
		\item $h(\uX)=1$ \item $g(T_n(\uX);\theta)=\theta^{\sum_{i=1}^{n}x_i}(1-\theta)^{n-\sum_{i=1}^{n}x_i}$
	\end{enumerate}
\end{eg}
\begin{eg}\label{eg:gauss}
		$(X_1... X_n)\in \{0,1\}^n$ from a $N(\theta,1)$. We want to verify that $T_n=\sum_{i=1}^{n} X_i$ is a sufficient statistic:
		\[
		\begin{split}
			\lf
			&=\prod_{i=1}^{n}\frac{1}{\sqrt{2 \pi}} \exp \bigg\{ -\frac{1}{2} (x_i-\theta)^2 \bigg\}\\
			&=(2\pi)^{-n/2}\exp \bigg\{ -\frac{1}{2} \sum_{i=1}^{n}(x_i-\theta)^2 \bigg\}\\
			&=(2\pi)^{-n/2}\exp \bigg\{ -\frac{1}{2} \sum_{i=1}^{n}x_i^2- \frac{n\theta^2}{2}+\theta\sum_{i=1}^{n} x_i \bigg\}\\
			&=(2\pi)^{-n/2}\exp \bigg\{ -\frac{1}{2} \sum_{i=1}^{n}x_i^2 \bigg\} \exp \bigg\{- \frac{n\theta^2}{2}+\theta\sum_{i=1}^{n} x_i \bigg\}
		\end{split}
		\]
		so
		\begin{itemize}
			\item $h(x)=\exp \bigg\{ -\frac{1}{2} \sum_{i=1}^{n}x_i^2 \bigg\}$
			\item $g(\sum_{i=1}^n x_i,\theta) = \exp \bigg\{- \frac{n\theta^2}{2}+\theta\sum_{i=1}^{n} x_i \bigg\}$
		\end{itemize}
\end{eg}

\begin{teo}\textbf{Fisher Theorem}\\
	If $\lfd$ is the joint density function or the joint probability mass function of $\uX$ and $q(t;\theta)$ is the density function or the probability mass function of $T_n(\uX)$, then $T_n(\uX)$ is sufficient for $\theta$ if for every point in the sample space, the ratio
	$$\frac{\lfd}{q(t;\theta)}$$ 
	is a constant function of $\theta$.
\end{teo}
\begin{proof}
	MISSING
\end{proof}
We can see now the prof of the corollary \ref{corol:Savage}
\begin{corol}\textbf{Savage}\\
	Let $\lfd$ be the joint PDF or PMF of a random sample $\uX=(X_1... X_n)$. A statistic $T_n$ is sufficient for $\theta$ if and only if there exist two non negative functions $g(t,\theta), h(\ux)$ such that for all $\ux$ in the sample space and for all $\theta\in \Theta$ 
	$$\lfd=g(T(\ux);\theta)h(\ux)$$
\end{corol}
\begin{proof}
	We are going to prove the theorem only in the discrete settings.\\
	\begin{itemize}
		\item["$\Rightarrow$"] Suppose that $T(\uX)$ is sufficient for $\theta$.\\
		Define:
		\begin{itemize}
			\item $g(t,\theta):=\p(T(\uX)=t)$
			\item $h(\ux):=\p\bigg(\uX=\ux \bigg|T(\uX)=T(\ux)\bigg)$
		\end{itemize}
	Because $T(\uX)$ is sufficient for $\theta$ the conditional probability defining $h(\ux)$ does not depend on $\theta$. Hence the choice of $g(t,\theta)$ and $h(\ux)$ is legitimate and for this choice we have
	\[
	\begin{split}
	\p(\uX=\ux)
	&=\p(\uX=\ux \wedge T(\uX)=T(\ux))\\
	&=\p(T(\uX)=T(\ux))\p(\uX=\ux|T(\uX)=T(\ux))\\
	&=g(t,\theta)h(\ux)
	\end{split}
	\]
	So we have the factorization and in particular we can see that
	$$\p\bigg(T(\uX)=T(\ux)\bigg)=g(t,\theta)$$
	$\implies g(T(\ux),\theta)$ is the PMF of $T(s)$
	\item["$\Leftarrow$"] We assume that the factorization holds.\\
	Let $q(t,\theta)$ be the PMF of $T(\uX)$. We study the ratio
	$$\frac{\lfd}{q(T(\ux);\theta)}$$
	in particular define
	$$A_{T(\ux)}=\{\underline y | T(\underline y)= T(\ux) \}$$
	Then 
	\[
	\begin{split}
	\frac{\lfd}{q(T(\ux);\theta)}
	&=\frac{g(T(\ux);\theta)h(\ux)}{q(T(\ux);\theta)}\\
	&=\frac{g(T(\ux);\theta)h(\ux)}{\sum_{\underline y \in A_{T(\ux)}}{g(T(\ux);\theta)h(\underline y)}}\\
	&=\frac{g(T(\ux);\theta)h(\ux)}{g(T(\ux);\theta) {\sum_{\underline y \in A_{T(\ux)}}h(\underline y)}}\\
	&=\frac{h(\ux)}{\sum_{\underline y \in A_{T(\ux)}}h(\underline y)}
	\end{split}
	\]
	This is constant \wrt $\theta$.\\
	Then by the Fisher Theorem $T(\uX)$ is sufficient for $\theta$.
	\end{itemize}

\end{proof}

\begin{eg}$(X_1... X_n)\in \{0,1\}^n$ from a $N(\mu,\sigma^2)$, $\sigma^2$ known. As we did in the example \ref{eg:gauss} we want to find if $T_n=\frac{1}{n}\sum_{i=1}^{n} X_i$ is a sufficient statistic for $\mu$.
	\[
	\begin{split}
	f_{\uX}(\ux;\mu\sigma^2)
	&=(2 \pi \sigma^2)^{-n/2} \exp \bigg\{ -\frac{1}{2\sigma^2}\sum_{i=1}^{n} (x_i-\mu)^2 \bigg\}\\
	&=(2 \pi \sigma^2)^{-n/2} \exp \bigg\{ -\frac{1}{2\sigma^2} \sum_{i=1}^{n}(x_i-\bar x_n -\bar x_n -\mu)^2 \bigg\}\\
	&=(2 \pi \sigma^2)^{-n/2} \exp \bigg\{ -\frac{1}{2\sigma^2}\bigg( \sum_{i=1}^{n} (x_i-\bar x_n)^2 + n(\bar x_n-\mu)^2  \bigg) \bigg\}\\
	\end{split}
	\]
	we already know the distribution of $T(\ux)=\bar x_n=\frac{1}{n}\sum_{i=1}^{n}x_i$ is $\bar X_n\sim N(\mu,\sigma^2/n)$.\\
	So we can apply Fisher theorem to the ratio:
	\[
	\frac{(2 \pi \sigma^2)^{-n/2} \exp \bigg\{ -\frac{1}{2\sigma^2}\bigg( \sum_{i=1}^{n} (x_i-\bar x_n)^2 + n(\bar x_n-\mu)^2  \bigg) \bigg\}}{(2 \pi \sigma^2/n)^{-1/2}\exp \bigg\{- \frac{n}{2\sigma^2} (\bar x_n -\mu)^2 \bigg\}}
	\]
		\[
	\frac{(2 \pi \sigma^2)^{-n/2} \exp \bigg\{ -\frac{1}{2\sigma^2} \sum_{i=1}^{n} (x_i-\bar x_n)^2   \bigg\} \exp \bigg\{ -\frac{n}{2\sigma^2}(\bar x_n-\mu)^2  \bigg\}}{(2 \pi \sigma^2/n)^{-1/2}\exp \bigg\{- \frac{n}{2\sigma^2} (\bar x_n -\mu)^2 \bigg\}}
	\]
	\[
	\frac{(2 \pi \sigma^2)^{-n/2} \exp \bigg\{ -\frac{1}{2\sigma^2} \sum_{i=1}^{n} (x_i-\bar x_n)^2   \bigg\} }{(2 \pi \sigma^2/n)^{-1/2}}
	\]
	Hence by Fisher Theorem $T_n=\frac{1}{n}\sum_{i=1}^{n} X_i$ is sufficient for $\mu$
\end{eg}
\begin{oss}
	Until now we found only one sufficient statistic for a fixed parametric model. However we can define many sufficient statistics.\\
	For example the statistic given by the identity $T(\ux)=\ux$ is always a sufficient statistic, indeed we can factorize the distribution $f_X(x,\theta)$ with
	\begin{itemize}
		\item $h(x)=1$
		\item $g(T(x);\theta)=f_X(x,\theta)$
	\end{itemize}
\end{oss}


\begin{oss}
 Given one sufficient statistic a way to produce more sufficient statistics is thru a one to one function.\\
Suppose $T(\ux)$ is a sufficient statistic for $\theta$, and define $T^*(\ux)=r(T(\ux))$ where $r$ is a one to one function with inverse $r^{-1}$.\\
By Savage's Theorem there exist $g,h$ such that
$$\lf=g(T(\ux),\theta)h(\ux)=g(r^{-1}(r(T(\ux),\theta)))h(\ux)=g(r^{-1}(T^*(\ux)),\theta)h(\ux)$$
So defining $g^*(t,\theta)=g(r^{-1}(t),\theta)$ we have that
$$\lf=g^*(T^*(\ux),\theta)h(\ux)$$
$\implies$ by Savage Theorem we have that $T^*(\ux)$ is a sufficient statistic.
\end{oss}
We saw that in principle we can define many sufficient statistics so it is natural to define a tool that allows us to decide when a sufficient statistic is better than another.\\
Recall that the purpose of statistic is to achieve data reduction without loss of information.\\
Therefore a statistic that achieve the most data reduction while still retaining all of the information about $\theta$ might be preferable.
\begin{oss}
	We saw in example \ref{eg:gauss} that if 	$(X_1... X_n)\in \{0,1\}^n$ from a $N(\theta,1)$,  $T_n=\sum_{i=1}^{n} x_i$ is a sufficient statistic. Instead of $\sum_{i=1}^{n} X_i$ we can use $T'(\ux)\bigg( \sum_{i=1}^{n} x_i, \sum_{i=1}^{n} x_i^2 \bigg)$. Clearly $T(X)$ s a greater data reduction than $T'(\ux)$ since we do not need to know the sample variance if we want to know $\theta$. Moreover we can write $T(\ux)$ as a function of $T'(\ux)$ by defining the function $r(a,b)=a$, then we can write
	$$T(\ux)=\bar x_n=r(\bar x_n, S^2_n)=r(T'(\ux))$$
	Since $T(\ux)$ and $T'(\ux)$ are both sufficient they contains the same information about $\mu$.
	In other terms the additional information given by the sample variance is null.
\end{oss} 
\begin{defi}
	A sufficient statistic $T(\ux)$ is called \textbf{minimal} if for any other sufficient statistic $T'(\ux)$, $T(\ux)$ is a function of $T'(\ux)$. 
\end{defi}
NOTE:\\
To say that $T(\ux)$  is a function of  $T'(\ux)$ simply means that if $T'(x)=T'(y)$ then $T(x)=T(y)$.\\
In other terms if $\{ B_t\} $ where $B_t:=\{t' : T'(t)=T'(t') \}$ is the partition set induced by $T'$ and $\{ A_t\} $ where $A_t:=\{t' : T(t)=T(t') \}$ is the partition set induced by $T$ then for every $t$, $B_t \subseteq A_t$.\\
$\implies$the partition of the sample space induced by a minimal statistic is the partition with the smallest cardinality.
\begin{teo}\textbf{Lehmann and Sheffe}
	let $\lfd$ be the joint density function or joint probability mass function of a \rs \  $\uX=(X_1...X_n)$. Suppose there exist a function $T$ such that for any two sample points $\ux$ , $\uy$ the ratio 
	\[
	\frac{\lfd}{f_{\uX}(\uy;\theta)}
	\]
	is constant as a function of $\theta$ if and only if $T(\ux)=T(\uy)$
	
	
	Then $T$ is a minimal sufficient statistic for $\theta$ 
\end{teo}

\begin{proof}
	To simplify the proof we assume $\lfd > 0 \ \forall \ux, \forall \theta$.\\
	First we show that $T(\ux)$ is sufficient.\\
	Define $\Tau$ as the image of the sample space under the function $\st$.
	$$\Tau:=\{ t:t=\st \text{for some $\ux$ in the sample space} \}$$
	Define $\{ A_t\} $ the partition set induced by $T$, where $A_t:=\{t' : T(t)=T(t') \}$ 
	For each $A_t$  choose and fix some elements $x_t\in A_t$. For any point in the space $\ux_{\st}$ is the fixed element that is in the same set ,$A_t$, as $\ux$. Since $\ux$ and $\ux_{\st}$ are in the same set $A_t$ then $\st=T(\ux_{\st})$ so by the assumptions the ratio
	\[
	\frac{\lfd}{f_{\uX}(\ux_{\st};\theta)}
	\]
Does not depend on $\theta$. Thus we can define $h(\ux):=\frac{\lfd}{f_{\uX}(\ux_{\st};\theta)}$.\\
Then define the function $g(\ux,\theta)\lfd$, so we have:
\[
\lfd =\frac{f_{\uX}(\ux_{\st};\theta)\lfd}{f_{\uX}(\ux_{\st};\theta)}=\gf h(\theta)
\]
and by Savage Theorem $\st$ is sufficient for $\theta$.\\


Now we will show that $\st$ is minimal sufficient.\\
Let  $T'(\ux)$ be another sufficient statistic. By Savage Theorem we know that exist $h',g'$ such that
\[
\lfd=g'(T'(\ux),theta)h'(\theta)
\]
Let $\ux,\uy$ be two sample points such that $T'(\ux)=T'(\uy)$ then we can study the ratio:
\[
\frac{\lfd}{f_{\uX}(\uy;\theta)}=\frac{g'(T'(\ux);\theta)h'(\ux)}{g'(T'(\uy);\theta)h'(\uy)}=\frac{h'(\ux)}{h'(\uy)}
\]
Since the ratio does not depend on $\theta$, by the assumption (the other implication of the IIF) implies $\st =T(\uy)$. So we can say that $T(\ux)$ is a function of $T'(\ux)$ therefore $\st$ is minimal.
\end{proof}
\section{Estimators}

\begin{defi}
	Suppose there is a fixed parameter $\theta$ that needs to be estimated. Then an \textbf{estimator} is a function that maps the sample space to a set of sample estimates. An estimator of $\theta$ is usually denoted by the symbol $\bar \theta$.
\end{defi}
Now we re going to introduce some definition of \textit{"good"} estimators.
\begin{defi}
	$T_n(\uX)$ is said to be \textbf{unbiased} for $\theta$ if $\e[T_n(\uX)]=\theta$
\end{defi}
\begin{oss}
	we use the expected value to define a "good"v estimator because of the linearity of the operator.
\end{oss}
\begin{defi}
	\textbf{Bias}:
	\[
		Bias_\theta (T_n(\uX))=\e\bigg[T_n(\uX)-\e[T_n(\uX)]\bigg]
	\]
\end{defi}

When we ask an estimator to be unbiased basically we are requiring it to be centred around $\theta$.\\
Another parameter that give us information about the goodness of an estimator is the variance. We can interpret the variance as a measure of the dispersion around the expected value, so before check the variance we mus be sure that the expected value overlap with $\theta$. In this scenario the less is the variance the best is the estimator.
\begin{oss}
	Variance is a good parameter to watch only if the estimator is unbiased
\end{oss}
To avoid this problem we can introduce the \textit{Mean Squared Error}
\begin{defi}\textbf{Mean Squared Error}(MSE)
	\[
	\e[(T_n(\uX)-\theta)^2]
	\]
\end{defi}
The importance of this quantity comes from the \textit{Chebyshev's Inequality} \ref{eq:Chebyshev}
\[
\p(|T_n(\ux)-\theta|< k)>1-\frac{\e[(T_n(\uX)-\theta)^2]}{k^2}
\]
Indeed we notice the smaller the MSE the greater is $\p(|T_n(\ux)-\theta|< k)$.
\begin{prop}
	$$\e[(T_n(\uX)-\theta)^2]=Var(T_n(\uX))+Bias_\theta(T_n(\uX))$$
\end{prop}
\begin{proof}
	\[
	\begin{split}
		\e[(T_n(\uX)-\theta)^2]
		&=\e[(T_n(\uX) - \e[T_n(\uX)] + \e[T_n(\uX)]-\theta)^2 ]\\
		&=\e[(T_n(\uX)-\e[T_n(\uX)])^2]+\e[(\e[T_n(\uX)]-\theta)^2]+2\e[(T_n(\uX)-\e[T_n(\uX)])(\e[T_n(\uX)]-\theta)]\\
		&=\e[(T_n(\uX)-\e[T_n(\uX)])^2]+\e[(\e[T_n(\uX)]-\theta)^2]\\
		&=Var(T_n(\uX))+Bias_\theta(T_n(\uX))
	\end{split}
	\]
\end{proof}
\begin{oss}
	If $\e[T_n(\uX)]=\theta$ then $MSE(T_n(\uX))=Var(T_n(\uX))$.
\end{oss}
\begin{defi}
	Let $X_1.. X_n$ from $X\sim f_X(x,\theta)$, $T_n'$ and $T_n''$ estimators for $\theta$. We say $T_n'$ is  \textbf{more efficient} than $T_n''$ if
	$$MSE(T_n')<MSE(T_n'')$$
\end{defi}
Usually we choose the estimator with the lower MSE even if it is biased.\\

\section{Properties of Estimators}
The problem of the MSE is that we can note be sure that there exist $T_n$ such that $MSE (\tn)$ is the lowest possible.\\
A solution for this comes from
\begin{teo}\label{teo:Cramer-Rao Bound}
	\textbf{Cramer-Rao Bound}\\
			Let $X=(X_1... X_n)$ be a random sample from a parametric model $X\sim f_X(x,\theta)$.\\
			Then, under condition of regularity, for any estimator $\tn$ of $\theta$ 
			\[
			Var(\tn)\geq \frac{[1+b'(\tn)]^2}{\ifn}
			\]
Where
\begin{itemize}
	\item[$b(\tn)$] is the bias of $\tn$
	\item[$\ifn$] is the Fisher Information
\end{itemize}
\end{teo}
\begin{proof}
	consider the estimator $\tn$.
	\begin{itemize}
		\item $\e[\tn]=\theta +b(\theta)$
		\item $\der \e[\tn]=1+b'(\theta)$
		\item $\e[V_n'(\theta)]=0 \leftarrow$ because we suppose our model regular
	\end{itemize}
$\implies Cov(\tn, V_n'(\theta))=\e[\tn V_n'(\theta)] - \e[\tn]\e[V_n'n]=\e[\tn V_n'(\theta)]$ 
\[
\begin{split}
\e[\tn T_n'(\theta)]&
=\int_{\mathbb{R}^n}\tn V_n'(\theta) \lfd d\ux\\
&=\int_{\mathbb{R}^n}\tn \frac{f_{\uX}'(\ux,\theta)}{\lfd}\lfd d\ux\\
&=\int_{\mathbb{R}^n}\tn \der \lfd d\ux\\
&=\der \int_{\mathbb{R}^n}\tn \lfd d\ux\\
&=\der \e[\tn]\\
&=1+b'(\theta)
\end{split}
\]
So
\[
Cov(\tn,V_n'(\theta))=1+b'(\theta)
\]
We know that in general for $X,Y$ \rv s such that $\e[X]=\mu, \e[Y]=\nu, \e[X^2]<\infty, \e[Y^2]<\infty$ it holds
\[
\begin{split}
(Cov(X,Y))^2&=\bigg( \e[(X-\mu)(T-\nu)] \bigg)^2\\
&\leq  \e[(X-\mu)]^2 \e[(T-\nu)]^2\\
&= Var(X)Var(Y)
\end{split}
\]
So replacing $X$ with $\tn$ and $Y$ with $V_n'(\theta)$ we  obtain
\[
\begin{split}
Var(\tn)&\geq \frac{(Cov(\tn,V_n'(\theta)))^2}{Var(V_n'(\theta))}\\
&=\frac{(Cov(\tn,V_n'(\theta)))^2}{\e[(V_n'(\theta)-\e[V_n'(\theta)])^2]}\\
&=\frac{(Cov(\tn,V_n'(\theta)))^2}{\e[V_n'(\theta)]^2}\\
&=\frac{(Cov(\tn,V_n'(\theta)))^2}{\ifn}
\end{split}
\]
\end{proof}
\begin{corol}
	Under condition of regularity
	\[MSE \geq \frac{(1+b'(\theta))^2}{\ifn} +b^2(\theta) \]
\end{corol}
\begin{proof}
	Directly from Cramer-Rao Bound \ref{teo:Cramer-Rao Bound} remembering that $MSE(\tn)=Var(\tn)+b^2(\theta)$
\end{proof}

\begin{corol}
	Let $\uX=(X_1... X_n)$ be a random sample from a regular model $X\sim f_X(x;\theta)$\\
	If there exist a unbiased estimator for $\theta$ whose variance is equal to the Cramer-Rao bound, Then $\tn$ is unique
\end{corol}
\begin{proof}
	Take $T_{1n}$, $\tnd$ be unbiased estimators for $\theta$ such that
	\[
	Var(\tnu)=Var(\tnd)=\frac{1}{\ifn}
	\]
	Define $\tn := \frac{\tnu + \tnd}{2}$\\
	$\e[\tn]=\frac{\e[\tnu]+\e[\tnd]}{2}=\frac{2}{2}\theta=\theta$\\
	$\implies \tn$ is also unbiased for $\theta$\\
	$\implies Var(\tn)\geq \frac{1}{\ifn}$
	\[
	\begin{split}
	Var(\tn)
	&=Var\bigg( \frac{\tnu + \tnd}{2}  \bigg)\\
	&=\frac{1}{4}\bigg[ Var(\tnu)+ Var(\tnd) +2 Cov(\tnu, \tnd) \bigg]\\
	&=\frac{1}{4 }\bigg[ Var(\tnu)+ Var(\tnd) +2 Cov(\tnu, \tnd) \bigg]\frac{[Var(\tnu)Var(\tnd)]^{1/2}}{[Var(\tnu)Var(\tnd)]^{1/2}}\\
	&=\frac{(1+Corr(\tnu,\tnd))}{2}\frac{1}{\ifn}
	\end{split}
	\]
	\\MEMO:$|Corr(X,Y)|\leq 1$\\
	Because $Var(\tn)\leq\frac{1}{\ifn}$ then we must have  $Corr(\tnu,\tnd)\geq 1 \implies Corr(\tnu,\tnd)= 1$.\\
	$\implies \tnd=a+b\tnu$.\\
	Hence we must have $\theta=\e[\tnu]=\e[a+b\tnu]=\e[a]+b\e[\tnu]=a+b\theta\implies a=0,b=1$\\
	$\implies \tnd=\tnu$
\end{proof}
\begin{defi}
	Consider a regular model $X\sim f_x(x\theta)$. We say that an unbiased estimator $\tn$ whose variance is 
	\[
	Var(\tn)=\frac{1}{\ifn}
	\]
	is \textbf{efficient} moreover we define the \textbf{efficiency} of an estimator as:
	\[
	Eff(\tn)=\frac{1}{Var(\tn)\ifn} \ \ \ \ \ \ \in [0,1]
	\]
\end{defi}
\begin{oss}
	\begin{enumerate}
		\item We introduce (absolute) efficiency at the cost of assuming regularity for the parametric model.
		\item The variance of an unbiased estimator $\tn$ of $\theta$ can not be smaller than the Cramer-Rao bound. However we do not know if there exist an estimator whose variance is equal to the Cramer-Rao bound.
		\item The proper lower bound involves the MSE
		$$MSE(\tn)\geq\frac{[1+b'(\theta)]}{\ifn}+b^2(\theta)$$ 
\end{enumerate}
\end{oss}
\begin{prop}
	Let $\uX=(X_1...X_n)$ be a random sample from a parametric model  $X\sim f_x(x\theta)$. Let $\tn$ be a unbiased estimator for $\theta$. Than $\tn$ i efficient for $\theta$ if and only if
	$$V_n'(\theta)=\ifn (T_n-\theta)$$
\end{prop}
\begin{proof}
	\[
	\begin{split}
	V_n'(\theta)=\ifn (\tn-\e[\tn])&
	\iff 	V_n'(\theta)^2=\ifn^2 (\tn-\e[\tn])^2\\
	&\iff 	\e[V_n'(\theta)^2]=\e[\ifn^2 (\tn-\e[\tn])^2]\\
	&i.e. \ \ifn=\ifn^2 \e[(\tn-\e[\tn])^2]\\
	&i.e. \ 1=\ifn Var(\tn)\\
	&\iff Var(\tn)=\frac{1}{\ifn}
	\end{split}
	\]
	i.e. $\tn$ is efficient.
\end{proof}
\begin{teo}
	\textbf{Rao-Blackwell}\\
	Let $\uX=(X_1 ... X_n)$ a random sample from a parametric model  $X\sim f_x(x\theta)$. Let
	\begin{itemize}
		\item $\tnu$ a sufficient estimator for $\theta$
		\item $\tnd$ a unbiased estimator for $\theta$
		\item $\tn:=\e[\tnd|\tnu]$
	\end{itemize}
Then
\begin{enumerate}
	\item $\tn$ is a function of $\tnu$
	\item $\e[\tn]=\theta$
	\item $Var(\tn)<Var(\tnd)$
\end{enumerate}
\end{teo}

\begin{defi}
	Let $X_1, X_2...$ real valued  \rv s with CDF $F_{X_1},F_{X_2}...$. We say that $(X_n)_n$ \textbf{converges in distribution} or \textbf{converges weakly} to a \rv  \  $X$ if $$\lim_{n\to \infty}F_{X_n}(x)=F_X(x)$$
\end{defi}
\begin{defi}
	A sequence $\{ X_n\}_n$ of \rv s \textbf{converges in probability} towards the \rv \ $X$ if for all $\epsilon>0$
	$$\lim_{n\to \infty}\p(|X_n-X|>\epsilon)=0$$
	We write $X_n\xrightarrow{p}X$
\end{defi}
\begin{defi}
	Given a real number $r \geq 1$, we say that the sequence $\{X_n\}$ converges in the \textbf{r-th mean} (or \textbf{in the $L^r$-norm}) towards the random variable $X$, if the r-th absolute moments $\e(|X_n|^r)$ and $\e(|X|^r)$ of $X_n$ and $X$ exist, and
	$$\lim_{n\to \infty}\e[|X_n-X|^r]=0$$
	We write $X_n \xrightarrow{L^r} X$
	For $r=2$ we say that $\{X_n\}$ converges in \textbf{mean square}  to $X$.
\end{defi}

\begin{prop}
	Convergence in probability $\implies$ convergence in distribution.\\
	If $X$ is a degenerate \rv \ we have also\\  convergence in distribution $\implies$ convergence in probability.
\end{prop}
\begin{defi}
	We say an estimator $\tn$ is \textbf{is consistent in mean squared} for $\theta$ if 
	$$\lim_{n\to \infty}MSE(\tn)=0$$
\end{defi}
NOTATION: we will use $b(\tn)=b(\theta)$
\begin{oss}
	Since $MSE(\tn)=Var(T_n)+b^2(\tn)$, then then $\lim_{n\to \infty}MSE(\tn)=0$ is equivalent to say
	\begin{itemize}
		\item  $\lim_{n\to \infty}Var(\tn)=0$
		\item $\lim_{n\to \infty}b^2(\tn)=0$
	\end{itemize}
\end{oss}
\begin{defi}
	$T_n$ is \textbf{asymptotically unbiased} for $\theta$ if
	\begin{itemize}
		\item $\lim_{n\to \infty}\e[\tn]=\theta$
		\item $\lim_{n\to \infty}b(\tn)=0$
	\end{itemize}
\end{defi}
\begin{prop}
	A consistent estimator in mean square is also asymptotically unbiased
\end{prop}

\begin{defi}
	We say $\tn$ is consistent in probability for $\theta$ if for all $\epsilon >0$
	$$\lim_{n\to \infty}\p(|\tn-\theta|<\epsilon)=1$$
\end{defi}
\begin{prop}
	The  consistencyin mean squared implies consistency in probability 
\end{prop}
\begin{proof}
	Using Chebyshev's inequality \ref{eq:Chebyshev}
	$$\p(|\tn-\theta|<\epsilon)\geq 1-\frac{MSE(\tn)}{\epsilon^2}$$
	$$\lim_{n\to \infty}\p(|\tn-\theta|<\epsilon)\geq 1-\lim_{n\to \infty}\frac{MSE(\tn)}{\epsilon^2}$$
\end{proof}
\begin{teo}
	\textbf{Central Limit Teorem}\\
	Let $\uX=(X_1... X_n)$ be a random sample of size $n$ with $X_i$ independent and identically distributed \rv s. With  expected value $\mu$ and finite variance $\sigma^2$. Then $S_n:=\frac{\sum_{i=1}^n}{n}$ converges in probability to the expected value $\mu$
	$$S_n\xrightarrow{p} \mu$$
\end{teo}

\begin{teo}
	\textbf{Week law of large numbers}\\
	Let $\uX=(X_1... X_n)$ be a random sample of size $n$ with $X_i$ independent and identically distributed \rv s. With  expected value $\mu$.
	Define $\bar X_n=\frac{1}{n}\sum_{i=1}^{n}X_i$, then
	$$\bar X_n \xrightarrow{p} \mu$$
	That is, for any $\epsilon>0$
	$$\lim_{n\to \infty}\p(|\bar X_n- \mu|>\epsilon)=0$$
\end{teo}


	\section{Method if Moments}

\begin{oss}
	Before we start, like the likelihood method, this method can also be done for irregular models.
\end{oss}


Let $X=(X_1,\ldots,X_n)$ be a random sample from a parametric model $x\tilde f_x(x,\theta)$.
Assume that $\theta\in\Theta$ s.t. $\Theta$ has dimension $r$. In order to apply the method of moment for estimating $\theta$ we need: 
\begin{itemize}
	\item
	A condition related to the dimension of $\Theta$. If the number of parameters that we need to estimate is $r$, then the moments of $X$ must exist up to the order $r$.
	$$E[{|x|}^r]<\infty.$$
	\item
	Explicit expression for the first $r$ moments of $X$.
\end{itemize}
Given that, the method of moments exists in solving the system of equations given by:
$$\frac{1}{n}=\sum_{i=1}^{n}{X_i}^{J}=E[X^{J}], \quad \forall\enspace J=1,\ldots,r$$
\begin{oss}
	We don't have any result on the theoretical qualities on this method for the simulation. We need to check case by case.
\end{oss}

