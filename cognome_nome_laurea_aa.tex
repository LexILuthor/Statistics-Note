%%%%%%%%%%%%%%%%%%%%%%%%%%%%%%%%%%%%%%%%%%%%%%%
%
% Template per Elaborato di Laurea

%
% 
%
% Per la generazione corretta del 
% pdflatex nome_file.tex
% bibtex nome_file.aux
% pdflatex nome_file.tex
% pdflatex nome_file.tex
%
%%%%%%%%%%%%%%%%%%%%%%%%%%%%%%%%%%%%%%%%%%%%%%%


%\documentclass[epsfig,a4paper,11pt,titlepage,twoside,openany]{book} %formato FRONTE RETRO
\documentclass[epsfig,a4paper,11pt,titlepage,oneside,openany]{book} %formato SOLO FRONTE


\usepackage{fancyhdr}  %%capitolo a inizio pagina
\usepackage{graphicx}
\usepackage{mathtools}
\usepackage{lipsum}
\usepackage{caption}
\usepackage{epsfig}
\usepackage{plain}
\usepackage{setspace}
\usepackage[verbose]{hyperref}  %%Indice con link
\usepackage{blindtext}
%\usepackage[italian]{babel}
\usepackage[british,UKenglish,USenglish,english,american]{babel}
\usepackage{enumitem}
\usepackage[paperheight=29.7cm,paperwidth=21cm,outer=1.5cm,inner=2.5cm,top=2cm,bottom=2cm]{geometry} % per definizione layout
\usepackage{titlesec} % per formato custom dei titoli dei capitoli



% supporto lettere accentate
%\usepackage[utf8x]{inputenc} % per Linux (richiede il pacchetto unicode);
\usepackage[latin1]{inputenc} % per Windows;
%\usepackage[applemac]{inputenc} % per Mac.

\singlespacing

%%%%%%%%%%%%%%%% immagini
\usepackage{graphicx}
\graphicspath{ {images/} }
\usepackage{tikz}

%pacchetti di matematica

\usepackage{amsthm}
\usepackage{amsmath}
\usepackage{amssymb}
\usepackage{graphicx}
\usepackage[T1]{fontenc}
\newtheorem{teo}{Theorem}[section]
\newtheorem{corol}{Corollary}[section]
\newtheorem{prop}{Proposition}[section]
\newtheorem{eg}{Example}[section]
\newtheorem{ex}{Exercise}
\newtheorem{lem}{Lemma}[section]
\theoremstyle{definition}
\newtheorem{defi}{Definition}[section]
\newtheorem{oss}{Observation}
\numberwithin{equation}{section}
\usepackage{mathtools}
\usepackage{systeme}
\usepackage{bbm}
%%%%%%%%%%%%%%%%




%%%%%%%%%%%%%%%%%
\newcommand{\iid}{independent and identically distributed }
\newcommand{\rs}{random sample }
\newcommand{\ux}{\underline x}
\newcommand{\uy}{\underline y}
\newcommand{\uX}{\underline X}
\newcommand{\gf}{g(T(\underline x),\theta)}
\newcommand{\Tau}{\mathcal{T}}
\newcommand{\st}{T(\underline{x})}
\newcommand{\nor}{N(\mu, \sigma^2)}
\newcommand{\lf}{\mathcal{L}_n(\theta,\underline x)}
\newcommand{\lfd}{f_{\underline X}(\underline{x};\theta)}
\newcommand{\ifn}{\mathcal{I}_n(\theta)}
\newcommand{\rsf}{$(x_1...x_n)$ from $X\sim F_X(x;\theta)$}
\newcommand{\spacep}{$(\Omega,\mathcal{F} ,\mathbb{P})$}
\newcommand{\p}{\mathbb{P}}
\newcommand{\lep}{legge di probabilit� $\{p_1...p_n \}$ }
\newcommand{\xii}{$(x_1...x_n)$}
\newcommand{\leggeq}{legge di probabilit� $\{q_1...q_m \}$ }
\newcommand{\va}{$\{a_1...a_n \}$ }
\newcommand{\vb}{$\{b_1...b_m \}$ }
\newcommand{\lepc}{$\{p(j|i); 1 \leq i \leq n , \ 1 \leq j \leq m \}$ }
\newcommand{\wrt}{with respect to }
\newcommand{\e}{\mathbb{E}}
\newcommand{\Gaus}{\frac{1}{\sqrt{2\pi \sigma ^2}} \exp  \Big\{ - \frac{(x-\mu)^2}{2 \sigma^2} \Big\}}
\newcommand{\acode}{$\{ c_1...c_r \}$}
\newcommand{\rv}{random variable}
\newcommand{\var}{due variabili casuali $X,Y$}
\newcommand{\sumj}{\sum_{j=1}^n}
\newcommand{\sumi}{\sum_{i=1}^N}
\newcommand{\sumin}{\sum_{i=1}^n}
\newcommand{\sumk}{\sum_{k=1}^m}
\newcommand{\suma}{\sum_{i_0...i_n=1}^N}
\newcommand{\sumaa}{\sum_{i_0...i_n,i_{n+1}=1}^N}
\newcommand{\limi}{\lim_{n\to \infty}}
\newcommand{\sumij}{-\sum_{i,j=1}^n}
\newcommand{\sume}{\sum_{k=0}^{[mp]}  {m \choose k} }
\newcommand{\g}{ \frac{1}{\sigma (2\pi)^{1/2}} \exp \bigg( - \frac{1}{2} \bigg( \frac{x-\mu}{ \sigma} \bigg)^2 \bigg)  }
\newcommand{\sig}{\mathcal{B}(y,r)}
\begin{document}



  % nessuna numerazione
  \pagenumbering{gobble} 
  \pagestyle{plain}

\thispagestyle{empty}

\begin{center}
 

  \vspace{2 cm} 

  \LARGE{Dipartimento Matematica\\}

  \vspace{1 cm} 


  \vspace{2 cm} 
  \Large\textsc{Lecture Notes for Statistics\\} 
  \vspace{1 cm} 
  %\Huge\textsc{Statistics\\}
  %\Large{\it{Sottotitolo (alcune volte lungo - opzionale)}}


  \vspace{2 cm} 
  \begin{tabular*}{\textwidth}{ l @{\extracolsep{\fill}} l }
  \Large{Professore} & \Large{Laureando}\\
  \Large{Stefano Favaro}& \Large{Claudio Meggio}\\
  \end{tabular*}

  \vspace{2 cm} 

  \Large{Academic Year 2018/2019}
  
\end{center}


  
  
  
  
  



    
		%%%%%%%%%pagina bianca
	
%%%%%%%%%%%%%%%%%%%%%%%%%%%%%%%%%%%%%%%%%%%%%%%%%%%%%%%%%%%%%%%%%%%%%%%%%%
%%%%%%%%%%%%%%%%%%%%%%%%%%%%%%%%%%%%%%%%%%%%%%%%%%%%%%%%%%%%%%%%%%%%%%%%%%
%% Nota
%%%%%%%%%%%%%%%%%%%%%%%%%%%%%%%%%%%%%%%%%%%%%%%%%%%%%%%%%%%%%%%%%%%%%%%%%%
%% Sezione Ringraziamenti opzionale
%%%%%%%%%%%%%%%%%%%%%%%%%%%%%%%%%%%%%%%%%%%%%%%%%%%%%%%%%%%%%%%%%%%%%%%%%%
%%%%%%%%%%%%%%%%%%%%%%%%%%%%%%%%%%%%%%%%%%%%%%%%%%%%%%%%%%%%%%%%%%%%%%%%%%
%  \thispagestyle{empty}

\begin{center}
  {\bf \Huge Ringraziamenti}
\end{center}

\vspace{4cm}


\emph{
  ...thanks to...
}

  \clearpage
  \pagestyle{plain} % nessuna intestazione e pie pagina con numero al centro

  
  % inizio numerazione pagine in numeri arabi
  \mainmatter

%%%%%%%%%%%%%%%%%%%%%%%%%%%%%%%%%%%%%%%%%%%%%%%%%%%%%%%%%%%%%%%%%%%%%%%%%%
%%%%%%%%%%%%%%%%%%%%%%%%%%%%%%%%%%%%%%%%%%%%%%%%%%%%%%%%%%%%%%%%%%%%%%%%%%
%% Nota
%%%%%%%%%%%%%%%%%%%%%%%%%%%%%%%%%%%%%%%%%%%%%%%%%%%%%%%%%%%%%%%%%%%%%%%%%%
%%
%% Nel conteggio delle facciate sono incluse 
%%   indice
%%   sommario
%%   capitoli
%% Dal conteggio delle facciate sono escluse
%%   frontespizio
%%   ringraziamenti
%%   allegati    
%%%%%%%%%%%%%%%%%%%%%%%%%%%%%%%%%%%%%%%%%%%%%%%%%%%%%%%%%%%%%%%%%%%%%%%%%%
%%%%%%%%%%%%%%%%%%%%%%%%%%%%%%%%%%%%%%%%%%%%%%%%%%%%%%%%%%%%%%%%%%%%%%%%%%

    % indice
    \tableofcontents
    \clearpage
    
    
          
    % gruppo per definizone di successione capitoli senza interruzione di pagina
    \begingroup
      % nessuna interruzione di pagina tra capitoli
      % ridefinizione dei comandi di clear page
      \renewcommand{\cleardoublepage}{} 
      \renewcommand{\clearpage}{} 
      % redefinizione del formato del titolo del capitolo
      % da formato
      %   Capitolo X
      %   Titolo capitolo
      % a formato
      %   X   Titolo capitolo
      
      \titleformat{\chapter}
        {\normalfont\Huge\bfseries}{\thechapter}{1em}{}
        
      \titlespacing*{\chapter}{0pt}{0.59in}{0.02in}
      \titlespacing*{\section}{0pt}{0.20in}{0.02in}
      \titlespacing*{\subsection}{0pt}{0.10in}{0.02in}
      
      
    
      
      
      % sommario
      \chapter*{Inlformazioni sul corso} % senza numerazione
\label{sommario}
\vspace{15pt}

\addcontentsline{toc}{chapter}{Sommario} % da aggiungere comunque all'indice
mail: stefano.favaro@unito.it\\
book: \cite{Casella}

\clearpage



%%%%%%%%%%%%%%%%%%%%%%%%%%%%%%%%%%%%%%%%%%%%%%%%%%%%%%%%%%%%%%%%%%%%%%%%%%
%%%%%%%%%%%%%%%%%%%%%%%%%%%%%%%%%%%%%%%%%%%%%%%%%%%%%%%%%%%%%%%%%%%%%%%%%%
%% Nota
%%%%%%%%%%%%%%%%%%%%%%%%%%%%%%%%%%%%%%%%%%%%%%%%%%%%%%%%%%%%%%%%%%%%%%%%%%
%% Sommario e' un breve riassunto del lavoro svolto dove si descrive 
%% l’obiettivo, l’oggetto della tesi, le metodologie e 
%% le tecniche usate, i dati elaborati e la spiegazione delle conclusioni 
%% alle quali siete arrivati.
%% Il sommario dell’elaborato consiste al massimo di 3 pagine e deve contenere le seguenti informazioni: 
%%   contesto e motivazioni
%%   breve riassunto del problema affrontato
%%   tecniche utilizzate e/o sviluppate
%%   risultati raggiunti, sottolineando il contributo personale del laureando/a
%%%%%%%%%%%%%%%%%%%%%%%%%%%%%%%%%%%%%%%%%%%%%%%%%%%%%%%%%%%%%%%%%%%%%%%%%%
%%%%%%%%%%%%%%%%%%%%%%%%%%%%%%%%%%%%%%%%%%%%%%%%%%%%%%%%%%%%%%%%%%%%%%%%%%      
      
      %%%%%%%%%%%%%%%%%%%%%%%%%%%%%%%%
      % lista dei capitoli
      %
      % \input oppure \include
      %
      
	  \newpage
      \chapter{Introduction}
\label{cha:intro}
\vspace{15pt}




\section{Inequalities}
\label{sec:inequalities}
\vspace{10pt}
\textbf{Markow's Inequality}:\\
Let $Y$ be a non negative random variable with finite expected value then
$$ \mathbb{P}(Y \geq t)\leq \frac{\mathbb{E}[Y]}{t}$$
\textbf{Chebyshev's Inequality}:\\
Let $X$ be a random variable with finite second moment and let $\sigma=\sqrt{Var(x)}$, then for any positive real h
\begin{equation}\label{eq:Chebyshev}
	\mathbb{P}(|X-\mathbb{E}[X]|\geq h\sigma)\leq \frac{1}{h^2}
\end{equation}
\begin{teo}\textbf{Schwartz}
If $X,Y$ are random variables with finite second moment then:
$$ ( \mathbb{E} [XY])^2 \leq \e [X^2] \e [Y^2]$$
\end{teo}

If $X$ is a random variable taking values in a set $I$ with $\e [X]=\mu$  and $f()$ is convex on $I$, then $f(x)\geq f(\mu) + h(x-\mu)$ holds with probability 1 for some choice of h.
By integrating both sides of the inequality \wrt the distribution of X we obtain:
$$\e [f(x)]\geq f(\e [X]) \textbf{     (Jensen's Inequality)}$$

\section{Common distributions}
\label{sec:dist}
\textbf{Gaussian}:\\
A continuous random variable $Y$ is said to have a Gaussian distribution with parameters $\mu$ and $\sigma ^ 2$ if the density function at $t$ is:
$$\Gaus$$


Y is unimodal and symmetric around the mode $t=\mu$ and we write
$$Y \sim  N(\mu , \sigma ^2)$$
It's \textit{characteristic function} is
$$\e [e^{ tiy }]=\exp \Big\{ it \mu -\frac{\sigma ^2 t ^2}{2} \Big\}$$
The derivative of  the characteristic function valued in $t=0$ give us the non centred moments of Y.\\

If $Y\sim N(\mu \sigma ^2)$ and $a,b \in \mathbb{R}$ then $(a+bY) \sim N (a+b\mu, b^2 \sigma ^2)$. This mean that the entire family of distribution can be generated by linear transformations starting from any member of the family (i.e. is a \textit{location scale family})\\


If $Y_1 \sim N( \mu_1 , \sigma_1^2)$ and $Y_2 \sim N( \mu_2 , \sigma_2^2)$ and $ Y_1 \amalg Y_2$ then $Y_1+Y_2 \sim N (\mu _1 + \mu_2 , \sigma_1^2 + \sigma_2^2)$.\\
This result can be extended to linear combinations of Gaussian \rv s.\\
\\
\\
\\
\textbf{Uniform distribution:}\\
A continuous random variable can $Y$ with density function:
$$f(t;a,b)=\frac{1}{b-a} \mathbbm{1}_[a,b](t)$$
is said to be \textit{uniformly distributed} in $[a,b]$ and we write $Y\sim U(,b)$.\\
\begin{teo}\textbf{Integral transformation theorem}\\
If $Z$ is a continuous \rv \  with distribution function $F$ then the \rv 
$$W:=F(Z) \sim U(0,1)$$
\end{teo}
\begin{proof}
	\[
	\begin{split}
	\mathbb{P}(W\leq t)
	&=\p (F(Z)\leq t)\\
	&=\p (Z\leq F^{-1}(t))\\
	&=F(F^{-1}(t))\\
	&=t
	\end{split}
	\]
	Which is the distribution function of uniform in $[0,1]$
	
\end{proof}
\textbf{Gamma distribution:}\\
The \textit{Gamma function} is:
$$\Gamma (x):= \int_{0}^{+ \infty}t^{x-1}e^{-t}dt$$
Some properties of this function are:
\begin{itemize}
	\item $\Gamma (x+1)= x \Gamma(x)$
	\item if $x$ is a positive integer $\Gamma(x)=(x-1)!$
	\item $\Gamma(1)=1$
	\item $\Gamma \bigg( -\frac{1}{2} \bigg)=\sqrt{\pi}$
\end{itemize}
\textit{Stirling	's Approximation}
$$n!\sim \sqrt{2\pi n}\bigg(\frac{n}{e} \bigg)^n$$

\begin{defi}
	We say that a continuous \rv \  $X$ has a \textit{Gamma distribution} with shape parameter $w$ and scale parameter $\lambda$ ($X\sim Gamma(w,\lambda)$) if its density function is:
	$$f(t;w,\lambda)=\frac{\lambda^w}{\Gamma(w)}t^{w-1}e^{-\lambda t} \mathbbm{1}_{[\mathbb{R}^+]}(t)$$
\end{defi}

\begin{prop}
If $Y_1\sim Gamma(w_1,\lambda)$ and $Y_2\sim Gamma(w_2,\lambda)$ and $Y_1 \coprod Y_2$ then:
$$Y_1 + Y_2\sim Gamma(w_1+w_2,\lambda)$$
\end{prop}



Other distributions:
\begin{enumerate}
	\item Beta distribution
	\item Binomial Distribution
	\item Hypergeometric Distribution
	\item Negative Binomial distribution
\end{enumerate}


\section{Linear Algebra}
\textbf{Matrix}:\\
Consider $A,B$ two $n \times n$ squared matrix

\textit{Notation}:
\begin{enumerate}
	\item[$I_n$] is the identity matrix of order n
	\item[$1_n$] is the $n \times 1$ (column) vector with all elements equal to 1
	\item[$\bigcirc$] is a matrix with all element equal to zero
	\item[$|A|$] denotes the determinant of $A$
\end{enumerate}



\textit{Definitions / Properties}:

\begin{defi}
	$A$ is called \textit{symmetric matrix} if 
	$$A=A^T$$
\end{defi}


\begin{prop}
	for two conformable matrix $A,B$ we have:
	$$|AB|=|A||B|$$
\end{prop}


\begin{defi}
if $|A| \not = 0$, $A$ is called \textit{non singular} or \textit{invertible} and there exist a matrix $A^{-1}$ called \textit{inverse} such that 
$$AA^{-1}=A^{-1}A=I_n$$
\end{defi}

\begin{defi}
	A \textit{diagonal matrix} is a matrix with all elements outside the main diagonal equal to zero
\end{defi}

\begin{defi}
	A matrix $A$ is called \textit{invertible} if there exist an invertible matrix $P$ such that $P^{-1}AP$ is a diagonal matrix
\end{defi}
\begin{prop}
It holds:
$$(A^T)^{-1}=(A^{-1})^T$$
$$(AB)^{-1}=B^{-1}A^{-1}$$
\end{prop}


\begin{defi}
A symmetric matrix $A$ is said to be \textit{positive semi-definite} if 
$$v^TAv \geq0, \ \forall v\in \mathbb{R}^n$$
\end{defi}

\begin{defi}
A matrix $A$ is called orthogonal if:
$$A^{-1}=A^T$$
\end{defi}

\begin{defi}
Given a matrix $A$, we call \textit{trace of $A$} the sum of all the elements on the main diagonal:
$$Tr(A):= \sum_{i=1}^{n} a_{ii}$$
\end{defi}
\begin{prop}
For any matrix $A,B$ we have:
$$Tr(AB)=Tr(BA)$$
\end{prop}

\begin{defi}
An \textit{idempotent matrix} A is a matrix which, when multiplied by itself, yields itself i.e.:
$$AA=A$$
\end{defi}

\begin{prop}
	properties of an idempotent matrix A:
	\begin{enumerate}
		\item $I-M$ is also an idempotent matrix
		\item A is idempotent if and only of for all positive integers k, $A^k=A$
		\item an idempotent matrix is always diagonalizable and its eigenvalues are either 0 or 1
		\item the trace of an idempotent matrix is always an integer and it is equal to its rank
	\end{enumerate}
\end{prop}


\begin{prop}
	Two identities:
	\begin{enumerate}
		\item $(A+BCD)^{-1}=A^{-1} -A^{-1}B(C^{-1}+DA^{-1}B)^{-1}DA^{-1}$
		\item $(A+bd^T)^{-1}=A^{-1}-\frac{1}{1+d^TA^{-1}b}A^{-1}bd^TA^{-1}$
	\end{enumerate}
\end{prop}



\begin{teo}\textbf{Spectral theorem}\\
	Let $A$ be a symmetric $n \times n$ matrix, then there exist an orthogonal matrix $Q$ such that:
	$$A=Q \Lambda Q^T$$
	where $\Lambda$ is a diagonal matrix whose diagonal elements are the eigenvalues $\lambda_1....\lambda_n$ of $A$.
\end{teo}

\begin{corol}
$$|A|=|\Lambda|=\prod_{i=1}^n \lambda_i$$
\end{corol}


\section{Multivariate analysis}

\begin{defi}
Take $X_1....X_n$ random variables defined on the same probability space, we define the \textit{random vector} or the \textit{multivariate \rv} $X$ as:
$$X=(x_1....x_n)^T$$

\end{defi}
\begin{defi}
The \textit{mean vector} of $X$ is obtained by forming the vector of the mean values of the components
$$\e=(\e[X_1] \dots \e[x_n])^T$$
\end{defi}

\begin{defi}
	Similarly we can define the \textit{variance matrix} as:
	\[
	Var[X]:=\begin{bmatrix}
	Var(X_1) & Cov(X_1,X_2) & \dots  & Cov(X_1, X_{n}) \\
	Cov(X_2,X_1) & Var(X_2) &  \dots  & \vdots \\
	\vdots & \vdots  & \ddots & \vdots \\
	Cov(X_n,X_1) & \dots &  \dots  & Var(X_n)
	\end{bmatrix}	
	\]
\end{defi}

\begin{defi}
	A generic element of the \textit{correlation matrix} is defined as following:
	\[
	Corr(x_i,x_j) := \frac{Cov(X_i,X_j)}{\sqrt{Var(x_i)Var(X_j)}}	
	\]
\end{defi}

\begin{lem}
Let $A=a_{ij}$ be a $k \times n$ matrix, $b=(b_1 \dots b_n)^T$ a $n\times 1$ vector and  $x=(X_1\dots X_n)$ a random vector with $\mathbb{E}[x]=\mu$, $Var(x)=V$ define 
$$Y:=Ax+b$$
then
$$\mathbb{E}[Y]=A\mu+b$$
$$Var[Y]=AvA^T$$
\end{lem}

\begin{lem}
	The variance matrix $V$ of the random vector $X$ i positive semi-definite and it is positive definite if there exist no vectors $b$ such that $b^T$ is a degenerate \rv .
\end{lem}

\begin{lem}
If $A=(a_{ij})$ is a $n \times n$ matrix then:
$$\e [X^TAX]=\mu^T A \mu + Tr(AV)$$
\end{lem}
\begin{proof}
		\[
	\begin{split}
	\e  [X^TAX]
	&=\e[\sum_{i=1}^{n}\sum_{j=1}^{n}x_i a{ij}x_j]\\
	&=\sum_{i=1}^{n}\sum_{j=1}^{n} a_{ij}\e[x_i x_j]\\
	&=\sum_{i=1}^{n}\sum_{j=1}^{n}a_{ij}\mu_i\mu_j + v_{ij}\\
	&= \mu^T A \mu + \sum_{i=1}^{n}(AV)_{ii}\\
	&=\mu^T A \mu + Tr(AV)
	\end{split}
	\]
\end{proof}

\textbf{Multivariate Gaussian distribution:}\\

Consider a vector $Z=(Z_1...Z_k)^T$ where $Z_1....Z_k$ are \iid standard Gaussian \rv s. Now set 
$$Y=AZ+\mu$$
Where $A$ is a non singular  $k \times k$ matrix and $\mu$ is a $k \times 1$ vector.\\
It is natural to define $Y$ as a \textit{k-generated distribution of the Gaussian distribution}.\\
We start from:
$$f_z=\frac{1}{(2\pi)^{k/2}}\exp \bigg\{ -\frac{1}{2}t^Tt \bigg\}$$
($Z$ are independent so we simply multiplied them).\\
Since $Z=A^{-1}(Y-\mu)$, the Jacobian of the transformation is:
$$\bigg| \frac{dz_i}{dy_i} \bigg|=|A|^{-1}=|V|^{-1/2}$$
Taking into account that $|V|=|AA^T|=|A|^2$.

Setting $Y=At+\mu \implies t=A^{-1} (Y- \mu)$ we obtain:
$$t^Tt=\{ A^{-1} (Y- \mu) \}^T \{ A^{-1}(Y-\mu) \}=(Y-\mu)^T V^{-1}(Y-\mu)$$
Therefore the density of $Y$ is:
$$f_Y(y)=\frac{1}{(2\pi)^{k/2}|V|^{1/2}}\exp \bigg\{ -\frac{1}{2}(y-\mu)^T V^{-1} (y-\mu) \bigg\} $$
We say that the random variable $Y=(Y_1...Y_n)^T$ with density function $f_Y$ is a multivariate Gaussian \rv \ with mean $\mu$ and variance $V$. $Y\sim N_k(\mu,V)$.\\

Now we will explore \textit{marginal and conditional} distribution of Y.\\
\begin{prop}
	If $A$ is a $k\times k$  positive matrix and $b$ is a $k\times 1$ vector then:
	\[
	\int_{\mathbb{R}^k} \frac{1}{(2\pi)^k/2} \exp\bigg\{ - \frac{1}{2} (y^TAy - 2 b^T y )\bigg\} dy = \frac{\exp \{ 1/2 b^TA^{-1} b \} }{|A|^{1/2}}
\]
\end{prop}
\begin{proof}
	Let $\mu A^{-1}b$ and within the integral expand $\exp{}$ by adding and subtracting  $\frac{1}{2}\mu^TA\mu$ so that
	$$\int_{\mathbb{R}^k} \frac{1}{(2\pi)^k/2} \exp\bigg\{ - \frac{1}{2} (y^TAy - 2 b^T y )\bigg\} dy=|A^{-1}|^{1/2}\exp \bigg\{ \frac{1}{2} \mu^TA\mu \bigg\} \int_{\mathbb{R}^k}g(y)$$
\end{proof}

\section{Basic Concepts of Random Samples}

\begin{defi}
	Let $X_1...X_n$ \iid   \rv s with distribution $\sim f_{X_i}(x_i; \theta)$. We call $X:=(X_1...X_n)	$ \textit{random sample}
\end{defi}
According with the definition of \rs the distribution of $X$ will be:
$$f_X(X;\theta)= \prod_{i=1}^{n}f_{x_i}(x_i;\theta)$$
\begin{defi}
	We denote by $x=(x_1...x_n)$ the observed sample
\end{defi}

\begin{defi}
	A \textit{statistical model} is defined as following
	$$\{ f_{X}(x;\theta) : \theta \in \Theta \}$$
	Where $\Theta $ is the \textit{parametric space}
\end{defi}
Usually $\Theta$ will be a open subset of $\mathbb{R}^n$.\\


\begin{defi}
	Let $X_1 , ... , X_n$ be a random sample of size $n$ from a population and let $T(x_i, ... ,x_n)$ be a real-valued or vector-valued function whose domain includes
	the sample space of $(X_1, ... , X_n)$. Then the random variable or random vector
	$T_n = T(X_1, ... ,X_n)$ is called a \textit{statistic}. The probability distribution of a statistic $T_n$	is called the \textit{sampling distribution} of $T_n$.
\end{defi}
Some examples of statistic are:
\begin{itemize}
	\item Sample mean: $\bar X_n:=\frac{1}{n} \sum_{i=1}^n x_i$
	\item Sample variance: $\tilde S^2:=\frac{1}{n} \sum_{i=1}^n (x_i - \bar X )^2 $
	\item Corrected sample variance: $S^2:=\frac{1}{n-1} \sum_{i=1}^n (x_i - \bar X )^2 $
	\item Sample moments of order $r$: $M_{r,n}:=\frac{1}{n}\sum_{i=1}^{n}x_i^r$
	\item Sample moments of order $r$: $\bar M_{r,n}:=\frac{1}{n}\sum_{i=1}^{n}(x_i-\bar x_n)^r$
	\item Ordered statistic: $X_{(m)}$
	\item Sample min: $X_{(1)}$
	\item Sample max: $X_{(n)}$
	\item  
	$
	\text{Sample median:   }	Me:=\begin{cases}
	\frac{1}{2}(X_{(n/2)}+X_{(n/2+1)}) \ \ \ if \ n-even\\
	X_{\big(\frac{n+1}{2}\big)}\ \ \ \ \ \  if \ n-odd
	\end{cases}
	$
\end{itemize}
Finding the distribution of $T_n$ in general it is complex. We can make it easier by putting constrains.\\
Suppose for example $X\sim N(\mu, \sigma^2)\leftarrow$ fair assumption because there is the Central Limit Theorem.
\begin{teo}\textbf{Fisher Cochran}\\
	\label{teo:fishC}
	Let $Q,Q_1,Q_2$ \rv s such that $Q=Q_1+Q_2$ and let $Q\sim \mathcal{X}_{g}^2$ and $Q_2\sim\mathcal{X}_{g_1}^2$. Then
	$$Q_2\sim \mathcal{X}_{g_2}^2 \ \ \ \text{       where $g_2=g-g_1$,}$$
	and $Q_1 \coprod Q_2$
\end{teo}


\begin{prop}
	Let $ X_i\sim N(\mu, \sigma^2)$ and $X=(X_1...X_n)$. Then
	\begin{enumerate}
		\item $\bar X_n:=\frac{1}{n} \sum_{i=1}^n x_i \sim N(\mu, \frac{\sigma^2}{n})$
		\item $\tilde S_n:=\frac{1}{n} \sum_{i=1}^n (x_i - \bar X )^2 \sim \frac{\sigma^2}{n} \mathcal{X}_{n-1}^2$
	\end{enumerate}
	where $\mathcal{X}_{n-1}^2$ is the Chi-squared distribution with $n-1$ degrees of freedom
\end{prop}
\begin{proof}
	\begin{enumerate}
		\item the first one is easily checked using the linearity of the Gaussian distribution.
		\item Consider the \rv \  $\tilde S^2=\frac{1}{n}\sum_{i=1}^{n}(x_i- \mu)^2$ and proceed as following:
		\[
		\begin{split}
		\frac{n \tilde S^2}{\sigma^2}
		&=\sum_{i=1}^{n} \bigg( \frac{x_i- \mu}{\sigma} \bigg)^2\\
		&=\sum_{i=1}^{n} \bigg( \frac{x_i- \mu +\bar x_n -\bar x_n}{\sigma} \bigg)^2\\
		&=\sum_{i=1}^{n} \bigg( \frac{x_i- \bar x_n}{\sigma} \bigg)^2 + \sum_{i=1}^{n} \bigg( \frac{ \bar x_n - \mu}{\sigma} \bigg)^2 + 2 \sum_{i=1}^{n} \bigg( \frac{x_i- \bar x_n}{\sigma} \bigg) \bigg( \frac{ \bar x_n - \mu}{\sigma} \bigg)
		\end{split}
		\]
		Now consider separately the tree terms of the sum:\\
		$\sum_{i=1}^{n} \bigg( \frac{x_i- \bar x_n}{\sigma} \bigg) =n \bar x_n \sum_{i=1}^{n}x_i=n \bar x_n -n \bar x_n =0$\\
		$\implies 2 \sum_{i=1}^{n} \bigg( \frac{x_i- \bar x_n}{\sigma} \bigg) \bigg( \frac{ \bar x_n - \mu}{\sigma} \bigg)=0$ \\
		For the other two terms consider:\\
		$\sum_{i=1}^{n} \bigg( \frac{x_i- \mu}{\sigma} \bigg)^2 \sim \mathcal{X}_1^2$\\
		And\\
		$\sum_{i=1}^{n} \bigg( \frac{ \bar x_n - \mu}{\sigma} \bigg)^2= \bigg( \frac{ \bar x_n - \mu}{\sigma/ \sqrt{n}} \bigg)^2\sim \mathcal{X}_n^2$\\
		So, using the theorem \ref{teo:fishC}\\
		$$\sum_{i=1}^{n} \bigg( \frac{x_i- \mu}{\sigma} \bigg)^2= \sum_{i=1}^{n} \bigg( \frac{x_i- \bar x_n}{\sigma} \bigg)^2 +  \bigg( \frac{ \bar x_n - \mu}{\sigma/ \sqrt{n}} \bigg)^2 \sim \mathcal{X}_{n-1}^2$$
		$\implies \tilde S_n^2\sim \frac{\sigma^2}{n} \mathcal{X}_{n-1}^2 $
	\end{enumerate}
\end{proof}

Let $X=(X_1...X_n)$ be a random sample where $X_i\sim N(\mu,\sigma^2)$, consider the statistic:
$$T_n=\frac{\bar X_n-\mu}{ S_n/\sqrt{n}}=\frac{(\bar X_n-\mu)/\sigma/\sqrt{n}}{\sqrt{S_n/\sigma^2}}=\frac{Z}{\sqrt{R/(n-1)}}$$
where $Z\sim N(0,1), R\sim(\mathcal{X}_n-1)$
We can say that $T_n\sim T-Student$ with $(n-1)$ degrees of freedom only if $Z \coprod R$. We can they are independent because of the following:
\begin{teo}
	If $(x_1...x_n)$ random sample with $X_i\sim N(\mu \sigma^2)$ then 
	$$\bar X_n =\frac{1}{n} \sum_{i=1}^{n} x_i$$	
	$$S^2=\frac{1}{n-1} \sum_{i=1}^{n} (x_i -\bar X_n)^2$$
	are independent
\end{teo}
The other way around is also true:
\begin{teo}
	If $\bar X_n =\frac{1}{n} \sum_{i=1}^{n} x_i$, $S^2=\frac{1}{n-1}\sum_{i=1}^{n} (x_i -\bar X_n)^2$ are independent then $X=(x_1...x_n)$ is random sample where $X_i\sim N(\mu \sigma^2)$ 
\end{teo}


      \newpage
      \chapter{Concentration Measure}
\label{cha:Prop R S}
\vspace{15pt}


We're now going to investigate some methods to study the tail of a distribution.\\
Consider a non negative \rv \  and let $t>0$. Then by Markow inequality we have:
$$\p (X\geq t)\leq \frac{\e [X]}{t}$$
We can try to improve this inequality using a function $\Phi$ that is strictly increasing with non negative values. Then we can write
$$\p (X\geq t)=\p (\Phi(X)\geq \Phi(t) )\leq \frac{ \e [\Phi(X)]}{\Phi(t)}$$
In particular we can take $\Phi(x)=x^q$ , $X\geq 0, q>0$ so we have
$$\p (|X- \e [X]|\geq t)\leq \frac{ \e [|X-\e [X]|^q]}{t^q}$$
In specific examples one can choose the value of $q$ that optimize the upper bound.\\
A related idea is a the basis of \textbf{Chernoff's bounding method}: taking $\Phi(X)=e^{sx}$ where $s$ is an arbitrary positive number for any random variable $X$ and $t\in \mathbb{R}$ we have:
\begin{equation}\label{ineq:Chernoffb}
	\p (X \geq t)= \p (e^{sX}\geq e^{st})\leq \frac{\e[e^{sX}]}{e^{st}}
\end{equation}

So we can bound the probability using the characteristic function which is usually easier than $\e [X^q]$ to compute.\\
However it can be proven that the bounding given form $\Phi (X)=x^q$ is always better than the one given by $\Phi(X)=E^{sX}$.\\
\begin{teo}\textbf{Cauchy Swartz inequality}\\
	Given two \rv s with finite second moments then:
	$$|\e[XY]|^2\leq \e[X^2]\e[Y^2]$$
\end{teo}

\begin{teo}
	let $t\geq 0$ then
	$$\p(X - \e[X]\geq t)\leq \frac{Var(X)}{Var(X)+t^2}$$
\end{teo}
\begin{proof}
	We assume that $E[X]=0$ (the proof for the general case is the same).\\
	For all $t$ we can write 
	$$t=\e[t]=\e[t]-e[X]=\e[t-X]\leq \e[(t-X)\mathbbm{1}_{[X<t]}(X)]$$
	Then for $t\geq 0$ from Cauchy Swartz inequality:
	\[
	\begin{split}
	t^2 
	& \leq \e[(t-X)^2] \e[(\mathbbm{1}_{[X<t]}(X))^2]\\
	&= \e[(t-X)^2] \p(X<t)\\
	&= (Var(X)+t^2)\p(x<t)
	\end{split}	
	\]
	$\implies \p(X<t)\geq \frac{t^2}{Var(X)+t^2}$\\
	$\implies \p(X\geq t)= 1-\p(X<t)\leq 1-\frac{t^2}{Var(X)+t^2}= \frac{Var(X)}{Var(X)+t^2} $
\end{proof}

\begin{teo}
	Let $f,g$ be non decreasing real valued functions defined on the real line. If $X$ is a real valued \rv \  then:$$\e[f(x)g(x)]\geq \e[f(x)]\e[g(x)]$$
	If $f$ is non increasing and $g$ is non decreasing then:
	$$\e[f(x)g(x)]\leq \e[f(x)]\e[g(x)]$$ 
\end{teo}
\begin{proof}
	Let $Y$ be a \rv \  with the same distribution of $X$ and $X\coprod Y$. Because $f,g$ are non decreasing functions we have $(f(x)-f(y))(g(x)-g(y))\geq 0$
	\[
		\implies  0 \leq \e[(f(x)-f(y))((g(x)-g(y))]=\e[f(x)g(x)-f(x)g(y)-f(y)g(x) + f(y)g(y)]
	\]
	$\implies$
	\[
	\begin{split}
\e[f(x)g(x)] &\geq \e[f(x)g(y)]+ \e[f(y)g(x)]- \e[f(x)g(y)] \\&=\e[f(x)g(y)]\\&=\e[f(x)]\e[g(y)]\\&=\e[f(x)]\e[g(x)]
	\end{split}
	\]
	The second part of the theorem can be proved in the same way.
\end{proof}
The previous theorem can be generalized as following:
\begin{teo}
	Let $f,g: \mathbb{R}^n\to \mathbb{R}$ be non increasing functions. Let $X_1...X_n$ be independent real valued \rv s and define the \rv \ $X=(X_1...X_n)$ that take values in $\mathbb{R}^n$ then:
	$$\e[f(x)g(x)]\geq \e[f(x)]\e[g(x)]$$
	If $f$ is non increasing and $g$ is non decreasing then:
	$$\e[f(x)g(x)]\leq \e[f(x)]\e[g(x)]$$
\end{teo}

\section{Concentration for sum of \rv s}

We want to bound the probability $\p(S_n-\e[S_n]\geq t)$ where $S_n=\sum_{i=1}^{n} X_i$ and $X_1...Xn$ are independent \rv s real valued.\\
An application of the Chebyshev's inequality give us:
$$\p(|S_n-\e[S_n]|\geq t)\leq \frac{Var(S_n)}{t^2}=\frac{\sum_{i=1}^{n}Var(X_i)}{t^2}$$
Applying the Chebyshev's inequality to $ \frac{1}{n} \sum_{i=1}^{n}x_i$ we get
\[
\begin{split}
\p\bigg( \bigg| \frac{1}{n} \bigg( \sum_{i=1}^{n}x_i - \e [X_i] \bigg) \bigg| \geq \epsilon \bigg)
&=\p\bigg( \bigg| S_n - \e [S_n] \bigg| \geq \epsilon n \bigg)\\
&\leq \frac{\sum_{i=1}^{n}Var(X_i)}{\epsilon^2 n^2}
\end{split}
\]
If we define $\sigma^2 :=\frac{1}{n}\sum_{i=1}^{n}x_i$ then:
\begin{equation} \label{eq:FromChebishev}
	\p\bigg( \bigg| \frac{1}{n} \sum_{i=1}^{n}x_i - \e [X_i] \bigg| \geq \epsilon \bigg)\geq \frac{\sigma^2}{n\epsilon^2}
\end{equation}
To understand why the equation \ref{eq:FromChebishev} is unsatisfying recall what appens with the \textit{Central Limit Theorem}:
$$\p\bigg(\sqrt{\frac{n}{\sigma^2}}\bigg( \frac{1}{n}\sum_{i=1}^{n} X_i -\e[X_i] \bigg) \geq y \bigg) \varinjlim^{n\to \infty} 1- \Phi(y)\leq \frac{1}{\sqrt{2\pi}}\frac{e^{-y^2/2}}{y}$$
(where $\Phi$ is the CDF of the standard Gaussian distribution)\\
so
$$\p\bigg(\sqrt{\frac{n}{\sigma^2}}\bigg( \frac{1}{n}\sum_{i=1}^{n} X_i -\e[X_i] \bigg) \geq \epsilon  \bigg)\lesssim \exp\bigg\{ \frac{-n\epsilon^2}{2\sigma} \bigg\}$$
So for $\p\bigg(\sqrt{\frac{n}{\sigma^2}}\bigg( \frac{1}{n}\sum_{i=1}^{n} X_i -\e[X_i] \bigg) \geq \epsilon \bigg)$ we have:
$$\exp\bigg\{ \frac{-n\epsilon^2}{2\sigma} \bigg\}\leftarrow \text{from Central Limit Theorem}$$
$$\frac{\sigma^2}{n\epsilon^2}\leftarrow \text{from Chebyshev's inequaity}$$
From here we can see that the Chebyshev's inequality doesen't work well for the sum of $n$ \rv s when $n$ is large. Meanwhile the Chebyshev's inequality works better than the Central Limit Theorem for small $n$.\\

Another instrument previously introduced that can be helpful for bounding tail probabilities of sum of independent \rv s is the \textbf{Chernoff bounding} \ref{ineq:Chernoffb}:
\begin{equation}
\label{eq:ChernoffSum}
\p(S_n-\e[S_n]\geq t)\leq e^{-st}\e[\exp \{ s \sum_{i=1}^{n} (x_i-\e[X_i]) \}]=e^{st} \prod_{i=1}^{n}\e[\exp \{ s(x_i-\e[X_i]) \}]	
\end{equation}
(remember that $s$ is an arbitrary positive number)

Now the problem of finding bond on the tail probability reduces to the problem of finding (upper) bounds for the moments generating function of $X_i-\e[X_i]$.


As we saw Chebyshev's inequality \ref{eq:Chebyshev} does not work well for sums of \rv s.\\
In this section we will see a partial solution given by \textit{Hoeffding's Inequality}, then a more complete solution given by \textit{Bernstein Inequality}.

\begin{lem}\label{lem:Hoeffding}
	Let $X$ be a \r with $\e [X]=0$ (actually it can be generalized for a \rv \  with any expected value), $a\leq X \leq b$ ($X$ bounded \rv). Then
	\[
	\e[e^{sx}] \leq \exp \bigg\{\frac{s^2(b-a)^2}{8} \bigg\}\ \ \ \ \ for  \ \ s>0
	\]
\end{lem}

\begin{proof}
	By the convexity of the exp function we have 
	\[
	e^{sx} \leq \frac{x-a}{b-a}e^{sb}+\frac{b-x}{b-a}e^{sa} \ \ \ \  \text{     with $a\leq x\leq b$}
	\]
	Using $\e[X]=0$ and defining $p:=\frac{-a}{b-a}$ we obtain
	\[
	\begin{split}
		\e[e^{sx}]
		&\leq \e[\frac{x-a}{b-a}e^{sb}+\frac{b-x}{b-a}e^{sa} ]\\
		& \leq \frac{b}{b-a}e^{sa}-\frac{a}{b-a}e^{sb}\\
		& =\frac{b-a+a}{b-a}e^{sa}+pe^{sb}\\
		&= (1-p)e^{sa}+pe^{sb}\\
		&= (1-p)e^{sa}+pe^{s(b-a+a)}\\
		&= (1-p)e^{sa}+pe^{s(b-a)}e^{sa}\\
		&= (1-p+pe^{s(b-a)})e^{sa}\\
		&= (1-p)e^{sa}+pe^{s(b-a)}e^{sa\frac{b-a}{b-a}}\\
		&= (1-p)e^{sa}+pe^{s(b-a)}e^{-ps(b-a)}\\
	\end{split}
	\]
	Then defining 
	$$\mu=s(s-a)$$
	$$\Phi(\mu)=-p\mu + \ln(1-p+pe^{\mu})$$
	so we have that the last equality $(1-p)e^{sa}+pe^{s(b-a)}e^{-ps(b-a)}=e^{\phi(\mu)}$\\
	It is possible to show 
	$$\Phi'(X)=-p +\frac{p}{p+(1-p)e^{-\mu}}$$
	therefore $\Phi(\mu)=\Phi'(0)=0$, moreover
	$$\Phi(\mu)=\frac{p(1-p)e^{-\mu}}{(p+(1+p)p^{-\mu})^4}\leq \frac{1}{4}$$
	by Taylor's theorem we have:
	$$\Phi(x)=\Phi(0)+\mu\Phi'(0)+\frac{\mu}{2}\Phi''(\sigma)\leq \frac{\mu^2}{8}=\frac{s^2(b-a)^2}{8}$$
	with $\sigma \in [0,\mu]$.
\end{proof}
We're now ready for the \textbf{Hoeffding's Inequality}
\begin{teo}
	Let \xii be independent \rv \ such that $x_i\in[a_i,b_i]$ then for any $t>0$
	$$\p(S_n-\e[S_n]\geq t)\leq e^{-\frac{2t^2}{\sum_{i=1}^{n}(b_i-a_i)^2}}$$
	$$\p(S_n-\e[S_n]\leq -t)\leq e^{-\frac{2t^2}{\sum_{i=1}^{n}(b_i-a_i)^2}}$$
\end{teo}


\begin{proof}
	Using the \textit{Chernoff's bounding} for sums of \rv s  \ref{eq:ChernoffSum} and the precedent lemma \ref{lem:Hoeffding} we obtain
	$$\p(S_n-\e[S_n]\geq t)\leq e^{-st} \prod_{i=1}^{n}e^{\frac{s^2(b-a)^2}{8}}=e^{-st}e^{\frac{s^2}{8}\sum_{i=1}^{n}(b_i-a_i)^2}=e^{-\frac{2t^2}{\sum_{i=1}^{n}(b_i-a_i)^2}}$$
	where we chose $s=\frac{4t}{\sum_{i=1}^{n}(b_i-a_i)^2}$
\end{proof}

This inequality has the same form as the one based on the central limit theorem except that the average variance $\sigma^2$ is replaced by the upper bound $\frac{1}{4}\sum_{i=1}^{n}(b_i-a_i)^2$. Next we will see \textit{Bernstein Inequality} an inequality that take into account also the variance.\\

\begin{lem} \label{lem:Bernstain}
	Assume that $\e[X_i]=0$ then if for all $X_i$, $|X_i|\leq c$ ($X_i$ are bounded):
	$$\e [e^{sx_i}] \leq \exp\bigg\{ s^2\sigma^2_i \frac{e^{sc}-1-sc}{sc} \bigg\}$$
	where $\sigma_i^2:=\e[X_i^2]$
\end{lem}
\begin{proof}
	define $F_i=\sum_{r=2}^{\infty} s^{r-2} \frac{\e [x_i^r]}{r! \sigma_i^2 }$.\\
	Since (for Taylor) $e^{sx}=1+sx+\sum_{r=2}^{\infty} s^r\frac{x^r}{r!}$ then taking into account  $\e[X_i]=0$
	$$\e[s^{sX_i}]=1+s \e [X_i] \sum_{r=2}^{\infty}s^r\frac{\e [x^r_i]}{r!}=1+s^2\sigma^2 F_i \leq e^{s^2\sigma_i^2F_i}$$
	Because we supposed $|X_i|\leq c$ for each index $r$ we have
	$$\e[X_i^r]= \e [X_i^{r-2}X_i^2] \leq \e [c^{r-2}X_i^2] = c^{r-2}\sigma_i^2$$
	Thus
	\[
	\begin{split}
	F_i
	& \leq \sum_{r=2}^{\infty}\frac{s^{r-2}c^{r-2} \not \sigma_i^2}{r! \not\sigma_i^2}\\
	& = \frac{1}{(sc)^2}\sum_{r=2}^{\infty}\frac{(sc)^{r}}{r!}\\
	&= \frac{e^{sc}-1-sc}{(sc)^2}
	\end{split}
	\]
	where in the last step we recognized the summation as the exponential wrote in Taylor series missing the first two terms
\end{proof}
\begin{teo}\textbf{Bernstein Inequality}\\
	Let \xii be independent real valued \rv s with $\e[X_i]=0$ and $|X_i|\leq c$. Set $\sigma^2=\frac{1}{n}\sum_{i=1}^{Var[X_i]}$ (note that $Var[X_i]=\e[X_i^2]$ because$\e[X_i]=0$). Then for $t>0$
	$$\p(\sum_{i=1}^{n}X_i\geq t)\leq \exp\bigg\{-\frac{n\sigma^2}{c^2} h\bigg(\frac{ct}{n\sigma^2} \bigg)\bigg\}$$
	where $h(\mu)=(1+\mu)\ln(1+\mu)-\mu$ for $\mu \geq 0$.
\end{teo}

\begin{proof}
		Using the \textit{Chernoff's bounding} for sums of \rv s  \ref{eq:ChernoffSum} we obtain and the precedent lemma \ref{lem:Bernstain} we obtain 
			$$\p(\sum_{i=1}^{n}X_i \geq t)\leq \exp \bigg\{\frac{n\sigma^2(e^{sc}-1-sc)}{c^2}-st\bigg\}$$
			and the bound is minimized by $s=\frac{1}{c}\ln\bigg(1+\frac{tc}{n\sigma^2} \bigg)$
\end{proof}

\begin{corol}
	Referring to the \textit{Bernstein Inequality} there is a lower bound for $h$:
	$$h(\mu)\geq \frac{\sigma^2}{2+2\frac{\mu}{\epsilon}}$$
	so for $\epsilon >0$ the \textit{Bernstein Inequality} becomes:
	$$\p(\sum_{i=1}^{n}X_i\geq t)\leq \exp\bigg\{-\frac{n\epsilon}{2\sigma^2+\frac{2}{3}c\epsilon}\bigg\}$$
\end{corol}
This result is extremely useful in hypothesis testing ($\p(T_n>t)=\alpha$) because usually to do the test we have to invert the CDF of $T_n$. With this result we can instead use the second term of the \textit{Bernstein Inequality} as $\alpha$ and then we can isolate the $\epsilon$ to find the small $t$. Sadly this work only if $T_n$ is a sum of independent \rv s which however is the most common situation.\\

We consider now the problem of deriving inequalities for the Variance of functions of independent \rv s.

\begin{lem}
	Let $\mathcal{X}$ be some set and let $g:\mathcal{X}^n\to \mathbb{R}$ be a measurable function. Define $Z:=g(X_1...X_n)$ where \xii are independent \rv s in $\mathcal{X}$ and $\e_iZ$ the expected value of $Z$ \wrt $X_i$ that is $\e_iZ=\e[Z|X_1...X_{i-1},X_{i+1}...X_n]$. Then
	$$Var(Z)\leq \sum_{i=1}^{n}\e[(Z-\e_iZ)^2]$$
\end{lem}

Directly from this lemma follows
\begin{teo}\textbf{Efron-Stein Inequality}
	Let $X_1'...X_n'$ be from an independent copy of $X_1...X_n$ and define $Z_i'=g(X_1...X_{i-1},X_{i}',X_{i+1}...X_n)$ then
	$$Var(Z)\leq \sum_{i=1}^{n}\e[(Z-Z_i')^2]$$
	when $g(X_1...X_n)=\sum_{i=1}^{n}X_i$ the inequality becomes an equality.
\end{teo}

      \newpage
      \chapter{Likelihood Function}
\label{cha:likf}
\vspace{15pt}



The \textit{likelihood function} is a function that contains all the statistical information required to make inference.\\
\begin{defi}
	Consider a random sample  $(X_1...X_n)$ from $ X \sim f_X(X; \theta)$, then the distribution of $(X_1..X_n)$ will be:
	$$f_{\underline X}(\underline x;\theta)=\prod_{i=1}^{n}f_{X_i}(x_i;\theta)$$
	when we see $f_{\underline X}(\underline x;\theta)$ as a function of $\theta$ for fixed $\underline x$ we call it \textbf{likelihood function}
	$$\mathcal{L}(\theta,\underline x)=\prod_{i=1}^{n}f_{X_i}(x_i;\theta)$$
\end{defi}
	An important function related to the likelihood function is the \textbf{log likelihood function} 
\begin{defi}
	The $\ln$ of the likelihood function is said \textit{log likelihood function}
	$$V_n(\theta)=\ln\mathcal{L}(\theta,\underline x)=\ln \bigg( \prod_{i=1}^{n}f_{X_i}(x_i;\theta) \bigg)=\sum_{i=1}^{n}\ln( f_{X_i}(x_i;\theta))$$
\end{defi}
\begin{defi}\label{defi:scoref}
	\textbf{Score function}:
	$$V'_n=\frac{d}{d \theta}V_n(\theta)=\frac{\mathcal{L}'(\theta, \ux)}{\mathcal{L}(\theta, \ux)}$$
\end{defi}
Note that if we fix $\theta$ then $\mathcal{L}(\theta,\underline x)$ is (related to) the probability that the particular value we fixed for $\theta$ has generated $\underline x$.\\
Suppose we fix two value $\theta,\theta_2\in \Theta$ and 
$\mathcal{L}(\theta_1,\underline x)>\mathcal{L}(\theta_2,\underline x)$ then
we say $\ux$ is \textit{"more likely"} generated under $\theta_1$.\\
Note that the same meaning is also applicable to the log likelihood function.\\
It is because of this that we usually search for the maximum of the likelihood function. Often to find it we just derive, but sometimes $\mathcal{L}$ is not regular enough so we have to \textit{"regularize"} it.
\section{Likekihood principles}
The statistical inference based on the likelihood function is a consequence of two principles:
\begin{enumerate}
	\item \textbf{Week likelihood principle:} for a fixed parametric model $X\sim F_X(x,\theta)$ if two observed samples $\ux$ and $ \underline{y}$ are such that $$\mathcal{L}(\theta,\underline x) \propto \mathcal{L}(\theta,\underline y)$$ 
	then the two likelihood functions are equivalent i.e. must produce the same inference result on $\theta$.
	\item \textbf{Strong likelihood principle:} let $\ux$ be an observed sample under the model $X\sim F_X(x,\theta)$ with likelihood function $\mathcal{L}(\theta,\underline x)$ and let $\underline y$ be an observed sample under the model $X\sim F_Y(y,\theta)$ with likelihood function $\mathcal{L}(\theta,\underline y)$.\\
	If $\mathcal{L}(\theta,\underline x) \propto \mathcal{L}(\theta,\underline y)$ the the two samples provides with the same inference.\\
	
	
	The fundamental difference between \textit{Probability} and \textit{Statistic} is that in the first one the goal is to find the chance of a \rv \ to take a particular value, statistic instead given the results of a experiment, try to find the distribution where it came from.\\
	\begin{eg}
		Take $(X_1,X_2,X_3)$ from one of the following distribution:
		\begin{enumerate}
			\item $X\sim Ber(\theta_1), \theta_1=\frac{1}{2}$
			\item $X\sim Ber(\theta_2), \theta_2=\frac{1}{3}$
			\item $X\sim Ber(\theta_3), \theta_3=\frac{1}{4}$
		\end{enumerate}
	$X\in [0,1]$ $\Theta = [0,1]$.\\
	We can imagine $(x_1,x_2,x_3)$ as the results of a experiment where we had to flip a coin 3 times. Now we want to know the parameter $\theta$ of the coin we flipped tree times and we have tree possibilities:$\theta_1=\frac{1}{2}, \theta_2=\frac{1}{3}, \theta_3=\frac{1}{4}$ .
	\\So $(x_1,x_2,x_3)\in \{0 ,1 \}^3$
	
	\begin{center}
		\begin{tabular}{ | c | c | c | c | }
			\hline
			$x_1,x_2,x_3$ & $\theta_1 \frac{1}{2}$ & $\theta_2=\frac{1}{3}$ & $\theta_3=\frac{1}{4}$ \\ \hline
			0,0,0 & $\frac{1}{8}$ & $\frac{8}{27}$ & $\frac{27}{64} \cdot$ \\ \hline
			0,0,1 & $\frac{1}{8}$ & $\frac{4}{27} \cdot$ & $\frac{9}{64}$ \\ \hline
			0,1,0 & $\frac{1}{8}$ & $\frac{4}{27} \cdot$ & $\frac{9}{64}$ \\ \hline
			1,0,0 & $\frac{1}{8}$ & $\frac{4}{27} \cdot$ & $\frac{9}{64}$ \\ \hline
			0,1,1 & $\frac{1}{8} \cdot$ & $\frac{2}{27}$ & $\frac{3}{64}$ \\ \hline
			1,0,1 & $\frac{1}{8} \cdot$ & $\frac{2}{27}$ & $\frac{3}{64}$ \\ \hline
			1,1,0 & $\frac{1}{8} \cdot$ & $\frac{2}{27}$ & $\frac{3}{64}$ \\ \hline
			1,1,1 & $\frac{1}{8} \cdot$ & $\frac{1}{27}$ & $\frac{1}{64}$ \\
			\hline
		\end{tabular}
	\end{center}
	Once we know the result of the throw we will "guess" the value of $\theta$ choosing the one that give us more probability for the given result.
	\end{eg}
\end{enumerate}
\section{Condition of Regularity}
In our investigations on $\theta$ we will assume some condition of regularity for our model.\\
Given $X\sim F_x(x,\theta)$
\begin{enumerate}
	\item we assume that $\theta \in \Theta$ where $\Theta$  is a open real set
	\item for any $\theta \in \Theta$ there exist the  derivative of $\mathcal{L}(\theta;z)$ \wrt $\theta$ at least up to the third order
	\item for any $\theta_0 \in \Theta$ there exist tree functions $g,h,H$ that are integrable in a neighborhood of $\theta_0$  and 
	\begin{itemize}
		\item $\bigg| \frac{d}{d\theta} f_X(x,\theta) \bigg|\leq g(x)$
		\item $\bigg| \frac{d2}{d^2\theta} f_X(x,\theta) \bigg|\leq h(x)$
		\item $\bigg| \frac{d^3}{d^3\theta} \ln(f_X(x,\theta)) \bigg|\leq H(x)$
	\end{itemize}
	\item for any $\theta \in \Theta$
	$$0<\e[(\ln(\mathcal{L}(\theta, \uX)))^2]<\infty$$
	(With $\uX$ we're tanking it as \rv).
\end{enumerate}
In addition there is the condition of identifiability.
\begin{itemize}
	\item[5.] We say that a statistical model is identifiable if for every  $\theta_1,\theta_2$ there is al least one event $E$ such that:
	$$\p(X\in E | \theta_1)\not =\p(X\in E | \theta_2)$$
\end{itemize}
(We will always take 5. as granted)
\section{Properties of the Likelihood Function}
\begin{prop}
	\begin{enumerate}Some properties of the score function \ref{defi:scoref} are:
		\item $\e[V_n'(\theta)]=0$
		\item $Var(V_n'(\theta)=\e[(V_n'(\theta))^2])=- \e[V_n''(\theta)]$
	\end{enumerate}
\end{prop}
\begin{proof}
	\begin{enumerate}
		\item 
		\[
			\begin{split}
			\e[V_n'(\theta)]
			&=\int_{\mathbb{R}^n} V_n'(\theta)f_{\uX}(\ux, \theta) dx\\
			&=\int_{\mathbb{R}^n}\frac{f_{\uX}'(\ux, \theta)}{f_{\uX}(\ux, \theta)}f_{\uX}(\ux, \theta)dx\\
			&=\int_{\mathbb{R}^n} \frac{d}{d\theta}f_{\uX}(\ux, \theta))dx\\
			&= \frac{d}{d\theta}\int_{\mathbb{R}^n}f_{\uX}(\ux, \theta))dx\\
			&= \frac{d}{d\theta} 1\\
			&=0
			\end{split}
		\]
		Where in the in the fourth equal we used Leibniz and for the fifth recall that $f_{\uX}(\ux, \theta)$ is the PDF of $\uX$
		\item Start by showing that $V_n''(\theta)=V_n''(\theta)=\frac{f_{\uX}''(\ux;\theta)}{f_{\uX}(\ux;\theta)}-\bigg( \frac{f_{\uX}'(\ux;\theta)}{f_{\uX}(\ux;\theta)} \bigg)^2$
		\[
		\begin{split}
		V_n''(\theta)
		&=\frac{d^2}{d \theta^2} \mathcal{L}(\theta,ux)\\
		&=\frac{d}{d\theta}\frac{f_{\uX}'(\ux;\theta)}{f_{\uX}(\ux;\theta)}\\
		&=\frac{f_{\uX}''(\ux;\theta)f_{\uX}(\ux;\theta)-f_{\uX}'(\ux;\theta)f_{\uX}'(\ux;\theta)}{|f_{\uX}(\ux;\theta)|^2}\\
		&=\frac{f_{\uX}''(\ux;\theta)}{f_{\uX}(\ux;\theta)}-\bigg( \frac{f_{\uX}'(\ux;\theta)}{f_{\uX}(\ux;\theta)} \bigg)^2\\
		&=\frac{f_{\uX}''(\ux;\theta)}{f_{\uX}(\ux;\theta)}-(V_n'(\theta))^2
		\end{split}		
		\]
		So now
		\[
		\begin{split}
			\e[V_n''(\theta)]
			&=\int_{\mathbb{R}^n}\frac{f_{\ux}''(\ux;\theta)}{f_{\ux}(\ux;\theta)}f_{\ux}(\ux;\theta)dx-\e[V_n'(\theta)^2]\\
			&=\int_{\mathbb{R}^n}\frac{d^2}{d\sigma^2}f_{\uX}(\ux;\theta)dx-\e[V_n'(\theta)^2]\\
			&=\frac{d^2}{d\sigma^2}\int_{\mathbb{R}^n}f_{\uX}(\ux;\theta)dx-\e[V_n'(\theta)^2]\\
			&=-\e[V_n'(\theta)^2]
		\end{split}
		\]
	\end{enumerate}
\end{proof}

\begin{defi}
	we define the \textbf{Fisher Information} as:
	$$\ifn=-\e[V_n''(\theta)]$$
\end{defi}
Note that this is the definition of Fisher information just for a particular case, there exist a more general one.\\
The Fisher information has a central role in statistic because it can be shown that for \textit{unbiased estimators} $\tilde \theta$ it holds: $Var(\tilde \theta)\geq \frac{1}{\ifn}$. So i we can find an estimator with  $Var(\tilde \theta)= \frac{1}{\ifn}$ we are sure that it is the one with the lowest variance.\\
P\begin{prop}
	Consider \rsf \  regular then
	$$\ifn=n\mathcal{I}_1(\theta)$$
\end{prop}
\begin{eg}
	Consider $(x_1...x_n)$ from $X\sim Ber(\theta)$
	$$\lf = \prod_{i=1}^{n} f_{\underline{X_i}}(\underline{x_i};\theta)=\theta^{\sum_{i=1}^{n}x_i}(1-\theta)^{n-\sum_{i=1}^{n}x_i}$$
	$$V_n(theta)=\ln(\lf)=\ln(\theta)\sum_{i=1}^{n}x_i+(\ln(1-\theta))\bigg(n-\sum_{i=1}^{n}x_i\bigg)$$
	$$V_n'(\theta)=\frac{\sum_{i=1}^{n}x_i}{\theta}-\frac{n- \sum_{i=1}^{n}x_i}{1-\theta}$$
	$$V_n''(\theta)=\frac{-\sum_{i=1}^{n}x_i}{\theta^2}-\frac{-\sum_{i=1}^{n}x_i}{(1-\theta)^2}$$
	\[
	\begin{split}
	\e[V_n(\theta)]
	&	=\frac{1}{\theta}\sum_{i=1}^{n}\e[x_i]-\frac{1}{1-\theta}\bigg( n- \sum_{i=1}^{n}\e[x_i]\bigg)\\
	&=\frac{n\not \theta}{\not \theta}-\frac{1}{1-\theta}(n-n\theta)\\
	&=n-n=0
	\end{split}
	\]
	\[
	\begin{split}
		[V_n''(\theta)]
		&=-\frac{1}{\theta}\sum_{i=1}^{n}\e[X_i]-\frac{1}{1-\theta}\bigg( n- \sum_{i=1}^{n}\e[X_i] \bigg)\\
		&=-\frac{n\theta}{\theta^2}-\frac{n-n\theta}{(1-\theta)^2}\\
		&=-\frac{n}{\theta}-\frac{n(1-\theta)}{(1-\theta)^2}\\
		&=-\frac{n\theta}{(1-\theta)\theta}
	\end{split}
	\]
	\[
	\begin{split}
	\ifn
	&=-\e[V_n''(\theta)]\\
	&=\frac{n}{(1-\theta)\theta}
	\end{split}
	\]
\end{eg}

\section{Exponential Families}
\begin{defi}
	We say that the distribution of a \rv is an element of an \textbf{Exponential Family}	$X\sim EF(\theta)$ if its PDF can be written as follow:
	$$f_X(x;\theta)=\exp \{ Q(\theta)A(x)+ C(x) -k(\theta) \}$$
\end{defi}
The generalization to \rs becomes:
\begin{defi}
	We say that the distribution of a \rs $(x_1...x_n)$ from $X\sim EF(\theta)$ is an element of an \textbf{Exponential Family}	$X\sim EF(\theta)$ if its PDF can be written as follow:
	$$\lfd=\exp \{ Q(\theta)\sum_{i=1}^{n}A(x_i)+ \sum_{i=1}^{n} C(x_i) -nK(\theta) \}$$
\end{defi}
\begin{eg}
	$X\sim Ber(\theta),X\in\{0,1\}, \Theta=(0,1)$
	\[
	\begin{split}
	P_X(x)
	&=\theta^x (1\theta)^{1-x}\mathbbm{1}_{\{0,1\}}(x)\\
	&=\exp\{x\ln(\theta)+(1-x)\ln(1-\theta)  \}\\
	&=\exp\{x\ln(\theta)+\ln(1-\theta)-x\ln(1-\theta)  \}\\
	&=\exp\bigg\{xln{\bigg(\frac{\theta}{1-\theta}\bigg)}+\ln(1-\theta)\bigg\}
	\end{split}
	\]
	so we get 
	\begin{itemize}
		\item[$Q(\theta)$]$=\ln\bigg( \frac{\theta}{1-\theta} \bigg)$
		\item[$A(x)$]$=x$
		\item[$C(x)$]$=0$
		\item[$K(\theta)$]$=-\ln(1-\theta)$
	\end{itemize}
	Note that for $K(\theta)$ we had to put a $-$because in the definition we have $-K$.
\end{eg}

\begin{prop}
	Let $X\sim EF(\theta)$ then
	\begin{enumerate}
		\item $\e[A(X)]=\frac{K'(\theta)}{Q'(\theta)}$
		\item $Var(A(X))=\frac{K(\theta)}{(Q'(\theta))^2}-\frac{Q''(\theta)}{(Q'(\theta))^2}\frac{K'(\theta)}{Q'(\theta)}$
	\end{enumerate}
\end{prop}
Note that this proposition gives us only the expectation and variance for $A(X)$, but it is not a problem because usually $A(X)=X$.
\begin{proof}
	\begin{enumerate}
		\item because the exponential family is regular we can use Leibniz so \[
		\begin{split}
		0
		&=\frac{d}{d\theta} 1\\
		&=\frac{d}{d\theta}\int f_X(x;\theta)dx\\
		&=\int \frac{d}{d\theta} f_X(x;\theta)dx\\
		&=\int (A(x)Q'(\theta)-K'(\theta))f_X(x;\theta)dx\\
		&=Q'(\theta)\int A(x) f_X(x;\theta)dx-K'(\theta)\int f_X(x;\theta)dx\\
		&=Q'(\theta)\e[A(X)]-K'(\theta)
		\end{split}
		\]
		\[\implies \e[A(X)]=\frac{K'(\theta)}{Q'(\theta)}\]
		\item because the exponential family is regular we can use Leibniz so
		\[
		\begin{split}
		0
		&=\frac{d^2}{d\theta^2}\int f_X(x;\theta)\\
		&=\int \frac{d^2}{\theta^2} f_X(x;\theta)\\
		&=\int (A(x)Q''(\theta)-K''(\theta))f_X(x;\theta)+(A(x)Q'(\theta)-K'(\theta))^2f_X(x;\theta)dx\\
		&=Q''(\theta)\e[A(X)]-K''(\theta)+(Q'(\theta))^2\int \bigg( A(x)-\frac{K'(\theta)}{Q'(\theta)} \bigg)^2 f_X(x;\theta)dx\\
		&=Q''(\theta)\frac{K'(\theta)}{Q'(\theta)}-K''(\theta)+(Q'(\theta))^2\int ( A(x)-\e[A(X)] )^2 f_X(x;\theta)dx\\
		&=Q''(\theta)\frac{K'(\theta)}{Q'(\theta)}-K''(\theta)+(Q'(\theta))^2Var(A(X))\\
		\end{split}
		\]
		\[
		\implies Var(A(X))=\frac{K(\theta)}{(Q'(\theta))^2}-\frac{Q''(\theta)}{(Q'(\theta))^2}\frac{K'(\theta)}{Q'(\theta)}=\frac{K(\theta)}{(Q'(\theta))^2}-\frac{Q''(\theta)}{(Q'(\theta))^2}\e[A(X)]
		\]
	\end{enumerate}
\end{proof}
\begin{oss}
	If $Q(\theta)=\theta$ we get$$\e[A(X)]=K'(\theta)$$
	$$Var(A(X))=K''(\theta)$$
\end{oss}
\section{Natural Exponential Families}
\begin{defi}
	We say that the distribution of a \rs $(x_1...x_n)$ from $X\sim NEF(\theta)$ is an element of a \textbf{Natural Exponential Family}	$X\sim NEF(\theta)$ if its PDF can be written as follow:
	$$f_X(x;\nu)=\exp \{ \nu x + C(x) -K(\nu) \}$$
\end{defi}



      \newpage
      \chapter{Statistics}
\vspace{15pt}



The notation of statistic was introduced by Fisher (1920).\\
The importance of sufficiency is that ot can be found in any statistical decision (point estimation, testing, confidential bound)
\begin{defi}
	Let $X=(X_1... X_n)$ be a random sample from a parametric model $X\sim f_X(x,\theta)$ for some $\theta \in \Theta$ unknown.\\
	We say that $T_n=T(X)$ is \textbf{sufficient for the parameter $\theta$} if the conditional distribution of $X$ given $T_n$ does not depend of $\theta$ i.e. defined:
	\begin{itemize}
		\item $f_{\uX|T_n=t}(\ux;t,\theta)$ the conditional distribution of $\uX$ given $T_n$
		\item $h_{\uX,T_n}(z,t,\theta)$ the joint distribution of $\uX$ and $T_n$
		\item $g_{T_n}(t,\theta)$ the marginal distribution of $T_n$
	\end{itemize}
then $T_n$ is \textbf{sufficient for the parameter $\theta$} only if $f_{\uX|T_n=t}(\ux;t,\theta)$ does not depend of $\theta$.\\
Note that
\[
f_{\uX|T_n=t}(\ux;t,\theta)=\frac{h_{\uX,T_n}(\ux,t,\theta)}{g_{T_n}(t,\theta)}
\]
\end{defi}
\begin{eg} \label{eg:ber}
	$(X_1... X_n)\in \{0,1\}^n$ from a $Ber(\theta)$, $\theta \in (0,1)$.\\
	Define $T_n=\sum_{i=1}^{n}X_i$, then we want to verify if $T_n$ is sufficient.\\
	\begin{itemize}
		\item $f_{\uX}(\ux;\theta) = \prod_{i=1}^{n} f_{{X_i}}({x_i};\theta)=\theta^{\sum_{i=1}^{n}x_i}(1-\theta)^{n-\sum_{i=1}^{n}x_i}$
		\item$g_{T_n}(t,\theta)={{n}\choose{t}}\sigma^t(1-\sigma)^{n-t}\mathbbm{1}_{0,1...n}(t)$
		\item$h_{\uX,T_n(z,t,\theta)}=\p(\uX=\ux,T_n=t)=\sigma^t(1-\sigma)^{n-t}$
	\end{itemize}
	so 
	$$f_{\uX|T_n=t}=\frac{\sigma^t(1-\sigma)^{n-t}}{{{n}\choose{t}}\sigma^t(1-\sigma)^{n-t}}=\frac{1}{{{n}\choose{t}}}$$
	So $T_n$ is a sufficient statistic for $\theta$.\\
	This is a really special case because all the $X_i$ are already in function of $T_n$.
\end{eg}

\begin{oss}
	If $T_n$ is sufficient for $\theta$ then all the statistical information of $\theta$ contained in the random sample is relocated in $T_n$. In the example above to infer about $\theta$ we just need $\sum_{i=1}^{n}X_i$.\\
\end{oss}
\begin{oss}
	The notation of sufficiency derive from the probability structure of the parametric family $X\sim f_X(x;\theta)$. We can talk about sufficiency for a parameter $\theta$ only after we have specified $X\sim f_X(x;\theta)$
\end{oss}
The definition of sufficiency based on conditional probability is not of practical use because we need this two distributions $
\begin{cases}
g_{T_n}(\cdot)\\
h_{\uX,T_n}(\cdot,\cdot)
\end{cases}
$ that can be difficult to find.
To avoid that we could use a corollary of the \textit{Fisher Factorization Theorem}:
\begin{corol}\label{corol:Savage}
	Let $\uX=(X_1... X_n)$ from $X\sim f_x(x,\theta)$. Then a statistic $T_n$ is sufficient for $\theta$ if and only if there exist two non negative functions $g(\cdot), h(\cdot)$ such that $\lf=g(T(\ux);\theta)h(\ux)$
\end{corol}
\begin{oss}
	\begin{itemize}
		\item $g$ is a function of the observed sample via $T_n$
		\item $h$ is a function of the observed sample and does not depend on $\theta$
 	\end{itemize}
\end{oss}
\begin{eg}
	Recall the example \ref{eg:ber} 
		$(X_1... X_n)\in \{0,1\}^n$ from a $Ber(\theta)$, $\theta \in (0,1)$.\\
	Define $T_n=\sum_{i=1}^{n}X_i$, then we want to verify if $T_n$ is sufficient. \\
	We have
	 $$f_{\uX}(\ux;\theta) = \prod_{i=1}^{n} f_{{X_i}}(x_i;\theta)=\theta^{\sum_{i=1}^{n}x_i}(1-\theta)^{n-\sum_{i=1}^{n}x_i}$$
	hence we can apply the previous theorem \ref{corol:Savage} using 
	\begin{enumerate}
		\item $h(\uX)=1$ \item $g(T_n(\uX);\theta)=\theta^{\sum_{i=1}^{n}x_i}(1-\theta)^{n-\sum_{i=1}^{n}x_i}$
	\end{enumerate}
\end{eg}
\begin{eg}\label{eg:gauss}
		$(X_1... X_n)\in \{0,1\}^n$ from a $N(\theta,1)$. We want to verify that $T_n=\sum_{i=1}^{n} X_i$ is a sufficient statistic:
		\[
		\begin{split}
			\lf
			&=\prod_{i=1}^{n}\frac{1}{\sqrt{2 \pi}} \exp \bigg\{ -\frac{1}{2} (x_i-\theta)^2 \bigg\}\\
			&=(2\pi)^{-n/2}\exp \bigg\{ -\frac{1}{2} \sum_{i=1}^{n}(x_i-\theta)^2 \bigg\}\\
			&=(2\pi)^{-n/2}\exp \bigg\{ -\frac{1}{2} \sum_{i=1}^{n}x_i^2- \frac{n\theta^2}{2}+\theta\sum_{i=1}^{n} x_i \bigg\}\\
			&=(2\pi)^{-n/2}\exp \bigg\{ -\frac{1}{2} \sum_{i=1}^{n}x_i^2 \bigg\} \exp \bigg\{- \frac{n\theta^2}{2}+\theta\sum_{i=1}^{n} x_i \bigg\}
		\end{split}
		\]
		so
		\begin{itemize}
			\item $h(x)=\exp \bigg\{ -\frac{1}{2} \sum_{i=1}^{n}x_i^2 \bigg\}$
			\item $g(\sum_{i=1}^n x_i,\theta) = \exp \bigg\{- \frac{n\theta^2}{2}+\theta\sum_{i=1}^{n} x_i \bigg\}$
		\end{itemize}
\end{eg}

\begin{teo}\textbf{Fisher Theorem}\\
	If $\lfd$ is the joint density function or the joint probability mass function of $\uX$ and $q(t;\theta)$ is the density function or the probability mass function of $T_n(\uX)$, then $T_n(\uX)$ is sufficient for $\theta$ if for every point in the sample space, the ratio
	$$\frac{\lfd}{q(t;\theta)}$$ 
	is a constant function of $\theta$.
\end{teo}
\begin{proof}
	MISSING
\end{proof}
We can see now the prof of the corollary \ref{corol:Savage}
\begin{corol}\textbf{Savage}\\
	Let $\lfd$ be the joint PDF or PMF of a random sample $\uX=(X_1... X_n)$. A statistic $T_n$ is sufficient for $\theta$ if and only if there exist two non negative functions $g(t,\theta), h(\ux)$ such that for all $\ux$ in the sample space and for all $\theta\in \Theta$ 
	$$\lfd=g(T(\ux);\theta)h(\ux)$$
\end{corol}
\begin{proof}
	We are going to prove the theorem only in the discrete settings.\\
	\begin{itemize}
		\item["$\Rightarrow$"] Suppose that $T(\uX)$ is sufficient for $\theta$.\\
		Define:
		\begin{itemize}
			\item $g(t,\theta):=\p(T(\uX)=t)$
			\item $h(\ux):=\p\bigg(\uX=\ux \bigg|T(\uX)=T(\ux)\bigg)$
		\end{itemize}
	Because $T(\uX)$ is sufficient for $\theta$ the conditional probability defining $h(\ux)$ does not depend on $\theta$. Hence the choice of $g(t,\theta)$ and $h(\ux)$ is legitimate and for this choice we have
	\[
	\begin{split}
	\p(\uX=\ux)
	&=\p(\uX=\ux \wedge T(\uX)=T(\ux))\\
	&=\p(T(\uX)=T(\ux))\p(\uX=\ux|T(\uX)=T(\ux))\\
	&=g(t,\theta)h(\ux)
	\end{split}
	\]
	So we have the factorization and in particular we can see that
	$$\p\bigg(T(\uX)=T(\ux)\bigg)=g(t,\theta)$$
	$\implies g(T(\ux),\theta)$ is the PMF of $T(s)$
	\item["$\Leftarrow$"] We assume that the factorization holds.\\
	Let $q(t,\theta)$ be the PMF of $T(\uX)$. We study the ratio
	$$\frac{\lfd}{q(T(\ux);\theta)}$$
	in particular define
	$$A_{T(\ux)}=\{\underline y | T(\underline y)= T(\ux) \}$$
	Then 
	\[
	\begin{split}
	\frac{\lfd}{q(T(\ux);\theta)}
	&=\frac{g(T(\ux);\theta)h(\ux)}{q(T(\ux);\theta)}\\
	&=\frac{g(T(\ux);\theta)h(\ux)}{\sum_{\underline y \in A_{T(\ux)}}{g(T(\ux);\theta)h(\underline y)}}\\
	&=\frac{g(T(\ux);\theta)h(\ux)}{g(T(\ux);\theta) {\sum_{\underline y \in A_{T(\ux)}}h(\underline y)}}\\
	&=\frac{h(\ux)}{\sum_{\underline y \in A_{T(\ux)}}h(\underline y)}
	\end{split}
	\]
	This is constant \wrt $\theta$.\\
	Then by the Fisher Theorem $T(\uX)$ is sufficient for $\theta$.
	\end{itemize}

\end{proof}

\begin{eg}$(X_1... X_n)\in \{0,1\}^n$ from a $N(\mu,\sigma^2)$, $\sigma^2$ known. As we did in the example \ref{eg:gauss} we want to find if $T_n=\frac{1}{n}\sum_{i=1}^{n} X_i$ is a sufficient statistic for $\mu$.
	\[
	\begin{split}
	f_{\uX}(\ux;\mu\sigma^2)
	&=(2 \pi \sigma^2)^{-n/2} \exp \bigg\{ -\frac{1}{2\sigma^2}\sum_{i=1}^{n} (x_i-\mu)^2 \bigg\}\\
	&=(2 \pi \sigma^2)^{-n/2} \exp \bigg\{ -\frac{1}{2\sigma^2} \sum_{i=1}^{n}(x_i-\bar x_n -\bar x_n -\mu)^2 \bigg\}\\
	&=(2 \pi \sigma^2)^{-n/2} \exp \bigg\{ -\frac{1}{2\sigma^2}\bigg( \sum_{i=1}^{n} (x_i-\bar x_n)^2 + n(\bar x_n-\mu)^2  \bigg) \bigg\}\\
	\end{split}
	\]
	we already know the distribution of $T(\ux)=\bar x_n=\frac{1}{n}\sum_{i=1}^{n}x_i$ is $\bar X_n\sim N(\mu,\sigma^2/n)$.\\
	So we can apply Fisher theorem to the ratio:
	\[
	\frac{(2 \pi \sigma^2)^{-n/2} \exp \bigg\{ -\frac{1}{2\sigma^2}\bigg( \sum_{i=1}^{n} (x_i-\bar x_n)^2 + n(\bar x_n-\mu)^2  \bigg) \bigg\}}{(2 \pi \sigma^2/n)^{-1/2}\exp \bigg\{- \frac{n}{2\sigma^2} (\bar x_n -\mu)^2 \bigg\}}
	\]
		\[
	\frac{(2 \pi \sigma^2)^{-n/2} \exp \bigg\{ -\frac{1}{2\sigma^2} \sum_{i=1}^{n} (x_i-\bar x_n)^2   \bigg\} \exp \bigg\{ -\frac{n}{2\sigma^2}(\bar x_n-\mu)^2  \bigg\}}{(2 \pi \sigma^2/n)^{-1/2}\exp \bigg\{- \frac{n}{2\sigma^2} (\bar x_n -\mu)^2 \bigg\}}
	\]
	\[
	\frac{(2 \pi \sigma^2)^{-n/2} \exp \bigg\{ -\frac{1}{2\sigma^2} \sum_{i=1}^{n} (x_i-\bar x_n)^2   \bigg\} }{(2 \pi \sigma^2/n)^{-1/2}}
	\]
	Hence by Fisher Theorem $T_n=\frac{1}{n}\sum_{i=1}^{n} X_i$ is sufficient for $\mu$
\end{eg}
\begin{oss}
	Until now we found only one sufficient statistic for a fixed parametric model. However we can define many sufficient statistics.\\
	For example the statistic given by the identity $T(\ux)=\ux$ is always a sufficient statistic, indeed we can factorize the distribution $f_X(x,\theta)$ with
	\begin{itemize}
		\item $h(x)=1$
		\item $g(T(x);\theta)=f_X(x,\theta)$
	\end{itemize}
\end{oss}


\begin{oss}
 Given one sufficient statistic a way to produce more sufficient statistics is thru a one to one function.\\
Suppose $T(\ux)$ is a sufficient statistic for $\theta$, and define $T^*(\ux)=r(T(\ux))$ where $r$ is a one to one function with inverse $r^{-1}$.\\
By Savage's Theorem there exist $g,h$ such that
$$\lf=g(T(\ux),\theta)h(\ux)=g(r^{-1}(r(T(\ux),\theta)))h(\ux)=g(r^{-1}(T^*(\ux)),\theta)h(\ux)$$
So defining $g^*(t,\theta)=g(r^{-1}(t),\theta)$ we have that
$$\lf=g^*(T^*(\ux),\theta)h(\ux)$$
$\implies$ by Savage Theorem we have that $T^*(\ux)$ is a sufficient statistic.
\end{oss}
We saw that in principle we can define many sufficient statistics so it is natural to define a tool that allows us to decide when a sufficient statistic is better than another.\\
Recall that the purpose of statistic is to achieve data reduction without loss of information.\\
Therefore a statistic that achieve the most data reduction while still retaining all of the information about $\theta$ might be preferable.
\begin{oss}
	We saw in example \ref{eg:gauss} that if 	$(X_1... X_n)\in \{0,1\}^n$ from a $N(\theta,1)$,  $T_n=\sum_{i=1}^{n} x_i$ is a sufficient statistic. Instead of $\sum_{i=1}^{n} X_i$ we can use $T'(\ux)\bigg( \sum_{i=1}^{n} x_i, \sum_{i=1}^{n} x_i^2 \bigg)$. Clearly $T(X)$ s a greater data reduction than $T'(\ux)$ since we do not need to know the sample variance if we want to know $\theta$. Moreover we can write $T(\ux)$ as a function of $T'(\ux)$ by defining the function $r(a,b)=a$, then we can write
	$$T(\ux)=\bar x_n=r(\bar x_n, S^2_n)=r(T'(\ux))$$
	Since $T(\ux)$ and $T'(\ux)$ are both sufficient they contains the same information about $\mu$.
	In other terms the additional information given by the sample variance is null.
\end{oss} 
\begin{defi}
	A sufficient statistic $T(\ux)$ is called \textbf{minimal} if for any other sufficient statistic $T'(\ux)$, $T(\ux)$ is a function of $T'(\ux)$. 
\end{defi}
NOTE:\\
To say that $T(\ux)$  is a function of  $T'(\ux)$ simply means that if $T'(x)=T'(y)$ then $T(x)=T(y)$.\\
In other terms if $\{ B_t\} $ where $B_t:=\{t' : T'(t)=T'(t') \}$ is the partition set induced by $T'$ and $\{ A_t\} $ where $A_t:=\{t' : T(t)=T(t') \}$ is the partition set induced by $T$ then for every $t$, $B_t \subseteq A_t$.\\
$\implies$the partition of the sample space induced by a minimal statistic is the partition with the smallest cardinality.
\begin{teo}\textbf{Lehmann and Sheffe}
	let $\lfd$ be the joint density function or joint probability mass function of a \rs \  $\uX=(X_1...X_n)$. Suppose there exist a function $T$ such that for any two sample points $\ux$ , $\uy$ the ratio 
	\[
	\frac{\lfd}{f_{\uX}(\uy;\theta)}
	\]
	is constant as a function of $\theta$ if and only if $T(\ux)=T(\uy)$
	
	
	Then $T$ is a minimal sufficient statistic for $\theta$ 
\end{teo}

\begin{proof}
	To simplify the proof we assume $\lfd > 0 \ \forall \ux, \forall \theta$.\\
	First we show that $T(\ux)$ is sufficient.\\
	Define $\Tau$ as the image of the sample space under the function $\st$.
	$$\Tau:=\{ t:t=\st \text{for some $\ux$ in the sample space} \}$$
	Define $\{ A_t\} $ the partition set induced by $T$, where $A_t:=\{t' : T(t)=T(t') \}$ 
	For each $A_t$  choose and fix some elements $x_t\in A_t$. For any point in the space $\ux_{\st}$ is the fixed element that is in the same set ,$A_t$, as $\ux$. Since $\ux$ and $\ux_{\st}$ are in the same set $A_t$ then $\st=T(\ux_{\st})$ so by the assumptions the ratio
	\[
	\frac{\lfd}{f_{\uX}(\ux_{\st};\theta)}
	\]
Does not depend on $\theta$. Thus we can define $h(\ux):=\frac{\lfd}{f_{\uX}(\ux_{\st};\theta)}$.\\
Then define the function $g(\ux,\theta)\lfd$, so we have:
\[
\lfd =\frac{f_{\uX}(\ux_{\st};\theta)\lfd}{f_{\uX}(\ux_{\st};\theta)}=\gf h(\theta)
\]
and by Savage Theorem $\st$ is sufficient for $\theta$.\\


Now we will show that $\st$ is minimal sufficient.\\
Let  $T'(\ux)$ be another sufficient statistic. By Savage Theorem we know that exist $h',g'$ such that
\[
\lfd=g'(T'(\ux),theta)h'(\theta)
\]
Let $\ux,\uy$ be two sample points such that $T'(\ux)=T'(\uy)$ then we can study the ratio:
\[
\frac{\lfd}{f_{\uX}(\uy;\theta)}=\frac{g'(T'(\ux);\theta)h'(\ux)}{g'(T'(\uy);\theta)h'(\uy)}=\frac{h'(\ux)}{h'(\uy)}
\]
Since the ratio does not depend on $\theta$, by the assumption (the other implication of the IIF) implies $\st =T(\uy)$. So we can say that $T(\ux)$ is a function of $T'(\ux)$ therefore $\st$ is minimal.
\end{proof}
\section{Estimators}

\begin{defi}
	Suppose there is a fixed parameter $\theta$ that needs to be estimated. Then an \textbf{estimator} is a function that maps the sample space to a set of sample estimates. An estimator of $\theta$ is usually denoted by the symbol $\bar \theta$.
\end{defi}
Now we re going to introduce some definition of \textit{"good"} estimators.
\begin{defi}
	$T_n(\uX)$ is said to be \textbf{unbiased} for $\theta$ if $\e[T_n(\uX)]=\theta$
\end{defi}
\begin{oss}
	we use the expected value to define a "good"v estimator because of the linearity of the operator.
\end{oss}
\begin{defi}
	\textbf{Bias}:
	\[
		Bias_\theta (T_n(\uX))=\e\bigg[T_n(\uX)-\e[T_n(\uX)]\bigg]
	\]
\end{defi}

When we ask an estimator to be unbiased basically we are requiring it to be centred around $\theta$.\\
Another parameter that give us information about the goodness of an estimator is the variance. We can interpret the variance as a measure of the dispersion around the expected value, so before check the variance we mus be sure that the expected value overlap with $\theta$. In this scenario the less is the variance the best is the estimator.
\begin{oss}
	Variance is a good parameter to watch only if the estimator is unbiased
\end{oss}
To avoid this problem we can introduce the \textit{Mean Squared Error}
\begin{defi}\textbf{Mean Squared Error}(MSE)
	\[
	\e[(T_n(\uX)-\theta)^2]
	\]
\end{defi}
The importance of this quantity comes from the \textit{Chebyshev's Inequality} \ref{eq:Chebyshev}
\[
\p(|T_n(\ux)-\theta|< k)>1-\frac{\e[(T_n(\uX)-\theta)^2]}{k^2}
\]
Indeed we notice the smaller the MSE the greater is $\p(|T_n(\ux)-\theta|< k)$.
\begin{prop}
	$$\e[(T_n(\uX)-\theta)^2]=Var(T_n(\uX))+Bias_\theta(T_n(\uX))$$
\end{prop}
\begin{proof}
	\[
	\begin{split}
		\e[(T_n(\uX)-\theta)^2]
		&=\e[(T_n(\uX) - \e[T_n(\uX)] + \e[T_n(\uX)]-\theta)^2 ]\\
		&=\e[(T_n(\uX)-\e[T_n(\uX)])^2]+\e[(\e[T_n(\uX)]-\theta)^2]+2\e[(T_n(\uX)-\e[T_n(\uX)])(\e[T_n(\uX)]-\theta)]\\
		&=\e[(T_n(\uX)-\e[T_n(\uX)])^2]+\e[(\e[T_n(\uX)]-\theta)^2]\\
		&=Var(T_n(\uX))+Bias_\theta(T_n(\uX))
	\end{split}
	\]
\end{proof}
\begin{oss}
	If $\e[T_n(\uX)]=\theta$ then $MSE(T_n(\uX))=Var(T_n(\uX))$.
\end{oss}
\begin{defi}
	Let $X_1.. X_n$ from $X\sim f_X(x,\theta)$, $T_n'$ and $T_n''$ estimators for $\theta$. We say $T_n'$ is  \textbf{more efficient} than $T_n''$ if
	$$MSE(T_n')<MSE(T_n'')$$
\end{defi}
Usually we choose the estimator with the lower MSE even if it is biased.\\

\section{Properties of Estimators}
The problem of the MSE is that we can note be sure that there exist $T_n$ such that $MSE (\tn)$ is the lowest possible.\\
A solution for this comes from
\begin{teo}\label{teo:Cramer-Rao Bound}
	\textbf{Cramer-Rao Bound}\\
			Let $X=(X_1... X_n)$ be a random sample from a parametric model $X\sim f_X(x,\theta)$.\\
			Then, under condition of regularity, for any estimator $\tn$ of $\theta$ 
			\[
			Var(\tn)\geq \frac{[1+b'(\tn)]^2}{\ifn}
			\]
Where
\begin{itemize}
	\item[$b(\tn)$] is the bias of $\tn$
	\item[$\ifn$] is the Fisher Information
\end{itemize}
\end{teo}
\begin{proof}
	consider the estimator $\tn$.
	\begin{itemize}
		\item $\e[\tn]=\theta +b(\theta)$
		\item $\der \e[\tn]=1+b'(\theta)$
		\item $\e[V_n'(\theta)]=0 \leftarrow$ because we suppose our model regular
	\end{itemize}
$\implies Cov(\tn, V_n'(\theta))=\e[\tn V_n'(\theta)] - \e[\tn]\e[V_n'n]=\e[\tn V_n'(\theta)]$ 
\[
\begin{split}
\e[\tn T_n'(\theta)]&
=\int_{\mathbb{R}^n}\tn V_n'(\theta) \lfd d\ux\\
&=\int_{\mathbb{R}^n}\tn \frac{f_{\uX}'(\ux,\theta)}{\lfd}\lfd d\ux\\
&=\int_{\mathbb{R}^n}\tn \der \lfd d\ux\\
&=\der \int_{\mathbb{R}^n}\tn \lfd d\ux\\
&=\der \e[\tn]\\
&=1+b'(\theta)
\end{split}
\]
So
\[
Cov(\tn,V_n'(\theta))=1+b'(\theta)
\]
We know that in general for $X,Y$ \rv s such that $\e[X]=\mu, \e[Y]=\nu, \e[X^2]<\infty, \e[Y^2]<\infty$ it holds
\[
\begin{split}
(Cov(X,Y))^2&=\bigg( \e[(X-\mu)(T-\nu)] \bigg)^2\\
&\leq  \e[(X-\mu)]^2 \e[(T-\nu)]^2\\
&= Var(X)Var(Y)
\end{split}
\]
So replacing $X$ with $\tn$ and $Y$ with $V_n'(\theta)$ we  obtain
\[
\begin{split}
Var(\tn)&\geq \frac{(Cov(\tn,V_n'(\theta)))^2}{Var(V_n'(\theta))}\\
&=\frac{(Cov(\tn,V_n'(\theta)))^2}{\e[(V_n'(\theta)-\e[V_n'(\theta)])^2]}\\
&=\frac{(Cov(\tn,V_n'(\theta)))^2}{\e[V_n'(\theta)]^2}\\
&=\frac{(Cov(\tn,V_n'(\theta)))^2}{\ifn}
\end{split}
\]
\end{proof}
\begin{corol}
	Under condition of regularity
	\[MSE \geq \frac{(1+b'(\theta))^2}{\ifn} +b^2(\theta) \]
\end{corol}
\begin{proof}
	Directly from Cramer-Rao Bound \ref{teo:Cramer-Rao Bound} remembering that $MSE(\tn)=Var(\tn)+b^2(\theta)$
\end{proof}

\begin{corol}
	Let $\uX=(X_1... X_n)$ be a random sample from a regular model $X\sim f_X(x;\theta)$\\
	If there exist a unbiased estimator for $\theta$ whose variance is equal to the Cramer-Rao bound, Then $\tn$ is unique
\end{corol}
\begin{proof}
	Take $T_{1n}$, $\tnd$ be unbiased estimators for $\theta$ such that
	\[
	Var(\tnu)=Var(\tnd)=\frac{1}{\ifn}
	\]
	Define $\tn := \frac{\tnu + \tnd}{2}$\\
	$\e[\tn]=\frac{\e[\tnu]+\e[\tnd]}{2}=\frac{2}{2}\theta=\theta$\\
	$\implies \tn$ is also unbiased for $\theta$\\
	$\implies Var(\tn)\geq \frac{1}{\ifn}$
	\[
	\begin{split}
	Var(\tn)
	&=Var\bigg( \frac{\tnu + \tnd}{2}  \bigg)\\
	&=\frac{1}{4}\bigg[ Var(\tnu)+ Var(\tnd) +2 Cov(\tnu, \tnd) \bigg]\\
	&=\frac{1}{4 }\bigg[ Var(\tnu)+ Var(\tnd) +2 Cov(\tnu, \tnd) \bigg]\frac{[Var(\tnu)Var(\tnd)]^{1/2}}{[Var(\tnu)Var(\tnd)]^{1/2}}\\
	&=\frac{(1+Corr(\tnu,\tnd))}{2}\frac{1}{\ifn}
	\end{split}
	\]
	\\MEMO:$|Corr(X,Y)|\leq 1$\\
	Because $Var(\tn)\leq\frac{1}{\ifn}$ then we must have  $Corr(\tnu,\tnd)\geq 1 \implies Corr(\tnu,\tnd)= 1$.\\
	$\implies \tnd=a+b\tnu$.\\
	Hence we must have $\theta=\e[\tnu]=\e[a+b\tnu]=\e[a]+b\e[\tnu]=a+b\theta\implies a=0,b=1$\\
	$\implies \tnd=\tnu$
\end{proof}
\begin{defi}
	Consider a regular model $X\sim f_x(x\theta)$. We say that an unbiased estimator $\tn$ whose variance is 
	\[
	Var(\tn)=\frac{1}{\ifn}
	\]
	is \textbf{efficient} moreover we define the \textbf{efficiency} of an estimator as:
	\[
	Eff(\tn)=\frac{1}{Var(\tn)\ifn} \ \ \ \ \ \ \in [0,1]
	\]
\end{defi}
\begin{oss}
	\begin{enumerate}
		\item We introduce (absolute) efficiency at the cost of assuming regularity for the parametric model.
		\item The variance of an unbiased estimator $\tn$ of $\theta$ can not be smaller than the Cramer-Rao bound. However we do not know if there exist an estimator whose variance is equal to the Cramer-Rao bound.
		\item The proper lower bound involves the MSE
		$$MSE(\tn)\geq\frac{[1+b'(\theta)]}{\ifn}+b^2(\theta)$$ 
\end{enumerate}
\end{oss}
\begin{prop}
	Let $\uX=(X_1...X_n)$ be a random sample from a parametric model  $X\sim f_x(x\theta)$. Let $\tn$ be a unbiased estimator for $\theta$. Than $\tn$ i efficient for $\theta$ if and only if
	$$V_n'(\theta)=\ifn (T_n-\theta)$$
\end{prop}
\begin{proof}
	\[
	\begin{split}
	V_n'(\theta)=\ifn (\tn-\e[\tn])&
	\iff 	V_n'(\theta)^2=\ifn^2 (\tn-\e[\tn])^2\\
	&\iff 	\e[V_n'(\theta)^2]=\e[\ifn^2 (\tn-\e[\tn])^2]\\
	&i.e. \ \ifn=\ifn^2 \e[(\tn-\e[\tn])^2]\\
	&i.e. \ 1=\ifn Var(\tn)\\
	&\iff Var(\tn)=\frac{1}{\ifn}
	\end{split}
	\]
	i.e. $\tn$ is efficient.
\end{proof}
\begin{teo}
	\textbf{Rao-Blackwell}\\
	Let $\uX=(X_1 ... X_n)$ a random sample from a parametric model  $X\sim f_x(x\theta)$. Let
	\begin{itemize}
		\item $\tnu$ a sufficient estimator for $\theta$
		\item $\tnd$ a unbiased estimator for $\theta$
		\item $\tn:=\e[\tnd|\tnu]$
	\end{itemize}
Then
\begin{enumerate}
	\item $\tn$ is a function of $\tnu$
	\item $\e[\tn]=\theta$
	\item $Var(\tn)<Var(\tnd)$
\end{enumerate}
\end{teo}

\begin{defi}
	Let $X_1, X_2...$ real valued  \rv s with CDF $F_{X_1},F_{X_2}...$. We say that $(X_n)_n$ \textbf{converges in distribution} or \textbf{converges weakly} to a \rv  \  $X$ if $$\lim_{n\to \infty}F_{X_n}(x)=F_X(x)$$
\end{defi}
\begin{defi}
	A sequence $\{ X_n\}_n$ of \rv s \textbf{converges in probability} towards the \rv \ $X$ if for all $\epsilon>0$
	$$\lim_{n\to \infty}\p(|X_n-X|>\epsilon)=0$$
	We write $X_n\xrightarrow{p}X$
\end{defi}
\begin{defi}
	Given a real number $r \geq 1$, we say that the sequence $\{X_n\}$ converges in the \textbf{r-th mean} (or \textbf{in the $L^r$-norm}) towards the random variable $X$, if the r-th absolute moments $\e(|X_n|^r)$ and $\e(|X|^r)$ of $X_n$ and $X$ exist, and
	$$\lim_{n\to \infty}\e[|X_n-X|^r]=0$$
	We write $X_n \xrightarrow{L^r} X$
	For $r=2$ we say that $\{X_n\}$ converges in \textbf{mean square}  to $X$.
\end{defi}

\begin{prop}
	Convergence in probability $\implies$ convergence in distribution.\\
	If $X$ is a degenerate \rv \ we have also\\  convergence in distribution $\implies$ convergence in probability.
\end{prop}
\begin{defi}
	We say an estimator $\tn$ is \textbf{is consistent in mean squared} for $\theta$ if 
	$$\lim_{n\to \infty}MSE(\tn)=0$$
\end{defi}
NOTATION: we will use $b(\tn)=b(\theta)$
\begin{oss}
	Since $MSE(\tn)=Var(T_n)+b^2(\tn)$, then then $\lim_{n\to \infty}MSE(\tn)=0$ is equivalent to say
	\begin{itemize}
		\item  $\lim_{n\to \infty}Var(\tn)=0$
		\item $\lim_{n\to \infty}b^2(\tn)=0$
	\end{itemize}
\end{oss}
\begin{defi}
	$T_n$ is \textbf{asymptotically unbiased} for $\theta$ if
	\begin{itemize}
		\item $\lim_{n\to \infty}\e[\tn]=\theta$
		\item $\lim_{n\to \infty}b(\tn)=0$
	\end{itemize}
\end{defi}
\begin{prop}
	A consistent estimator in mean square is also asymptotically unbiased
\end{prop}

\begin{defi}
	We say $\tn$ is consistent in probability for $\theta$ if for all $\epsilon >0$
	$$\lim_{n\to \infty}\p(|\tn-\theta|<\epsilon)=1$$
\end{defi}
\begin{prop}
	The  consistencyin mean squared implies consistency in probability 
\end{prop}
\begin{proof}
	Using Chebyshev's inequality \ref{eq:Chebyshev}
	$$\p(|\tn-\theta|<\epsilon)\geq 1-\frac{MSE(\tn)}{\epsilon^2}$$
	$$\lim_{n\to \infty}\p(|\tn-\theta|<\epsilon)\geq 1-\lim_{n\to \infty}\frac{MSE(\tn)}{\epsilon^2}$$
\end{proof}
\begin{teo}
	\textbf{Central Limit Teorem}\\
	Let $\uX=(X_1... X_n)$ be a random sample of size $n$ with $X_i$ independent and identically distributed \rv s. With  expected value $\mu$ and finite variance $\sigma^2$. Then $S_n:=\frac{\sum_{i=1}^n}{n}$ converges in probability to the expected value $\mu$
	$$S_n\xrightarrow{p} \mu$$
\end{teo}

\begin{teo}
	\textbf{Week law of large numbers}\\
	Let $\uX=(X_1... X_n)$ be a random sample of size $n$ with $X_i$ independent and identically distributed \rv s. With  expected value $\mu$.
	Define $\bar X_n=\frac{1}{n}\sum_{i=1}^{n}X_i$, then
	$$\bar X_n \xrightarrow{p} \mu$$
	That is, for any $\epsilon>0$
	$$\lim_{n\to \infty}\p(|\bar X_n- \mu|>\epsilon)=0$$
\end{teo}



      \newpage
      \chapter{Exercises}
\label{cha:ex}
\vspace{15pt}
\begin{ex}
	Let $X_1$ an $X_2$ two \rv s independent and uniformly distributed on the interval $[0,1]$. Find the distribution of:
	\begin{enumerate}
		\item $Y=X_1+X_2$
		\item $W=\frac{X_1}{X_2}$
		\item $Z=X_1X_2$
	\end{enumerate}
	
	
	\textit{Solution:}
	\begin{enumerate}
		\item Let's start with the sum of two generic \rv s:\\
		we know that $f_{Y|X_1}(y|x_1)=f_{X_2}(y-x)$ and the joint distribution on two random variables is:
		\[
		f_{X_1,Y}(x_1,y)=f_{Y|X_1}(y|x_1)f_{X_1}(x_1)
		\]
		so in our case:
		\[
		f_{X_1,Y}(x_1,y)=f_{X_2}(y-x_1)f_{X_1}(x_1)
		\]
		and now we can calculate the PDF of $Y$ simply by integrating the PDF of the joint distribution:
		\[
		\begin{split}
		f_{Y}(y)
		&=\int_{- \infty}^{\infty}f_{X_1,Y}(x_1,y)dx_1\\
		&= \int_{- \infty}^{\infty}f_{X_2}(y-x_1)f_{X_1}(x_1)dx_1
		\end{split}
		\]
		(For a more detailed analysis see the convolution product).\\
		Now we can proceed replacing he generic PDF with the one of a uniform distribution on the interval $[0,1]$ is $\mathbbm{1}_{[0,1]}(t)$, we have:
		\[
		\begin{split}
		f_{Y}(y)
		&= \int_{- \infty}^{\infty}\mathbbm{1}_{[0,1]}(y-x_1)\mathbbm{1}_{[0,1]}(x_1)dx_1\\
		&= \int_{0}^{1}\mathbbm{1}_{[0,1]}(y-x_1)dx_1
		\end{split}
		\]
		and by separating the integral in various cases we can solve it obtaining:
		\[
		f_{Y}(y)=
		\begin{cases}
		0 \ \  if \ y < 0 \\
		y \ \  if \  y\in[0,1]\\
		2-y \ if \  y\in(1,2]\\
		0 \ \ if \  y>2
		\end{cases}
		\]
		\item for the distribution of $W$ we will use the CDF function:\\
		fir of all it is easy to prove that for$w\leq 0,F_W(w)=0$, so in the next passage we can assume $w>0$ 
		\[
		\begin{split}
		F_W(w)
		&=\p [W<w]\\
		&=\p[\frac{X_2}{X_1}<w]\\
		&=\p[\frac{X_2}{X_1}<w,X_1>0]+\p[\frac{X_2}{X_1}<w,X_1<0]\\
		&=\p[X_2<X_1w,X_1>0]+0\\
		&=\int_{0}^{\infty}f_{x_1}(x_1)\int_{-\infty}^{wx_1}f_{x_2}(x_2)dx_2dx_1\\
		&=\int_{0}^{\infty}\mathbbm{1}_{[0,1]}(x_1)\int_{-\infty}^{wx_1}\mathbbm{1}_{[0,1]}(x_2)dx_2dx_1\\
		&=\int_{0}^{1}	\begin{cases}
		1 \ \ \ \ \  if \ wx_1>1 \\
		wx_1 \ \  if \  wx_1<1
		\end{cases}dx_1\\
		&=\begin{cases}
		\int_0^1 wx_1 dx \ \ \ \ \ \ \ \ \ \ \ \ \ \ \ \ \ \ if \ w\leq 1\\
		\int_{0}^{\frac{1}{w}} wx_1dx_1+\int_{\frac{1}{w}}^{1} 1\ dx_1 \ \  if \ w>1
		\end{cases} \\
		&=\begin{cases}
		\frac{w}{2} \ \ \ if \ 0\leq w\leq 1\\
		1-\frac{1}{2w} \ \  if \ w>1
		\end{cases}
		\end{split}
		\]
		\item for the distribution of $Z$ we will adopt a more straight forward approach an we will use the CDF function:
		\[
		\begin{split}
		F_Z(z)
		&=\p [Z<z]\\
		&=\p[X_1X_2<z]\\
		&=\p[X_1X_2<z,X_1>0]+\p[X_1X_2<z,X_1<0]\\
		&=\p[X_2<\frac{z}{X_1},X_1>0]+0\\
		&=\int_{0}^{\infty}f_{X_1}(x_1)\int_{-\infty}^{z/x_1}f_{X_2}(x_2)dx_2 dx_1\\
		&=\int_{0}^{\infty}\mathbbm{1}_{[0,1]}(x_1)\int_{-\infty}^{z/x_1}\mathbbm{1}_{[0,1]}(x_2)dx_2 dx_1\\
		&=\int_{0}^{\infty} \mathbbm{1}_{[0,1]}(x_1)
		\begin{cases}
		1 \ \ \ \  if \ \frac{z}{x_1}>1 \\
		\frac{z}{x_1} \ \  if \  \frac{z}{x_1}<1
		\end{cases}dx_1\\
		&=\int_{0}^{1}
		\begin{cases}
		1 \ \ \ \  if \ \frac{z}{x_1}>1 \\
		\frac{z}{x_1} \ \  if \  \frac{z}{x_1}<1
		\end{cases}dx_1\\
		&=\begin{cases}
		\int_{0}^{z}1\ dx_1+ \int_{z}^{1} \frac{z}{x_1} dx_1 \ \  if  \ z<1 \\
		\int_0^1 1 \ \ \ \ \ \ \ \ \ \ \ \ \ \ \ \ \ \ \ \ \ \ if \ z \geq 1
		\end{cases}\\
		&=\begin{cases}
		z-z\ln(z) \  \   if \  0<z<1\\
		1 \ \ \ \ \ \ \ \ \ \ \ \ \ if \ \ z\geq 1
		\end{cases}.
		\end{split}
		\]
	\end{enumerate}
\end{ex}

\begin{ex}
	Consider $X_1,X_2,X_3$ tree \rv s \iid with distribution $\sim Exp(\frac{1}{2})$.\\
	Find the distribution of:
	\begin{enumerate}
		\item $U=\frac{X_2}{X_1}$
		\item $W=\sum_{i=1}^{3}X_1$
	\end{enumerate}
	(in this exercise we will consider the PDF of the exponential $f_X=\lambda e^{-\lambda x}$).

\textit{Solution:}
\begin{enumerate}
	\item For the distribution of $U$ we will use the CDF function:\\
	first of all it is easy to prove that for$u\leq 0,F_U(u)=0$, so in the next passage we can assume $u>0$ 
	\[
	\begin{split}
	F_(u)
	&=\p [U<u]\\
	&=\p[\frac{X_2}{X_1}<u]\\
	&=\p[\frac{X_2}{X_1}<u,X_1>0]+\p[\frac{X_2}{X_1}<u,X_1<0]\\
	&=\p[X_2<X_1u,X_1>0]+0\\
	&=\int_{0}^{\infty}f_{x_1}(x_1)\int_{-\infty}^{ux_1}f_{x_2}(x_2)dx_2dx_1\\
	&=\int_{0}^{\infty}\lambda e^{-\lambda x_1}\int_{0}^{ux_1}\lambda e^{-\lambda x_2}dx_2dx_1\\
	&=\int_{0}^{\infty}\lambda e^{-\lambda x_1}(1-e^{-\lambda ux_1})x_1\\
	&=1-\frac{1}{1+u}
	\end{split}
	\]
	\item Here we will use the moment-generating function to demonstrate that $W\sim Gamma(\alpha =3, \beta=\frac{1}{2})$.\\
	The the moment-generating function of the exponential with parameter $2$ is: $M_X(t)=\frac{\frac{1}{2}}{\frac{1}{2}-t}$.
	\[
	\begin{split}
	M_W(t)
	&=\e [e^{wt}]\\
	&=\e [e^{\sum_{i=1}^{3}X_it}]\\
	&=\e [\prod_{i=1}^{3}e^{X_it}]\\
	&=\prod_{i=1}^{3} \e [e^{X_it}]\\
	&=\prod_{i=1}^{3} \frac{\frac{1}{2}}{\frac{1}{2}-t} \\
	&=\bigg( \frac{\frac{1}{2}}{\frac{1}{2}-t} \bigg) ^3\\
	&=\bigg( \frac{\frac{1}{2}-t}{\frac{1}{2}} \bigg)^{-3}\\
	&=\bigg( 1-\frac{t}{\frac{1}{2}} \bigg)^{-3}
	\end{split}
	\]
	which is the MGF of a $Gamma\big( 3,\frac{1}{2} \big)$
\end{enumerate}
\end{ex}
\begin{ex}
	Let $(X,Y)$ be a bivariate \rv such that $X\sim U(-1,1)$ and $Y|X\sim U(x,x+1)$. Find he distribution of $Z=-\ln(Y-X)$

\textit{Solution:}
\[
\begin{split}
\p (Z<z)
&=\p (-log(X-Y)<t)\\
&=\p(log(X-Y)\geq z)\\
&=\p(Y\geq X+ e^{-z})\\
&=\int_{-1}^{1}\int_{x+e^{-z}}^{x+1} \frac{1}{2} dxdy\\
&=1-e^{-z}
\end{split}
\]
Which is the distribution function of neg exp
\end{ex}
\begin{ex}
	Let $A,B$ be two \iid \rv s with distribution $\sim U(0,h)$. Compute the probability that the equation $Z^2 -2AZ+B=0$ doesn't admit real solutions.

\textit{Solution:}
We are asked to compute $\p(A^2-B<0)$.
\end{ex}
\begin{ex}
	Let $X\sim Gamma(r,1), Y\sim Gamma(s,1)$ independent \rv s. Find the distribution of
	\begin{enumerate}
		\item $W:=X+Y$
		\item $Z:=\frac{X}{W}$
		\item $(Z,w)$
	\end{enumerate}

\textit{Solution:}
Did in class.
\end{ex}
\begin{ex}
	Let $X_1,X_2,X_3$ \rv s with distribution:
	\begin{itemize}
		\item $X_1 \sim Gamma (\alpha_1,1)$
		\item $X_2 \sim Gamma (\alpha_2,1)$
		\item $X_3 \sim Gamma (\alpha_3,1)$
	\end{itemize}
	Define:
	\begin{itemize}
		\item $Z=\frac{X_1}{X_1+X_2+X_3}$
		\item $W=\frac{X_2}{X_1+X_2+X_3}$
	\end{itemize}
	Find the distribution of $(Z,W)$



For this exercise we will use the theorem on page 165 of \cite{Casella}

%%%%%%%%%%%%%%%%%%%%%%%%%%%%%%%%%%%%%%%%%%%%%%%%%%%%%%%%%%%%%%%%%%%%%%%%%%%%%%%%%%%%%

\iffalse
For this exercise we will use the following theorem:

\begin{teo}
	Let $n\leq N$ positive integers, $C \subset \mathbb{R}^{n}$ a compact, $\phi:C\to \mathbb{R}^N$ a $(n,N)$ regular parametrization and $f:\phi (C)\to \mathbb{R}$ a continuous function, then:
	$$\int_{\phi(C)} f d\mathcal{H}^n= \int_{C}(f \circ \phi) J_{\phi}d \mathcal{L}^n$$
	Where
	$$J_{\phi}:=(det[D_{\phi}^T \times D_{\phi}])^{1/2}\not = 0$$
\end{teo}
${H}^n$ and $\mathcal{L}^n$ indicates the Hausdorff and Lebesgue function and
\[
D_{\phi}=\begin{bmatrix}
\frac{\partial \phi_1}{\partial x_1} & \dots  & \frac{\partial \phi_1}{\partial x_n} \\
\vdots &  \ddots & \vdots \\
\frac{\partial \phi_N}{\partial x_1} &  \dots  & \frac{\partial \phi_N}{\partial x_n}
\end{bmatrix}	
\]
$$\phi (W,Z,S)=(ZS,WS,S-S(Z+W))$$

\fi
%%%%%%%%%%%%%%%%%%%%%%%%%%%%%%%%%%%%%%%%%%%%%%%%%%%%%%%%%%%%%%%%%%%%%%%%%%%%%%%%%

\textit{Solution:}
To have a $(3,3)$ parametrization we will add another \rv $S:=X_1+X_2+X_3$.
Consider the parametrization
\begin{itemize}
\item $X_1=ZS$
\item $X_2=WS$
\item $X_3=S-S(Z+W)$
\end{itemize}
The Jacobian matrix is defined as:
\[J:=\begin{bmatrix}
\frac{\partial \phi_1}{\partial x_1} & \dots  & \frac{\partial \phi_1}{\partial x_n} \\
\vdots &  \ddots & \vdots \\
\frac{\partial \phi_N}{\partial x_1} &  \dots  & \frac{\partial \phi_N}{\partial x_n}
\end{bmatrix}	
\]
in our situation then:
\[
|J|=\Bigg| \begin{bmatrix}
S & 0  & Z \\
0 &  S & W \\
-S &  -S  & 1-Z-W
\end{bmatrix}	
\Bigg|=S^2
\]
So by the previous theorem we have:
\[f_{Z,W,S}(z,w,s) = f_{X_1,X_2,X_3}(zs,ws,s-s(z+w)) |J|
\]
Remembering that $X_1,X_2,X_3$ are independent then $f_{X_1,X_2,X_3}(zs,ws,s-s(z+w))=f_{X_1}(zs)f_{X_2}(ws)f_{X_3}(s-s(z+w))$
so
\[
\begin{split}
f_{Z,W,S}(z,w,s)
& =f_{X_1}(zs)f_{X_2}(ws)f_{X_3}(s-s(z+w))s^2\\
&=\frac{(zs)^{\alpha_1-1}(ws)^{\alpha_2-1}(s-s(z+w))^{\alpha_3-1}e^{- s}}{\Gamma(\alpha_1)\Gamma(\alpha_2)\Gamma(\alpha_3)} s^2 \mathbbm{1}_{[\min(zs,ws,s-s(z+w)),\infty)}(0)\\
&=\frac{z^{\alpha_1-1}w^{\alpha_2-1} (1-(z+w))^{\alpha_3-1}}{\Gamma (\alpha_1)\Gamma (\alpha_2)\Gamma (\alpha_3)}s^{\alpha_1+\alpha_2+\alpha_3-1} e^{-s}
\end{split}
\]
Notice that the second member is the kernel of a Gamma distribution with parameters $\alpha_1+\alpha_2+\alpha_3, 1$.\\
We know that $s\geq 0$ so $\min(zs,ws,s-s(z+w))$ has the same sign of $\min(z,w,1-(z+w))$.\\
To get the distribution of $Z,W$ we have to integrate $f_{Z,W,S}(z,w,s)$ with respect to $s$:
\[
\begin{split}
f_{Z,W}(z,w)
&=\int_{0}^{\infty}f_{Z,W,S}(z,w,s)\mathbbm{1}_{[\min(z,w,1-(z+w)),\infty)}(0)ds\\
&=\mathbbm{1}_{[\min(z,w,1-(z+w)),\infty)}(0)\int_{0}^{\infty}\frac{z^{\alpha_1-1}w^{\alpha_2-1} (1-(z+w))^{\alpha_3-1}}{\Gamma (\alpha_1)\Gamma (\alpha_2)\Gamma (\alpha_3)}s^{\alpha_1+\alpha_2+\alpha_3-1} e^{-s}ds\\
&=\mathbbm{1}_{[\min(z,w,1-(z+w)),\infty)}(0)\frac{z^{\alpha_1-1}w^{\alpha_2-1} (1-(z+w))^{\alpha_3-1}}{\Gamma (\alpha_1)\Gamma (\alpha_2)\Gamma (\alpha_3)} \int_{0}^{\infty}s^{\alpha_1+\alpha_2+\alpha_3-1} e^{-s}ds\\
&=\frac{z^{\alpha_1-1}w^{\alpha_2-1} (1-(z+w))^{\alpha_3-1}}{\Gamma (\alpha_1)\Gamma (\alpha_2)\Gamma (\alpha_3)}\Gamma(\alpha_1+\alpha_2+\alpha_3)\mathbbm{1}_{[\min(z,w,1-(z+w)),\infty)}(0)
\end{split}
\]
\end{ex}
\begin{ex}
	Show that the moment generating function of a \rv $X\sim NEF(\nu)$ is
	$$\e[e^{sx}]=e^{K(s+\nu)-K^\nu}$$
\textit{Solution:}
\[
\begin{split}
e[e^{sx}]
&=\int e^{sx}e^{x\nu+C(x)-K(\nu)}dx=\int e^{x(s+\nu)+C(x)+K(s+\nu)-K(s+\nu)-K(\nu)}dx\\
&=e^{K(s+\nu)-K(\nu)}\int e^{x(s+\nu)+C(x)-K(s+\nu)}dx\\
&=e^{K(s+\nu)-K(\nu)} 1\\
&=e^{K(s+\nu)-K(\nu)}
	\end{split}
	\]
\end{ex}



\begin{ex}
	We want to estimate the proportion $\theta$ of individuals in a population for which a certain feature $X$ takes value in $A$. We take a sample from the population of size $n$ and we measure the feature $X$. Let $Z_n$ be the number of individuals with feature $X$ in $A$. 
	\begin{itemize}
		\item
		Is $\frac{Z_n}{n}$ unbiased for $\theta$? Is it consistent ? Is asymptotically Gaussian ?
		\item
		Find a maximum likelihood estimator for $\theta$. 
	\end{itemize}
\end{ex}



\begin{ex}
	Consider a finite population of individuals such that only $40\%$ of individuals survive after one week. After one week we check the population and we find $r$ individuals. We want to estimate the size of population.
\end{ex}  

\begin{ex}
	$(X_1,X_2,\ldots,X_n)$ from $X$ $(f_x(x,\theta)$ where $f_x(x,\theta)=2\theta x exp\{ -\theta x^2\} 1_{\mathbb{R}}(x), \quad \theta > 0$.
	\begin{itemize}
		\item
		Find the ML estimator for $\theta$
		\item
		Find a sufficient statistic for $\theta$
		\item
		Find the moment estimator for $\theta$
		\item
		Compare the estimators.
	\end{itemize}
\end{ex}

      
    \endgroup

	%mostra sempre

    % bibliografia in formato bibtex
    %
    % aggiunta del capitolo nell'indice
    \addcontentsline{toc}{chapter}{Bibliografia}
    % stile con ordinamento alfabetico in funzione degli autori
    \bibliographystyle{plain}
    \bibliography{biblio}
%%%%%%%%%%%%%%%%%%%%%%%%%%%%%%%%%%%%%%%%%%%%%%%%%%%%%%%%%%%%%%%%%%%%%%%%%%
%%%%%%%%%%%%%%%%%%%%%%%%%%%%%%%%%%%%%%%%%%%%%%%%%%%%%%%%%%%%%%%%%%%%%%%%%%
%% Nota
%%%%%%%%%%%%%%%%%%%%%%%%%%%%%%%%%%%%%%%%%%%%%%%%%%%%%%%%%%%%%%%%%%%%%%%%%%
%% Nella bibliografia devono essere riportati tutte le fonti consultate 
%% per lo svolgimento della tesi. La bibliografia deve essere redatta 
%% in ordine alfabetico sul cognome del primo autore. 
%% 
%% La forma della citazione bibliografica va inserita secondo la fonte utilizzata:
%% 
%% LIBRI
%% Cognome e iniziale del nome autore/autori, la data di edizione, titolo, casa editrice, eventuale numero dell’edizione. 
%% 
%% ARTICOLI DI RIVISTA
%% Cognome e iniziale del nome autore/autori, titolo articolo, titolo rivista, volume, numero, numero di pagine.
%% 
%% ARTICOLI DI CONFERENZA
%% Cognome e iniziale del nome autore/autori (anno), titolo articolo, titolo conferenza, luogo della conferenza (città e paese), date della conferenza, numero di pagine. 
%% 
%% SITOGRAFIA
%% La sitografia contiene un elenco di indirizzi Web consultati e disposti in ordine alfabetico. 
%% E’ necessario:
%%   Copiare la URL (l’indirizzo web) specifica della pagina consultata
%%   Se disponibile, indicare il cognome e nome dell’autore, il titolo ed eventuale sottotitolo del testo
%%   Se disponibile, inserire la data di ultima consultazione della risorsa (gg/mm/aaaa).    
%%%%%%%%%%%%%%%%%%%%%%%%%%%%%%%%%%%%%%%%%%%%%%%%%%%%%%%%%%%%%%%%%%%%%%%%%%
%%%%%%%%%%%%%%%%%%%%%%%%%%%%%%%%%%%%%%%%%%%%%%%%%%%%%%%%%%%%%%%%%%%%%%%%%%
    

    \titleformat{\chapter}
        {\normalfont\Huge\bfseries}{Allegato \thechapter}{1em}{}
    % sezione Allegati - opzionale
    \appendix
    %


\end{document}
