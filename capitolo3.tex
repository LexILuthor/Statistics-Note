\chapter{Likelihood Function}
\label{cha:likf}
\vspace{15pt}



The \textit{likelihood function} is a function that contains all the statistical information required to make inference.\\
\begin{defi}
	Consider a random sample  $(X_1...X_n)$ from $ X \sim f_X(X; \theta)$, then the distribution of $(X_1..X_n)$ will be:
	$$f_{\underline X}(\underline x;\theta)=\prod_{i=1}^{n}f_{X_i}(x_i;\theta)$$
	when we see $f_{\underline X}(\underline x;\theta)$ as a function of $\theta$ for fixed $\underline x$ we call it \textbf{likelihood function}
	$$\mathcal{L}(\theta,\underline x)=\prod_{i=1}^{n}f_{X_i}(x_i;\theta)$$
\end{defi}
	An important function related to the likelihood function is the \textbf{log likelihood function} 
\begin{defi}
	The $\ln$ of the likelihood function is said \textit{log likelihood function}
	$$V_n(\theta)=\ln\mathcal{L}(\theta,\underline x)=\ln \bigg( \prod_{i=1}^{n}f_{X_i}(x_i;\theta) \bigg)=\sum_{i=1}^{n}\ln( f_{X_i}(x_i;\theta))$$
\end{defi}
\begin{defi}\label{defi:scoref}
	\textbf{Score function}:
	$$V'_n=\frac{d}{d \theta}V_n(\theta)=\frac{\mathcal{L}'(\theta, \ux)}{\mathcal{L}(\theta, \ux)}$$
\end{defi}
Note that if we fix $\theta$ then $\mathcal{L}(\theta,\underline x)$ is (related to) the probability that the particular value we fixed for $\theta$ has generated $\underline x$.\\
Suppose we fix two value $\theta,\theta_2\in \Theta$ and 
$\mathcal{L}(\theta_1,\underline x)>\mathcal{L}(\theta_2,\underline x)$ then
we say $\ux$ is \textit{"more likely"} generated under $\theta_1$.\\
Note that the same meaning is also applicable to the log likelihood function.\\
It is because of this that we usually search for the maximum of the likelihood function. Often to find it we just derive, but sometimes $\mathcal{L}$ is not regular enough so we have to \textit{"regularize"} it.
\section{Likekihood principles}
The statistical inference based on the likelihood function is a consequence of two principles:
\begin{enumerate}
	\item \textbf{Week likelihood principle:} for a fixed parametric model $X\sim F_X(x,\theta)$ if two observed samples $\ux$ and $ \underline{y}$ are such that $$\mathcal{L}(\theta,\underline x) \propto \mathcal{L}(\theta,\underline y)$$ 
	then the two likelihood functions are equivalent i.e. must produce the same inference result on $\theta$.
	\item \textbf{Strong likelihood principle:} let $\ux$ be an observed sample under the model $X\sim F_X(x,\theta)$ with likelihood function $\mathcal{L}(\theta,\underline x)$ and let $\underline y$ be an observed sample under the model $X\sim F_Y(y,\theta)$ with likelihood function $\mathcal{L}(\theta,\underline y)$.\\
	If $\mathcal{L}(\theta,\underline x) \propto \mathcal{L}(\theta,\underline y)$ the the two samples provides with the same inference.\\
	
	
	The fundamental difference between \textit{Probability} and \textit{Statistic} is that in the first one the goal is to find the chance of a \rv \ to take a particular value, statistic instead given the results of a experiment, try to find the distribution where it came from.\\
	\begin{eg}
		Take $(X_1,X_2,X_3)$ from one of the following distribution:
		\begin{enumerate}
			\item $X\sim Ber(\theta_1), \theta_1=\frac{1}{2}$
			\item $X\sim Ber(\theta_2), \theta_2=\frac{1}{3}$
			\item $X\sim Ber(\theta_3), \theta_3=\frac{1}{4}$
		\end{enumerate}
	$X\in [0,1]$ $\Theta = [0,1]$.\\
	We can imagine $(x_1,x_2,x_3)$ as the results of a experiment where we had to flip a single coin 3 times. Now we want to know the parameter $\theta$ of the coin we flipped tree times and we have tree possibilities:$\theta_1=\frac{1}{2}, \theta_2=\frac{1}{3}, \theta_3=\frac{1}{4}$ .
	\\So $(x_1,x_2,x_3)\in \{0 ,1 \}^3$
	
	\begin{center}
		\begin{tabular}{ | c | c | c | c | }
			\hline
			$x_1,x_2,x_3$ & $\theta_1 = \frac{1}{2}$ & $\theta_2=\frac{1}{3}$ & $\theta_3=\frac{1}{4}$ \\ \hline
			0,0,0 & $\frac{1}{8}$ & $\frac{8}{27}$ & $\frac{27}{64} \cdot$ \\ \hline
			0,0,1 & $\frac{1}{8}$ & $\frac{4}{27} \cdot$ & $\frac{9}{64}$ \\ \hline
			0,1,0 & $\frac{1}{8}$ & $\frac{4}{27} \cdot$ & $\frac{9}{64}$ \\ \hline
			1,0,0 & $\frac{1}{8}$ & $\frac{4}{27} \cdot$ & $\frac{9}{64}$ \\ \hline
			0,1,1 & $\frac{1}{8} \cdot$ & $\frac{2}{27}$ & $\frac{3}{64}$ \\ \hline
			1,0,1 & $\frac{1}{8} \cdot$ & $\frac{2}{27}$ & $\frac{3}{64}$ \\ \hline
			1,1,0 & $\frac{1}{8} \cdot$ & $\frac{2}{27}$ & $\frac{3}{64}$ \\ \hline
			1,1,1 & $\frac{1}{8} \cdot$ & $\frac{1}{27}$ & $\frac{1}{64}$ \\
			\hline
		\end{tabular}
	\end{center}
	Once we know the result of the throw we will "guess" the value of $\theta$ choosing the one that give us more probability for the given result.
	\end{eg}
\end{enumerate}
\section{Condition of Regularity}
In our investigations on $\theta$ we will assume some condition of regularity for our model.\\
Given $X\sim F_x(x,\theta)$
\begin{enumerate}
	\item we assume that $\theta \in \Theta$, where $\Theta$  is a open real set
	\item for any $\theta \in \Theta$, there exist the  derivative of $\mathcal{L}(\theta;z)$ \wrt $\theta$ at least up to the third order
	\item for any $\theta_0 \in \Theta$, there exist three functions $g,h,H,$ that are integrable in a neighborhood of $\theta_0$  and 
	\begin{itemize}
		\item $\bigg| \frac{d}{d\theta} f_X(x,\theta) \bigg|\leq g(x)$
		\item $\bigg| \frac{d2}{d^2\theta} f_X(x,\theta) \bigg|\leq h(x)$
		\item $\bigg| \frac{d^3}{d^3\theta} \ln(f_X(x,\theta)) \bigg|\leq H(x)$
	\end{itemize}
	\item for any $\theta \in \Theta$
	$$0<\e[(\ln(\mathcal{L}(\theta, \uX)))^2]<\infty$$
\end{enumerate}
In addition there is the condition of identifiability.
\begin{itemize}
	\item[5.] We say that a statistical model is identifiable if for every  $\theta_1,\theta_2$ there is al least one event $E$ such that:
	$$\p(X\in E | \theta_1)\not =\p(X\in E | \theta_2)$$
\end{itemize}
(We will always take 5. as granted)
\section{Properties of the Likelihood Function}
\begin{prop}
	\begin{enumerate}Some properties of the score function \ref{defi:scoref} are:
		\item $\e[V_n'(\theta)]=0$
		\item $Var(V_n'(\theta))=\e[(V_n'(\theta))^2])=- \e[V_n''(\theta)]$
	\end{enumerate}
\end{prop}
\begin{proof}
	\begin{enumerate}
		\item 
		\[
			\begin{split}
			\e[V_n'(\theta)]
			&=\int_{\mathbb{R}^n} V_n'(\theta)f_{\uX}(\ux, \theta) dx\\
			&=\int_{\mathbb{R}^n}\frac{f_{\uX}'(\ux, \theta)}{f_{\uX}(\ux, \theta)}f_{\uX}(\ux, \theta)dx\\
			&=\int_{\mathbb{R}^n} \frac{d}{d\theta}f_{\uX}(\ux, \theta))dx\\
			&= \frac{d}{d\theta}\int_{\mathbb{R}^n}f_{\uX}(\ux, \theta))dx\\
			&= \frac{d}{d\theta} 1\\
			&=0
			\end{split}
		\]
		Where in the in the fourth equal we used Leibniz and for the fifth recall that $f_{\uX}(\ux, \theta)$ is the PDF of $\uX$
		\item Start by showing that $V_n''(\theta)=\frac{f_{\uX}''(\ux;\theta)}{f_{\uX}(\ux;\theta)}-\bigg( \frac{f_{\uX}'(\ux;\theta)}{f_{\uX}(\ux;\theta)} \bigg)^2$
		\[
		\begin{split}
		V_n''(\theta)
		&=\frac{d^2}{d \theta^2} \ln\mathcal{L}(\theta,\ux)\\
		&=\frac{d}{d\theta}\frac{f_{\uX}'(\ux;\theta)}{f_{\uX}(\ux;\theta)}\\
		&=\frac{f_{\uX}''(\ux;\theta)f_{\uX}(\ux;\theta)-f_{\uX}'(\ux;\theta)f_{\uX}'(\ux;\theta)}{|f_{\uX}(\ux;\theta)|^2}\\
		&=\frac{f_{\uX}''(\ux;\theta)}{f_{\uX}(\ux;\theta)}-\bigg( \frac{f_{\uX}'(\ux;\theta)}{f_{\uX}(\ux;\theta)} \bigg)^2\\
		&=\frac{f_{\uX}''(\ux;\theta)}{f_{\uX}(\ux;\theta)}-(V_n'(\theta))^2
		\end{split}		
		\]
		So now
		\[
		\begin{split}
			\e[V_n''(\theta)]
			&=\int_{\mathbb{R}^n}\frac{f_{\ux}''(\ux;\theta)}{f_{\ux}(\ux;\theta)}f_{\ux}(\ux;\theta)dx-\e[V_n'(\theta)^2]\\
			&=\int_{\mathbb{R}^n}\frac{d^2}{d\sigma^2}f_{\uX}(\ux;\theta)dx-\e[V_n'(\theta)^2]\\
			&=\frac{d^2}{d\sigma^2}\int_{\mathbb{R}^n}f_{\uX}(\ux;\theta)dx-\e[V_n'(\theta)^2]\\
			&=-\e[V_n'(\theta)^2]
		\end{split}
		\]
	\end{enumerate}
\end{proof}

\begin{defi}
	we define the \textbf{Fisher Information} as:
	$$\ifn=-\e[V_n''(\theta)]$$
\end{defi}
Note that this is the definition of Fisher information just for a particular case, there exist a more general one.\\
The Fisher information has a central role in statistic because it can be shown that for \textit{unbiased estimators} $\tilde \theta$ it holds: $Var(\tilde \theta)\geq \frac{1}{\ifn}$. So i we can find an estimator with  $Var(\tilde \theta)= \frac{1}{\ifn}$ we are sure that it is the one with the lowest variance.\\
\begin{prop}
	Consider \rsf \  regular then
	$$\ifn=n\mathcal{I}_1(\theta)$$
\end{prop}
\begin{eg}
	Consider $(x_1...x_n)$ from $X\sim Ber(\theta)$
	$$\lf = \prod_{i=1}^{n} f_{\underline{X_i}}(\underline{x_i};\theta)=\theta^{\sum_{i=1}^{n}x_i}(1-\theta)^{n-\sum_{i=1}^{n}x_i}$$
	$$V_n(\theta)=\ln(\lf)=\ln(\theta)\sum_{i=1}^{n}x_i+(\ln(1-\theta))\bigg(n-\sum_{i=1}^{n}x_i\bigg)$$
	$$V_n'(\theta)=\frac{\sum_{i=1}^{n}x_i}{\theta}-\frac{n- \sum_{i=1}^{n}x_i}{1-\theta}$$
	$$V_n''(\theta)=\frac{-\sum_{i=1}^{n}x_i}{\theta^2}-\frac{-\sum_{i=1}^{n}x_i}{(1-\theta)^2}$$
	\[
	\begin{split}
	\e[V_n'(\theta)]
	&	=\frac{1}{\theta}\sum_{i=1}^{n}\e[x_i]-\frac{1}{1-\theta}\bigg( n- \sum_{i=1}^{n}\e[x_i]\bigg)\\
	&=\frac{n\not \theta}{\not \theta}-\frac{1}{1-\theta}(n-n\theta)\\
	&=n-n=0
	\end{split}
	\]
	\[
	\begin{split}
		\e[V_n''(\theta)]
		&=-\frac{1}{\theta}\sum_{i=1}^{n}\e[X_i]-\frac{1}{1-\theta}\bigg( n- \sum_{i=1}^{n}\e[X_i] \bigg)\\
		&=-\frac{n\theta}{\theta^2}-\frac{n-n\theta}{(1-\theta)^2}\\
		&=-\frac{n}{\theta}-\frac{n(1-\theta)}{(1-\theta)^2}\\
		&=-\frac{n\theta}{(1-\theta)\theta}
	\end{split}
	\]
	\[
	\begin{split}
	\ifn
	&=-\e[V_n''(\theta)]\\
	&=\frac{n}{(1-\theta)\theta}
	\end{split}
	\]
\end{eg}

\section{Exponential Families}
\begin{defi}
	We say that the distribution of a \rv \  is an element of an \textbf{Exponential Family}	$X\sim EF(\theta)$, if its PDF can be written as follows:
	$$f_X(x;\theta)=\exp \{ Q(\theta)A(x)+ C(x) -k(\theta) \}$$
\end{defi}
The generalization to \rs becomes:
\begin{defi}
	We say that the distribution of a \rs $(x_1...x_n)$ from $X\sim EF(\theta)$ is an element of an \textbf{Exponential Family}	$X\sim EF(\theta)$ if its PDF can be written as follow:
	$$\lfd=\exp \{ Q(\theta)\sum_{i=1}^{n}A(x_i)+ \sum_{i=1}^{n} C(x_i) -nK(\theta) \}$$
\end{defi}
\begin{eg}
	$X\sim Ber(\theta),X\in\{0,1\}, \Theta=(0,1)$
	\[
	\begin{split}
	P_X(x)
	&=\theta^x (1-\theta)^{1-x}\mathbbm{1}_{\{0,1\}}(x)\\
	&=\exp\{x\ln(\theta)+(1-x)\ln(1-\theta)  \}\mathbbm{1}_{\{0,1\}}(x)\\
	&=\exp\{x\ln(\theta)+\ln(1-\theta)-x\ln(1-\theta)  \}\mathbbm{1}_{\{0,1\}}(x)\\
	&=\exp\bigg\{xln{\bigg(\frac{\theta}{1-\theta}\bigg)}+\ln(1-\theta)\bigg\}\mathbbm{1}_{\{0,1\}}(x)
	\end{split}
	\]
	so we get 
	\begin{itemize}
		\item[$Q(\theta)$]$=\ln\bigg( \frac{\theta}{1-\theta} \bigg)$
		\item[$A(x)$]$=x$
		\item[$C(x)$]$=0$
		\item[$K(\theta)$]$=-\ln(1-\theta)$
	\end{itemize}
	Note that for $K(\theta)$ we had to put a $-$because in the definition we have $-K$.
\end{eg}

\begin{prop}
	Let $X\sim EF(\theta)$ then
	\begin{enumerate}
		\item $\e[A(X)]=\frac{K'(\theta)}{Q'(\theta)}$
		\item $Var(A(X))=\frac{K(\theta)}{(Q'(\theta))^2}-\frac{Q''(\theta)}{(Q'(\theta))^2}\frac{K'(\theta)}{Q'(\theta)}$
	\end{enumerate}
\end{prop}
Note that this proposition gives us only the expectation and variance for $A(X)$, but it is not a problem because usually $A(X)=X$.
\begin{proof}
	\begin{enumerate}
		\item because the exponential family is regular we can use Leibniz so \[
		\begin{split}
		0
		&=\frac{d}{d\theta} 1\\
		&=\frac{d}{d\theta}\int f_X(x;\theta)dx\\
		&=\int \frac{d}{d\theta} f_X(x;\theta)dx\\
		&=\int (A(x)Q'(\theta)-K'(\theta))f_X(x;\theta)dx\\
		&=Q'(\theta)\int A(x) f_X(x;\theta)dx-K'(\theta)\int f_X(x;\theta)dx\\
		&=Q'(\theta)\e[A(X)]-K'(\theta)
		\end{split}
		\]
		\[\implies \e[A(X)]=\frac{K'(\theta)}{Q'(\theta)}\]
		\item because the exponential family is regular we can use Leibniz so
		\[
		\begin{split}
		0
		&=\frac{d^2}{d\theta^2}\int f_X(x;\theta)\\
		&=\int \frac{d^2}{d\theta^2} f_X(x;\theta)\\
		&=\int (A(x)Q''(\theta)-K''(\theta))f_X(x;\theta)+(A(x)Q'(\theta)-K'(\theta))^2f_X(x;\theta)dx\\
		&=Q''(\theta)\e[A(X)]-K''(\theta)+(Q'(\theta))^2\int \bigg( A(x)-\frac{K'(\theta)}{Q'(\theta)} \bigg)^2 f_X(x;\theta)dx\\
		&=Q''(\theta)\frac{K'(\theta)}{Q'(\theta)}-K''(\theta)+(Q'(\theta))^2\int ( A(x)-\e[A(X)] )^2 f_X(x;\theta)dx\\
		&=Q''(\theta)\frac{K'(\theta)}{Q'(\theta)}-K''(\theta)+(Q'(\theta))^2Var(A(X))\\
		\end{split}
		\]
		\[
		\implies Var(A(X))=\frac{K(\theta)}{(Q'(\theta))^2}-\frac{Q''(\theta)}{(Q'(\theta))^2}\frac{K'(\theta)}{Q'(\theta)}=\frac{K(\theta)}{(Q'(\theta))^2}-\frac{Q''(\theta)}{(Q'(\theta))^2}\e[A(X)]
		\]
	\end{enumerate}
\end{proof}
\begin{oss}
	If $Q(\theta)=\theta$ we get$$\e[A(X)]=K'(\theta)$$
	$$Var(A(X))=K''(\theta)$$
\end{oss}
\section{Natural Exponential Families}
\begin{defi}
	We say that the distribution of a \rs $(x_1...x_n)$ from $X\sim NEF(\theta)$ is an element of a \textbf{Natural Exponential Family}	$X\sim NEF(\theta)$ if its PDF can be written as follow:
	$$f_X(x;\nu)=\exp \{ \nu x + C(x) -K(\nu) \}$$
\end{defi}


