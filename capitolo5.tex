\chapter{Hypothesis Testing}
\vspace{15pt}

$\uX=(X_1 ... X_n)$ a random sample from $X\sim f_X(x;\theta)$ $\theta \in \Theta$. We want to use the observed sample to evaluate $\theta$.\\
First we need a bit of terminology.
\begin{defi}
	 a \textbf{statistical test} is defined as a decision rule on the sample space. Using this decision rule and the observed sample we decide to accept or reject an hypothesis about $\theta$.\\
	 The hypothesis we decide to accept or refuse is called \textbf{ the null hypothesis $H_0$} 
\end{defi} 
\begin{defi}
	We define a \textbf{statistical hypothesis} any sentence that specifies a partition of the statistical model $\{ f_x(X;\theta): \te \in \Theta \}$.\\If the specification of the statistical model is complete then the statistical hypothesis is called \textbf{simple $\te=\te^*$}.\\If the specification is partial is called composite, in particular $\te >\theta^*, \theta \in R^*\subset \Theta$ is called \textbf{composite unidirectional hypothesis}.\\
	We denote by $\omega_0 \subset \Theta$ the set of values that are specified by the null statistical hypothesis \h .\\
	We want	to heck if data support hypothesis of this type:
	\begin{itemize}
		\item[\h]:$\theta\in \Theta$
		\item[$H_1$]:$\te \not \in \Theta$
	\end{itemize}
\end{defi}
The null statistical hypothesis is something that exist before the experiment. It is an hypothesis that exist until proved otherwise. The alternative of \h (denoted by $H_1$) is the complement of \h.\\ The decision rule about accepting or rejecting \h has different interpretations, in particular:
\begin{itemize}
	\item if we decide, based on data, to reject \h, this has a unique consequence
	\item if we decide, based on data, to accept \h this does not imply the support of \h.
\end{itemize}
The hypothesis test is a decision on the sample space and this decision rule is characterized by a function defined on the set of all the possible value of $(X_1 ...X_n)$. There will be some points $\uX \in R_0\subset \mathbb{R}^n$ such that the decision rule leads o reject \h and some other points such that the decision rule leads to do not reject \h.\\
\begin{defi}
	the subspace $R_0$ is called \textbf{rejection region for \h}.
\end{defi}
The decision rule is determined by the following criteria:\begin{itemize}
	\item if $x\in R_0$ then reject \h
	\item if $x\not \in R_0$ do not reject \h .
\end{itemize}
With this space, the decision rule provides a partition of the sample space into two subsets. However we can work with a sub partition of the sample space ($\mathbb{R}^n$) only if $n$ is small.\\
To reduce the dimensionality of the decision rule (or the partition of the sample space) we can use a sufficient statistic $\tn=T(\ux)$ with $T:\mathbb{R}^n\to \mathbb{R}^m$ with $m\leq n$. The original decision rule is transformed in the new decision rule:
\begin{itemize}
	\item If $\tn \in C_0$ then we reject \h
	\item if $\tn \not \in C_0$ then we do no reject \h
\end{itemize}
Where $C_0$ is our new rejection region.\\
The statistical hypothesis defines a partition of the parametric space $\Theta$.\\
The decision rule defines  a partition of the sample space (or a lower dimensional counterpart).\\
So we can see a test as a link between the two partitions.\\

Confronting the choices done with the test with the real situation it can happen 4 different scenarios:
\begin{center}
	\begin{tabular}{ | c | c | c | c | }
		\hline
		                              & $X\in R_0$ (reject \h)     & $X\not \in R_0$ (accept \h)  \\ \hline
		\h:$\te \in \omega_0$         & $E_1$          & $G_1$            \\ \hline
		$H_1$:$\te \not \in \omega_0$ & $G_2$          & $E_2$            \\
		\hline
	\end{tabular}
\end{center}
	\begin{itemize}
		\item[$G_1$] accept \h when \h is true , so we took the right decision 
		\item[$G_2$] reject \h when \h is false, so we took the right decision
		\item[$E_1$] reject \h when \h is true, so we took the wrong decision
		\item[$E_2$] accept \h when \h is false, so we took the wrong decision
	\end{itemize}
\begin{defi}
	we define the function $\gamma(\theta) $ with $\te \in \Theta$ in this way
	\begin{itemize}
		\item if $\theta \in \Theta_0$   $\gamma(\theta):=\p(\uX \in R_0 ; \theta \in \Theta_0)$
		\item if $\theta\in \Theta_0^c$    $\gamma(\theta):= \p(\ux \in R_0, \theta \not \in \Theta_0)$
	\end{itemize}
\end{defi}
\begin{defi}Some terminology.\\
	$E_1$ is called \textbf{error of the first type} and  $$\alpha:=\p(E_1)=\p(\text{reject \h, \h \ \ true})=\p(\ux \in R_0 , \theta \in \omega_0)$$ $\alpha$ is called \textbf{size of the test}.\\
	$E_2$ is called \textbf{error of the second type} and $$\beta:=\p(E_2)=\p(\text{accept \h, \h \ \ false})=\p(\ux \not \in R_0 , \theta \not \in \omega_0)$$
	note that $\gamma=1-\beta=\p(\ux \in R_0, \theta \not \in \omega_0)$.\\$1-\beta$ is called \textbf{power of the test}.
\end{defi}
\begin{defi}for $\alpha\in [0,1]$ a test with power function $\gamma (\theta)$ is a \textbf{size $\alpha$ test} is :
	$$\sup_{\te \in \Theta_0}\gamma (\te)=\alpha$$
\end{defi}
\begin{defi}
	for $\alpha \in [o,1]$ a test with power function $\gamma(\te)$ is a \textbf{level $\alpha$ test} if 
	$$\sup_{\te \in \Theta_0}\gamma(\te)\leq \alpha$$
\end{defi}
We want to find a rejection region $R_0$ where $\alpha$ and $\beta$ are small.\\
But it is impossible to reduce $\alpha$ and at the same time $\beta$ as shown in the next observation \ref{obs:alpha}
\begin{oss}\label{obs:alpha}
	Fix a rejection region $C_0$  and define 
	\begin{itemize}
		\item$\alpha=\alpha(C_0)$
		\item$\beta=\beta(C_0)$
	\end{itemize}
Suppose that $C_0'$ is a different reject  region such that
$$\alpha(C_0')<\alpha(C_0)$$
Then we want to show that $\beta(C_0')>\beta(C_0)$.\\
Indeed if $\alpha(C_0')<\alpha(C_0)$ then $\p(C_0')<\p(C_0)$ because 
$$\alpha (C_0')=\p(\ux \in C_0'|H_0)< \p(\ux \in C_0|H_0)=\alpha(C_0)$$
so $\p((C_0')^c)>\p(C_0^c)$
$$\beta (C_0')=\p(\ux \not \in C_0'|H_1)\p(\ux \in (C_0')^c|H_1 )> \p(\ux \in C_0^c|H_1)= \p(\ux \not \in C_0|H_1)=\beta(C_0)$$
So we have that if $\alpha$ decrease $\beta$ increase.
\end{oss}
The most dangerous error is the type one error so usually we fix $\alpha$ to control it and look for the reject region $C_0$ which has $\alpha$ as probability of committing an error of the first type an the smallest possible $\beta$ as probability of committing an error of the second type.\\
\begin{eg}
	Consider $\uX=(X_1 ... X_n)$ a random sample from $X\sim Ber(\theta)$ (Flip of a coin).\\
	$\hat \theta_n=\frac{1}{n}\sum_{i=1}^{n}X_i$.\\
	We consider the following test:
	\begin{itemize}
		\item[$H_0$]: $\theta=\frac{1}{2}$
		\item[$H_1$]: $\theta=\frac{2}{3}$
	\end{itemize}
	The easiest way to test this hypothesis is by taking an experiment.\\
	Suppose we film the coin $30$ times, we have:
	$\sumi X_i\sim Bin(30,\theta)$.\\
	Now guess some reject regions
	\begin{itemize}
		\item[$C_0'$] we reject \h if the number of heads is strictly greater than $25$
		\item[$C_0''$] we reject \h if the number of heads is strictly greater than $17$
	\end{itemize}
	In $C_0'$ we expect a large $\beta$ because we will "reject very often".\\
	In $C_0''$ we expect a large $\alpha$ because we will "reject quite rarely".\\
	\begin{itemize}
		\item $\alpha(C_0') = \p \big( \sum_{i=1}^{30} x_i > 25 | \te = \frac{1}{2} \big) = \sum_{M=25}^{30} \binom{30}{M} \big(\frac{1}{2}\big)^M \big( \frac{1}{2} \big)^{30-M}=0.000029...$
		\item $\beta(C_0') = \p \big( \sum_{i=1}^{30} x_i < 25 | \te = \frac{2}{3} \big) = \sum_{M=25}^{30} \binom{30}{M} \big(\frac{2}{3}\big)^M \big( \frac{1}{3} \big)^{30-M}=0.9877...$
		\item $\alpha(C_0'') = \p \big( \sum_{i=1}^{30} x_i > 17 | \te = \frac{1}{2} \big) = \sum_{M=17}^{30} \binom{30}{M} \big(\frac{1}{2}\big)^M \big( \frac{1}{2} \big)^{30-M}=0.18...$
		\item $\beta(C_0'') = \p \big( \sum_{i=1}^{30} x_i < 17 | \te = \frac{2}{3} \big) = \sum_{M=17}^{30} \binom{30}{M} \big(\frac{2}{3}\big)^M \big( \frac{1}{3} \big)^{30-M}=0.166...$
	\end{itemize}
\end{eg}
\begin{defi}
	Let $C$ be the class of tests for testing \h :$\te \in \Theta_0$ versus $H_1$:$\theta \in \Theta_0^c$.\\
	A test in the class $C$, with power function $\gamma(\te)$ is a \textbf{uniformly most powerful (UMP)} class $C$ test if
	$$\gamma(\te)\geq \gamma'(\te)$$
	for every $\theta \in \Theta_0^c$ and every $\gamma'$ that is the power function of a test in the class $C$.\\
	For us the class $C$ will be the class of tests of size $\alpha$. Then we will write \textbf{UMP level $\alpha$ test}
\end{defi}
In most cases this test does not exist. A situation in which this test exist is when $H_0, H_1$ are simple.\\
In other words the requirements of this definition are very strong, in particular the fact that we are considering arbitrary hypothesis. In order to identify the UMP level $\alpha$ test we restrict to the class of simple hypothesis. So we can use the "Neyman-Pearson lemma".
\begin{teo}\label{teo:NP} \textbf{Neyman-Pearson lemma}\\
	Consider the prolem of testing $H_0: \theta=\theta_0$ versus $H_1:\theta =\theta_1$. Let $\lfd$ be the joint densisty function of $\uX=(X_1...X_n)$. Consider the test with the rejection region $R_0$ that satisfies
	\begin{itemize}
		\item[A:]$\ux \in R_0$ if $f_{\ux}(\ux;\theta_1) > k f_{\ux}(\ux;\theta_0)$ and $\ux \in R_0^c$ if $ f_{\ux}(\ux;\theta_1)< k f_{\ux}(\ux;\theta_0)$ for some $k \geq 0$
		\item[B:]$\alpha=\p(\ux\in R_0; \theta_0)$
	\end{itemize}
A is called the \textbf{shape of the rejection region}\\
B is called \textbf{size of the rejection region}.\\
Then
\begin{itemize}
	\item any test that satisfies A and B is a UMP level $\alpha$ test
	\item if exist a test that satisfies A and B with $k\geq 0$ then every UMP level $\alpha$ test is a size $\alpha$ test (B) and every UMP level $\alpha$ test satisfies A except on a set such that:
	$$\p(\ux \in A;\theta_0)=\p(\ux\in A; \theta_1)=0$$ 
\end{itemize}
\end{teo}  
\begin{eg}
	Let $\uX=(X_1 ... X_n)$ a random sample from $X\sim N(\te, \sigma^2)$, where $\sigma^2$ is known. We test:
	$$H_0: \theta= \theta_0 \ \ \ \ vs\ \ \ \ \ H_1: \theta= \theta_1 $$
	with $\te_1>\te_0$ \\
	$$\frac{\mathcal{L}(\theta_1;\ux)}{\mathcal{L}(\theta_0;\ux)}\geq c$$
	with $\alpha=\p(\ux \in R_0;\theta_0)$ fixed\\
	
	\[
	\begin{split}
	\implies 
	&
	\exp \bigg\{ -\frac{1}{2\sigma^2}\bigg[ \sumi (x_i- \theta_1)^2 - \sumi(x_i-\theta_0)^2 \bigg] \bigg\} \geq c \\
	&\iff \exp \bigg\{ -\frac{1}{2\sigma^2}\bigg[ \sumi x_i^2 - n\theta^2_1 -2\theta_1\sumi x_i-\sumi x_i^2 - n \theta_0^2 + 2\theta_0 \sumi x_i \bigg] \bigg\} \geq c\\
	&\iff \exp \bigg\{ -\frac{1}{2\sigma^2}\bigg[ - n\theta^2_1 -2\theta_1\sumi x_i - n \theta_0^2 + 2\theta_0 \sumi x_i \bigg] \bigg\} \geq c\\
	&\iff \exp \bigg\{ -\frac{1}{2\sigma^2}\bigg[ n(\theta_1^2 -\theta_0^2)- 2 \sumi x_i (\theta_1 - \theta_0) \bigg] \bigg\} \geq c\\
	&\iff \exp \bigg\{ -\frac{1}{2\sigma^2}(\theta_1 - \theta_0)\bigg[ n(\theta_1 +\theta_0)- 2 \sumi x_i \bigg] \bigg\} \geq c\\
	&\iff -(\theta_1 - \theta_0)\bigg[ n(\theta_1 +\theta_0)- 2 \sumi x_i \bigg] \bigg\} \geq 2\ln(c)\sigma^2\\
	&\iff  n(\theta_1 +\theta_0)- 2 \sumi x_i \leq - \frac{2\ln(c)\sigma^2}{\theta_1-\theta_0}\\
	&\iff -2 \sumi x_i \leq  - \frac{2\ln(c)\sigma^2}{\theta_1-\theta_0} -  n(\theta_1 +\theta_0)\\
	&\iff  \sumi x_i \geq  - \frac{\ln(c)\sigma^2}{\theta_1-\theta_0} -  \frac{n}{2}(\theta_1 +\theta_0)
	\end{split}
	\]
	So $\bar X_n \geq c$ is the shape of the rejection region. Where now $c$ is: $- \frac{\ln(c)\sigma^2}{\theta_1-\theta_0} -  \frac{n}{2}(\theta_1 +\theta_0)$  So to fin the value of $c$ given $\alpha$ we remember that $\bar X_n\sim N\big( \theta, \frac{\theta}{n} \big)$ so:
	\[
		\begin{split}
		\alpha&=\p(\bar X_n \geq c| H_0: \te =\te_0)\\&=\p(\frac{\bar X_n-\theta_0}{\sigma/\sqrt{n}}\geq \frac{c-\theta_0}{\sigma/\sqrt{n}})\\&=\p(Z \geq \frac{c-\theta_0}{\sigma/\sqrt{n}})\\&=1- \Phi\big( \frac{c-\theta_0}{\sigma/\sqrt{n}} \big)
		\end{split}
	\]
	Where $Z$ is the standard Gaussian.
	In the end $c=\theta_0 + z_{\alpha} \frac{\sigma}{\sqrt{n}}$.\\
	According to the Neyman-Pearson lemma \ref{teo:NP}
	$$\big\{ \bar X_n \geq \theta_0 + z_\alpha \frac{\sigma}{\sqrt{n}} \big\} \ \ \ \ \ \ \text{ is the UMP level $\alpha$}$$
\end{eg}
\begin{eg}
		Let $\uX=(X_1 ... X_n)$ a random sample from $X\sim Ber(\te)$.
		 We test:
		$$H_0: \theta= \theta_0 \ \ \ \ vs\ \ \ \ \ H_1: \theta= \theta_1 $$
		With $\theta_1>\theta_0$
		\[
			\begin{split}
		\implies & \frac{\mathcal{L}(\theta_0;\ux)}{\mathcal{L}(\theta_1;\ux)}\leq k\\
		&\frac{\theta_0^{\sumi x_i}(1-\theta_0)^{n-\sumi x_i}}{\theta_1^{\sumi x_i}(1-\theta_1)^{n-\sumi x_i}}=\bigg( \frac{\theta_0}{\theta_1} \bigg)^{\sumi} \bigg(  \frac{1-\theta_0}{1-\theta_1} \bigg)^{n-\sumi x_i}\leq k\\
		& \ln\bigg(\frac{\te_0}{\te_1}\bigg)\sumi x_i + \bigg(n-\sumi x_i\bigg)\ln\bigg(\frac{1-\theta_0}{1-te_1}\bigg)\leq \ln k\\
		&\ln\bigg(\frac{\te_0}{\te_1}\bigg)\sumi x_i + n \ln\bigg( \frac{1-\theta_0}{1-te_1} \bigg)-\ln\bigg(\frac{1-\theta_0}{1-te_1}\bigg)\sumi x_i\leq \ln k\\
		&\sumi x_i \bigg[ \ln\bigg(\frac{\te_0}{\te_1}\bigg)   -\ln\bigg(\frac{1-\theta_0}{1-te_1}\bigg) \bigg]	\leq \ln k - n \ln\bigg( \frac{1-\theta_0}{1-te_1} \bigg)	
			\end{split}
		\]
		because $ \ln\bigg(\frac{\te_0}{\te_1}\bigg)$   and $ -\ln\bigg(\frac{1-\theta_0}{1-te_1}\bigg)$ are $\leq 0$ we have
		$$\sumi x_i\geq k$$
		where $\sumi x_i \sim Bin(\te, n)$.\\
		$\alpha =\p \bigg( \sumi x_i \geq k ; \te =\te_0 \bigg)=\sum_{r=k}^n \binom{n}{r} \theta_0^r (1-\theta_0)^{n-r}$.\\
		We have $k\in \mathbb{N}$ so we can not find a  $k$ such that $\p \bigg( \sumi x_i \geq k ; \te =\te_0 \bigg)$ is exactly equal to $\alpha$. So we look for a value that provides a value very close to $\alpha$
\end{eg}
\begin{eg}\label{eg:exponentialf}
		Let $\uX=(X_1 ... X_n)$ a random sample from $X\sim EF()$ (Exponential Family) so $f_X(x;\te)=\exp \{ A(x) Q(\te)+ C(x)- K(\te) \}$. Assume $Q$ monotone increasing. We want to test 
			$$H_0: \theta= \theta_0 \ \ \ \ vs\ \ \ \ \ H_1: \theta= \theta_1 $$
			With $\te_0< \te_1$. Apply the Neyman-Pearson lemma \ref{teo:NP}
			
			\[
			\begin{split}
			\implies & \frac{\mathcal{L}(\theta_0;\ux)}{\mathcal{L}(\theta_1;\ux)}\leq k\\
			&\iff \frac{\prod_{i=1}^{n} \exp \{ A(x_i) Q(\te_0)+ C(x_i)- K(\te_0) \} }{\prod_{i=1}^{n} \exp \{ A(x_i) Q(\te_1)+ C(x_i)- K(\te_1) \}}\leq k\\
			&\iff  \exp \{  Q(\te_0) \sumi A(x_i)+ \sumi C(x_i)- nK(\te_0) - Q(\te_1)  \sumi A(x_i)+\sumi C(x_i)+nK(\te_1) \}\leq k\\
			&\iff  \exp \{  Q(\te_0) \sumi A(x_i) - nK(\te_0) - Q(\te_1)  \sumi A(x_i)+nK(\te_1) \}\leq k\\
			&\iff  \exp \{   \sumi A(x_i)(Q(\te_0) -  Q(\te_1)) - n(K(\te_0) - K(\te_1)) \}\leq k\\
			&\iff    \sumi A(x_i)(Q(\te_0) -  Q(\te_1)) - n(K(\te_0) - K(\te_1)) \leq \ln k\\
			&\iff    \sumi A(x_i)(Q(\te_0) -  Q(\te_1))  \leq c\\
			&\iff \sumi A(X_i) \geq c
			\end{split}
			\]
			where in the last passage we used $	(Q(\te_0) -  Q(\te_1))< 0$ because $\te_1 > \te_0$ and $Q$ is strictly increasing.\\
			$\sumi A(X_i) \geq c$ is the shape of the reject region for an arbitrary exponential family with $Q$ strictly increasing
\end{eg}

\begin{proof} of the Neyman-Pearson lemma \ref{teo:NP}
	Any test that satisfies $B$  is a size  $\alpha$ test, and hence a level $\alpha$ test because
	$$\sup_{\theta \in \Theta_0} \p(\ux \in \ R)=\alpha$$
	Consider a test function in the sample space that is $1$ if $\ux\in R$ and $0$ if $\ux\in R^C$.
	\begin{itemize}
		\item let $\Phi(\ux)$ be a test function that satisfies $A$ and $B$
		\item let $\Phi'(\ux)$ be a test function (dichotomic) of any otherlevel $\alpha$ test.
		\item let $\gamma(\theta)$ and $\gamma'(\te)$ be the corresponding power functions (when $\theta \in \Theta_0^c$)
	\end{itemize}
Observe that 
$\Phi(\ux)=1$ if $\ux\in R$ if $f_{\uX}(\ux;\te_1) > k f_{\uX}(\ux;\te_0)$\\
$\Phi(\ux)=0$ if $\ux\in R^c$ if $f_{\uX}(\ux;\te_1) < k f_{\uX}(\ux;\te_0)$\\
From this we get
$$(\Phi(\ux)-\Phi'(\ux))(f_{\uX}(\ux;\te_1)-k f_{\uX}(\ux;\te_0))\geq 0$$
Then
$$\int (\Phi(\ux)-\Phi'(\ux))(f_{\uX}(\ux;\te_1)-k f_{\uX}(\ux;\te_0))=\gamma(\theta_1)-\gamma'(\theta_1)-k[\gamma(\theta_0)- \gamma(\theta_1)]\geq 0$$
Now observe that $\Phi'$ is a level $\alpha$ test and $\phi$ is a size $\alpha$ test, this implies that 
$$0	\leq \gamma(\theta_1) -\gamma'(\theta_1) - k \gamma(\theta_0) -\gamma'(\theta_0)\leq \gamma(\theta_1) -\gamma'(\theta_1)$$
$\implies 0\leq \gamma(\theta_1) -\gamma'(\theta_1)$\\
$\implies \gamma(\theta_1) \geq \gamma'(\theta_1)$
We proved the test defined by Neyman-Pearson condition $A,B$ is a UMP level $\alpha$ test.\\
Second part\\
Let $\Phi'$ be a test function for every UMP level $\alpha$ test. By the previous part, that is the test $\Phi$ satisfying $A;B$ is also a UMP level $\alpha$ test. This implies 
$$\gamma(\theta_1)=\gamma'(\theta_1)$$
This implues that for any $k>0$
$$\alpha - \gamma'(\theta_0)=\gamma(\theta_0) - \gamma'(\theta_1)\leq 0$$
Now since $\Phi'$ is a  level $\alpha$ test we have that
$$ \gamma'(\theta_0)\leq \alpha \implies  \gamma'(\theta_0)=\alpha$$
This implies that $\Phi'$ is a size $\alpha$ test. Finally Recall that 
$$(\Phi(\ux)-\Phi'(\ux))(f_{\uX}(\ux;\te_1)-k f_{\uX}(\ux;\te_0))\geq 0$$
if we integrate over the sample  space, we have 0 only if $\Phi'$ satisfies assumption A. Except on a set A of probability 0:
$$\int_{A}f_{\uX}(\ux;\te_1)dx=0 \ \ \ i=0,1$$
\end{proof}
We will see extensions of the Neyman-Pearson Lemma for
\begin{itemize}
	\item unidirectional hypothesis (robin test)
	\item arbitrary hypothesis (likelihood ratio test)
\end{itemize}
\begin{corol}\label{cor:NP}
	(link between Neyman-Pearson Lemma and sufficiency)\\
	Consider the hypothesis of the Neyman-Pearson Lemma \ref{teo:NP}. Suppose $T(\uX) $ a sufficient statistic for the parameter $\theta$. Denote by $f_{\uX}(\ux; \te_i) $ for $i=0,1$ the joint density distribution function of the random sample and by $g(t;\te_i), i=0,1$ the density function of the sufficient statistic.\\
	Then any test based on $T$ whit rejection region $S$ (a subset o the sample space defined by $T$) is a UMP level  $\alpha$ test if it satisfies the two following:
	\begin{itemize}
		\item $t\in S$ if $g(t;\theta_1)>k g(t;\te_0)$
		\item $t\in S^x$ if $g(t;\te_1)< k g(t; \te_0)$
	\end{itemize}  
	for some $k \geq 0$, where $\alpha = \p(T\in S; \theta_0)$
\end{corol}
\begin{proof}
	In terms of the random sample $(X_1...X_n)$, the test based on the sufficient static  $T$ has the following rejection region $$R=\{\ux : T(\ux) \in S\}$$
	By the factorization theorem the distribution of the random sample  $(X_1...X_n)$ can be written as 
	$$f_{\uX}(\ux;\te_i)=g(T(\ux);\te_i)h(\ux) \ \ \ \ i=0,1$$
	Consider the condition $t\in S$ if $g(t;\te_1)> g(t;\te_0)$ and multiply both sides of the inequality by the non negative function $h$. We see that the rejection region $R$ satisfies $\ux\in R$ if 
	$$f_{\uX}(\ux;\te_1)=g(T(\ux);\te_1)h(\ux)< g(T(\ux);\te_0)h(\ux)=f_{\uX}(\ux;\te_0)$$
	By the size condition we have
	$$	\p (\uX \in R; \te_0)=\p(T(\uX)\in S; \te_0)=\alpha$$
	At this point we remember the Neyman-Pearson Lemma \ref{teo:NP} and the test based on the  sufficient statistic $T$ is a UMP level $\alpha$ test.
\end{proof}
\begin{eg}Remember the example \ref{eg:exponentialf} of with the exponential family
	$f_X(x;\theta)=\exp \{ A(x) Q(\te)+ C(x)- K(\te) \}$ we know a sufficient statistic is $\sumi A(X_i)$
	We want to test 
	$$H_0: \theta= \theta_0 \ \ \ \ vs\ \ \ \ \ H_1: \theta= \theta_1 $$
	Assume $Q$ monotone increasing and $\te_1>\te_0$. So $Q(\te_1)> Q(\te_0)$.\\
	The shape of the rejection region is 
	$$\sumi A(x_i)\geq c$$
	which is a   function of a sufficient statistic for $\theta$
\end{eg}
In general, by the previous corollary \ref{cor:NP} the shape of the reject region defined by the Neyman-Pearson Lemma is a function of the sufficient statistic for the parameter of interest.